\documentclass[a4paper,10pt]{book}
%------------------------------------------------------------------------------------------
%                                   Cours de Physique
%------------------------------------------------------------------------------------------
\usepackage[latin1]{inputenc}
\usepackage[francais]{babel}
\usepackage[T1]{fontenc}
\usepackage{color,amsmath,amssymb} %fontes math�matiques
\usepackage{pst-all}               %postscript
\usepackage{graphicx}              %encore du postscript
\usepackage{ulem}                  %soulignages
\usepackage{/usr/share/texmf/tex/generic/perso/maths}
\usepackage{makeidx}               %index
\makeindex
\newcommand{\ml}{m_{\ell}}
\title{Cours de Physique}
\author{J.M~\textsc{Mercier}\\ Mise en forme : R.Portalez}
\date{12 d�cembre 2005}
\declarecountsphi
\taillebook
%
%
\begin{document}
%
%
\maketitle
\tableofcontents
\chapter{Pr�liminaires}
On part du principe qu'en abordant ce cours on conna�t d�j� bien toutes les math�matiques n�c�ssaires
� sa compr�hension, � savoir l'analyse vectorielle, les fonctions de plusieurs variables, les espaces
euclidiens, les courbes du plan et de l'espace, les surfaces dans $\rr[3]$. Ainsi il est vain de
l'aborder en fin de sup~: il ne s'agit que d'un r�sum� de cours destin� aux r�visions pour l'oral,
c'est-�-dire qu'on doit d�j� avoir une vue d'ensemble du programme pour tout bien saisir. De plus, 
ce cours va parfois au del� du programme par exemple avec la physique statistique ou les grandeurs 
extensives.\par
Volontairement, je n'y ai pas fait figurer l'�lectrocin�tique et l'optique g�o\-m�\-tri\-que ainsi qu'un 
certain nombre d'autres notions qui sont soit faciles, soit d�pendant directement du cours de sup.\par
Pour chaque r�sultat ou th�or�me, la d�monstration n'est pas int�gralement r�dig�e dans un but de 
concision. Toutefois afin d'assurer la rigueur du travail effectu�, l'id�e et les �tapes de la 
d�monstration sont rappel�es � chaque fois. De plus, les raisonnements les plus difficiles comme la 
d�monstration du th�or�me de Faraday en induction sont tr�s d�taill�s car il est difficile pass� un 
certain temps de s'en rappeler.\par
Il y a dans ce cours un certain nombre d'exercices. La plupart sont difficiles car l'�nonc� est 
volontairement vague. Ils n�cessitent une parfaite connaissance du cours pour les r�ussir.
\partie{Grandeurs extensives.}
La notion de grandeur extensive peut �tre compar�e � la population d'un pays, dont la variation peut
�tre calcul�e � partir de deux termes, l'un correspondant � l'immigration et l'�migration, l'autre
correspondant aux naissances et aux d�c�s au sein du pays. 
\sspartie{D�finition d'une grandeur extensive.}
Une grandeur physique est \textsl{extensive} d�s qu'on peut lui attribuer une densit� volumique, 
not�e $x_v$ ou de mani�re �quivalente, une densit� massique, $x$ telle que $\rho x = x_v$, o\`u $\rho$ 
est la masse volumique.
\index{Grandeur extensive}
Une telle d�finition montre que le volume, la masse totale, la charge totale, l'�nergie cin�tique sont
des grandeurs extensives. Ainsi la quantit� d'une grandeur $X$ pr�sente dans un volume $\tau$ est
donn�e par l'int�grale~:\par
$$X = \integrale{\tau}{}{x_v \dd{\tau}}$$\par
\sspartie{Syst�me.}\index{systeme@Syst�me}
Un syst�me physique par d�finition est l'int�rieur d'un surface ferm�e. Celle-ci �volue �ventuellement 
dans le temps.\par
Ainsi il n'est pas n�cessairement ferm� puisque de la mati�re ou d'autres grandeurs peuvent traverser la 
paroi. Par exemple, quand on m�lange deux gaz, on ne peut pas prendre comme syst�me l'un des deux gaz
puisque dans l'�tat final on ne peut pas d�finir de fronti�re.\par
Par une relation de Chasles, si $X_1$ et $X_2$ d�signe la quantit� de $X$ au sein de deux syst�mes
disjoints, la quantit� de $X$ dans leur r�union est $X_1 + X_2$. Une telle d�finition est tr�s utile
mais ne constitue pas une d�finition, comme on peut le voir dans certains ouvrages.
\spartie{Grandeur extensive re�ue par un syst�me.}
\sspartie{D�finitions.}
Une grandeur extensive peut g�\-n�\-ra\-le\-ment tra\-ver\-ser la fron\-ti\-�re d'un syst�me. Un tel 
transfert est caract�ris�e par la quantit� $X^{r}$ appel�e grandeur re�ue par le syst�me, compt�e dans 
le sens de la normale rentrante. Un syst�me est dit isol� vis-�-vis de $X$ si $X^r = 0$ en toute 
circonstance~:\par
\vspace{3pt}
\centerline{\includegraphics*[width=6cm,height=3cm]{images/systeme.eps}}\par
\sspartie{Courant d'un grandeur extensive.}
Le transfert d'une grandeur extensive $X$ est d�crit localement par un vecteur courant $\ve{J_X}$ tel 
que $\dd{t}\ve{J_X}\cdot\ve{n}\dd{S}$ est la quantit� de $X$ qui traverse $\dd{S}$ \textsl{fixe} dans 
le sens de la normale $\ve{n}$ pendant $\dd{t}$. Ainsi la quantit� de $X$ re�ue � travers $S$ pendant 
$\dd{t}$ est :
\encadre{$\delta X^r = \dd{t}\integrale{S}{}{\ve{J_X}\cdot\ve{n}\dd{S}}$}
\sspartie{Courant non-convectif.}
\index{Courant non-convectif}
On appelle convectif tout transfert de $X$ par d�placement de la mati�re et non-convectif tout autre 
transfert pouvant exister dans un r�f�rentiel o\`u la mati�re est immobile.\par
Si la fronti�re du syst�me �pouse le mouvement de la mati�re, elle est qualifi�e de convective.\par
La mati�re qui traverse $\dd{S}$ pendant $\dd{t}$ dans le sens de $\ve{n}$ est celle qui se trouve dans
ce cylindre, qui peut �tre d�fini car la vitesse $\ve{v}$ est localement uniforme.\par
\centerline{\includegraphics*[width=6cm,height=3cm]{images/cylindre.eps}}\par
\rmq Le mouvement de la mati�re d�pend en fait du constituant et on fait alors ce raisonnement pour un 
seul constituant. Ensuite on somme pour tous les constituants pour obtenir le r�sultat en d�finissant~:
\par
$$\rho = \sum_i\rho_i\;\mathrm{et}\;\ve{v} = \frac{\sum_i\rho_i\ve{v_i}}{\sum_i\rho_i}$$\par
\noindent C'est la vitesse barycentrique des divers constituants.\par
Le volume de ce cylindre vaut $\delta V = \ve{v}\dd{t}\cdot\ve{n}\dd{S}$ et il contient $x_v\delta V$
de $X$. On en d�duit la contribution convective $x_v\ve{v}$ au courant total. Le courant non-convectif
s'�crit alors :
\encadre{$\ve{J_X^{\star}} = \ve{J_X} - x_v\ve{v}$}
On obtient alors avec $\ve{J} = \rho\ve{v}$ que le transfert de masse est purement convectif.
\spartie{Grandeur extensive produite.}
\sspartie{D�finition.}
Une grandeur extensive peut aussi �tre produite au sein de la fronti�re du syst�me, de m�me qu'il y a 
des naissances et d�c�s dans les fronti�res d'un pays. Cette production est donn�e par une quantit� 
alg�brique $X^p$.
\sspartie{Source d'une grandeur extensive.}
La production de $X$ dans un �l�ment de volume $\dd{\tau}$ entre $t$ et $t+\dd{t}$ est donn�e par 
$\sigma_X\dd{\tau}\dd{t}$, de sorte que la production (alg�brique) globale s'�crive :
\encadre{$\dd{t}\integrale{\tau}{}{\sigma_X \dd{\tau}}$}
\sspartie{Conservation.}\index{Conservation}
Une grandeur est dite \textsl{conservative} si quoi qu'il arrive sa source est nulle.\par
Il en est ainsi de l'�nergie totale (cf~: thermodynamique) ou plus simplement du volume. Il faut faire
attention avec ce terme ``conservative'' car il est souvent galvaud�. Dans la suite, on l'emploiera
toujours dans ce sens bien pr�cis.
\spartie{Bilan d'une grandeur extensive.}
\sspartie{Bilan global.}
On �crit simplement qu'entre deux instants $t_1$ et $t_2$, la variation de $X$ au sein du syst�me est la
somme de ce qui est produit et de ce qui est re�u :\par
$$X_2 - X_1 = X^r + X^p$$\par
Entre deux instants voisins le bilan s'�crit~:\par
$$\dd{X} = \delta X^r + \delta X^p$$\par
Pour tout syst�me on conna�t l'expression de la production par contre ce qui est re�u d�pend du syst�me
consid�r� selon qu'il est � fronti�re fixe ou convective.
\begin{itemize}
\item{Fonti�re fixe~:}
On a directement le bilan~:
\encadre{$\dd{X} = \dd{t}\integrale{S}{}{\ve{J_X}\cdot\ve{n}\dd{S}} + \dd{t}\integrale{\tau}{}{\sigma_X
               \dd{\tau}}$}
\item{Fronti�re convective~:}
Cette fois la quantit� de $X$ tranversant $\dd{S}$ dans le sens de $\ve{n}$ pendant $\dd{t}$ est donn�e
par le courant non-convectif~:
\encadre{$\dd{X} = \dd{t}\integrale{S}{}{\ve{J_X^{\star}}\cdot\ve{n}\dd{S}} + \dd{t}\integrale{\tau}{}{
            \sigma_X\dd{\tau}}$}
\end{itemize}
\sspartie{Bilan local.}
\begin{itemize}
\item{Fronti�re fixe~:}
Une simple application du th�or�me d'Ostrogradsky fournit l'expression du bilan global entre $t$ et 
$t+\dd{t}$~:\par
$$\integrale{\tau}{}{(\div\ve{J_X}+\dpa{x_v}{t}-\sigma_X)\dd{\tau}} = 0$$\par
Et comme cette expression est valable quelle que soit la taile du syst�me, on en d�duit l'�quation 
locale du premier type~:
\encadre{$\div \ve{J_X} + \dpa{x_v}{t} = \sigma_X$}
\item{Fronti�re convective~:}
De mani�re analogue on obtient l'�quation locale du second type~:
\encadre{$\div\ve{J_X^{\star}} +\rho\df{x}{t} = \sigma_X$}
O\`u $\df{}{t} = \dpa{}{t}+\ve{v}\cdot\grad$ est l'op�rateur ``d�riv�e particulaire'', et $x$ est la 
densit� massique de $X$ tandis que $\rho$ est la masse volumique locale du syst�me.
\end{itemize}
\index{derivee@D�riv�e particulaire}
\index{Bilan local}
\exercice En faisant le bilan entre $t$ et $t+\dd{t}$ de la grandeur $X$ contenue dans un syst�me �
fronti�re fixe, d�montrer l'�quation locale du premier type puis, apr�s avoir fait de m�me avec un 
syst�me � fronti�re convective et l'�quation du second type, d�montrer l'�quivalence entre ces deux
�quations.
\exercice En partant de l'�quation locale traduisant la conservation de la masse~: 
$\div(\rho \ve{v}) + \dpa{\rho}{t} = 0$ (qu'on red�montrera), d�montrer la relation
$\div(-\ve{v})+\rho\df{w}{t} = 0$ o\`u $w$ est le volume massique, puis en consid�rant une particule de
fluide de masse constante, d�montrer la relation : $\dd{(\delta V)} = \div\ve{v}\delta V\dd{t}$ et en
d�duire � quelle condition un fluide est incompressible.
\exercice En appliquant le th�or�me de la r�sultante cin�tique � une particule de fluide de masse 
constante d�montrer l'�quation d'Euler~: $\rho\df{\ve{v}}{t} = -\grad P + \rho \ve{f}$, o\`u $\ve{f} = 
\ve{f'} -\grad e_p$ d�signe la densit� massique des forces de champ subies par le fluide, d�compos�e
en un terme d�rivant d'une �nergie potentielle est un autre terme. En faisant le produit scalaire par
$\ve{v}$ de cette �quation, en d�duire l'�quation du second type traduisant le bilan local d'�nergie
m�canique.\index{Equation!euler@d'Euler}
%%%%%%%%%%%%%%%%%%%%%%%%%%%%%%%%%%%%%%%%%%%%%%%%%%%%%%%%%%%%%%%%%%%%%%%%%%%%%%%%%%%%%%%%%%%%%%%%%%%%%%%
\partie{Physique statistique.}
\spartie{Etude statistique du gaz parfait.}
\sspartie{Rappels.}
On �tudie une particule de gaz parfait. On introduit pour cela les coordonn�es sph�riques~:\par
\centerline{\includegraphics*[width=6cm,height=6cm]{images/spheriques.eps}}\par
Il est connu que l'�l�ment de volume vaut $\dd{{}^3 r} = r^2\sin{\theta}\dd{\theta}\dd{\phim}$.
Afin d'�tudier les vitesses, on fait de m�me avec les vitesses et on rappelle l'�l�ment d'angle
solide~: $\delta\Omega = \sin\theta\dd{\theta}\dd{\phim}$.
\sspartie{Loi de Boltzmann.}\index{Loi!boltzmann@de Boltzmann}
Dans un gaz en �quilibre, la probabilit� pour une particule de se trouver en 
$(\ve{r}, \ve{v}, \ldots)$ � $(\dd{{}^3 r},\dd{{}^3 v}, \ldots)$ pr�s vaut~:
\encadre{$P(\ve{r},\ve{v},\ldots) = C exp\left\lbrack -\dfrac{e_m(\ve{r},\ve{v},\ldots)}{k T}\right
                                                         \rbrack\dd{{}^3 r}\dd{{}^3 v}\ldots$}
O\`u $k$ est la constante de Boltzmann.
\rmq Cette probabilit� a donc une densit�. On verra que dans le cas le plus simple du gaz parfait, c'est
une densit� gaussienne. La position d'une particule dans l'espace des vitesses est donc une variable 
al�atoire gaussienne centr�e. Cela justifie en partie la pr�dominance de cette loi de probabilit� en 
physique comme en math�matiques.
\rmq Puisque le gaz est en �quilibre, la temp�rature est uniforme (cf : thermodynamique).\par
$C$ est une constante servant � normaliser l'int�grale, dont l'expression d�pend du gaz consid�r�.
\sspartie{Probabilit� pour la localisation, dans un gaz parfait :}
Calculons la probabilit� pour une mol�cule de gaz parfait d'�tre � $\ve{r}$ � $\dd{{}^3r}$ pr�s. Dans 
un gaz parfait, on n'a pas d'interaction mutuelle entre les particules, et si l'on suppose 
n�gligeables les actions ext�rieures, on a : $e_m = e_c = \frac{1}{2}m v^2$. En sommant alors
l'expression de la probabilit� sur tout l'espace des vitesses, on a :\par
$$P(\ve{r}) = C\left\lbrack \integrale{-\infty}{+\infty}{e^{-\frac{mv_x^2}{2kT}}\dd{v_x}}
                            \integrale{-\infty}{+\infty}{e^{-\frac{mv_y^2}{2kT}}\dd{v_y}}
                            \integrale{-\infty}{+\infty}{e^{-\frac{mv_z^2}{2kT}}\dd{v_z}}\right\rbrack$$
\par
On en d�duit imm�diatement que la probabilit� pour une particule de se trouver dans un volume $\delta V$
au sein d'une enceinte de volume $V$ vaut 
\encadre{$P = \dfrac{\delta V}{V}$}
Et la r�partition des particules est uniforme au sein de l'enceinte.
\sspartie{Probabilit� pour la vitesse, dans un gaz parfait :}
Cette fois on int�gre sur tout l'espace ordinaire et en notant $a = \dfrac{m}{2kT}$, on trouve :
\encadre{$P(\ve{v}) = Ae^{-av^2}\dd{{}^3v}$}
La constante $A$ peut �tre calcul�e en sommant sur l'espace des vitesses et on obtient au final :\par
\encadre{$P(\ve{v}) = \left(\dfrac{a}{\pi}\right)^{\frac{3}{2}}e^{-av^2}\dd{{}^3v}$}
On en d�duit que cette probabilit� ne d�pend pas de $\phim, \theta$ : on parle d'isotropie dans l'espace
des vitesses.
\sspartie{Vitesse, vitesse quadratique moyenne :}
$<v> \;=\;\integrale{E_v}{}{vP(v)} = \left(\dfrac{a}{\pi}\right)^{\frac{3}{2}}%
                                      \integrale{0}{\infty}{\integrale{0}{\pi}{\integrale{0}{2\pi}{
                                               v^3e^{-av^2}\dd{v}\sin\theta\dd{\theta}\dd{\phim}}}}$
Apr�s calculs, on trouve alors :
\encadre{$<v> = \sqrt{\dfrac{8 k T}{m\pi}}$}
De m�me, on obtient la vitesse quadratique moyenne :
\encadre{$\sqrt{<v^2>} = \sqrt{\dfrac{3kT}{m}}$}
D'un point de vue ordre de grandeurs, on remarque que la vitesse moyenne des particules est de l'ordre
de la vitesse du son dans le gaz.
\sspartie{Energie interne :}
Par d�finition, $U = e_{c,i} + e_{p,i}$. Si le nombre $N$ de particules est suffisamment grand, on peut
�crire :\par
$e_{c,i} = N<e_c> = \frac{N}{2}m<v^2> = \frac{3}{2}Nk T$\par
Et comme il n'y a pas d'interactions, $e_{p,i} = 0$ soit :
\encadre{$U = \dfrac{3}{2}nRT$}
\sspartie{Force de pression :}
\centerline{\includegraphics*[width=9cm,height=3cm]{images/particules_incidentes.eps}}\par
Calculons la force moyenne exerc�e par les $\delta N$ particules arrivant sur $\dd{S}$ pendant $\delta t$ et 
montrons qu'elle a pour expression~:
\encadre{$\ve{F} = \dfrac{1}{\delta t}\serie{i = 1}{\delta N}{\ve{p_i}-\ve{p'_i}}$}
o\`u $\ve{p_i}$ est la quantit� de mouvement de la particule $i$ avant son interaction avec la paroi et 
$\ve{p'_i}$ celle apr�s interaction. En effet $\ve{f_i} = \int_{t_i}^{t_i'}\dd{\ve{f_i}}$ Or le TRC nous 
dit que~: $\df{\ve{p_i}}{t} = \dd{\ve{f_i}}$, l'intervalle de temps �tant infinit�simal, on a le r�sultat.\par
Puis en utilisant l'isotropie dans l'espace des vitesses :
$F_y = F_x = 0$ et $F_z = \frac{2 m}{\delta t}\sum_i v_{i,z}$.\par
Puis, en notant $\nu$ la densit� particulaire (uniforme), la contribution � cette somme de la part des
particules ayant une vitesse $\ve{v}$ (sch�ma) vaut :\par
\centerline{$\underbrace{v\cos\theta}_{\mathrm{contribution}}\;\;$%
$\underbrace{\nu\cdot v\dd{t}\cos\theta\dd{S}}_{\mathrm{nombre\;de\;particules}}\;\;$%
$\underbrace{P(\ve{v})\dd{{}^3v}}_{\mathrm{Proba\;d'avoir\;cette\;vitesse}}$}\par
En int�grant sur $v$ de $0$ � $\infty$, sur $\phim$ de $0$ � $2\pi$ et sur $\theta$ de $0$ � 
$\frac{\pi}{2}$, on trouve le r�sultat suivant :
\encadre{$\ve{F} = P\dd{S}\vz$ avec $P = \nu k T$}
\sspartie{Equation d'�tat :}
Un calcul fournit alors l'�quation d'�tat du gaz parfait~:
\encadre{$PV = nRT$}
\spartie{Constante di�lectrique d'un gaz :}
Un gaz est constitu� de particules quasi-ponctuelles dont on n�glige les interactions mutuelles. Chacune
de ces particules poss�de un moment �lectrique $\ve{p}$ de norme constante. On plonge le gaz dans un
champ �lectrique constant et on suppose le gaz � l'�quilibre.\par
\exercice Calculer la probabilit� pour une particule d'avoir un moment �lectrique $\ve{p}$, la valeur 
moyenne $<\ve{p}>$ du moment dipolaire puis le vecteur polarisation $\ve{P}$ du gaz.


\chapter{Electromagn�tisme}
\partie{Electrostatique.}
\spartie{Th�or�me fondamental.}
\sspartie{Enonc�.}
Toute distribution stationnaire de charge engendre un champ vectoriel $\ve{E}$ appel� champ �lectrostatique 
tel que la force subie par une charge ponctuelle $q$ de la part de cette distribution ait pour expression :
\index{Force!lorentz@de Lorentz}
\encadre{$\ve{f} = q\ve{E}$}
Le champ �lectrostatique satisfait aux �quations locales :\index{Champ!electrostatique@Electrostatique}
\encadre{$\begin{array}{c}
          \rot \ve{E} = \ve{0}\\
          \div \ve{E} = \dfrac{\rho}{\eps_0}
          \end{array}$}
\sspartie{Cons�quences imm�diates.}
\point{Potentiel scalaire~:} $\rot \ve{E} = \ve{0}$~: le champ �lectromagn�tique est � circulation conservative. Il d�rive donc
d'un potentiel scalaire $V$ appel� potentiel �lectrostatique :\index{Potentiel!electrostatique@Electrostatique}
\encadre{$\ve{E} = -\grad V $}
A cette �quation locale correspond l'�quation int�grale~:
\encadre{$\integrale{L}{{}}{\ve{E}\cdot\ve{t}\dd{L}} = V(A) - V(B)$}
\rmq Le choix de $-\grad V$ se justifie par le fait que la plupart des formules de physique en sont 
simplifi�es, par exemple dans les expressions du potentiel.\par
Le potentiel �lectrostatique n'est d�fini qu'� une constante additive pr�s, il n'est pas mesurable.\par
\point{Th�or�me de Gauss~:}
On sait que~: $\div \ve{E} = \dfrac{\rho}{\eps_0}$~: � cette �quation locale correspond l'�quation 
int�grale~:\par
\encadre{$\integrales{S}{{}}{\ve{E}\cdot\ve{t}\dd{S}} = \dfrac{q_S}{\eps_0}$}\par
C'est le th�or�me de Gauss.\par\index{theoreme@Th�or�me!gauss@de Gauss}
\point{Equation de Poisson~:} Des deux �quations locales~: $\ve{E} = -\grad V $ et $\div \ve{E} = \dfrac{\rho}{\eps_0}$, on 
tire l'�quation locale~:
\encadre{$\Delta V = -\dfrac{\rho}{\eps_0}$}
C'est l'�quation de Poisson-scalaire. Dans le cas o\`u $\rho = 0$, c'est l'�quation de Laplace.
\index{Equation!poisson@de Poisson}
\index{Equation!laplace@de Laplace}
\spartie{Champ et potentiel.}
\sspartie{Distributions volumiques.}
On admet le th�or�me suivant valable dans le cas d'une distribution volumique born�e telle que
$V(\infty)=0$, l'�quation de Poisson scalaire admet pour solution unique~:
\encadre{$V(P) = \dfrac{1}{4\pi\eps_0}\integraless{V}{{}}{\dfrac{\rho(M)}{r}\dd{\tau}}$}
En exprimant ensuite le champ comme le gradient du potentiel, on trouve~:
\encadre{$\ve{E}(P) = \dfrac{1}{4\pi\eps_0}\integraless{V}{{}}{\rho(M)\dfrac{\ve{u}}{r^2}\dd{\tau}}$}
O\`u $r$ d�signe la distance entre le point $P$ d'observation et le point $M$ source qui varie dans 
l'int�grale. On montre facilement que le champ et le potentiel �lectrostatique cr�es par une distribution 
volumique de charge sont continus en tout point de l'espace.
\sspartie{Distribution surfacique.}
Ce sont les m�mes formules en changeant $\rho$ en $\sigma$ et $V$ en $S$.
Par contre il y a un probl�me avec le champ $\ve{E}$ qui n'est pas d�fini sur la distribution en question.
Sa discontinuit� est donn�e par la relation~:
\index{condition@Condition aux limites}
\encadre{$\ve{E_2}-\ve{E_1} = \dfrac{\sigma(M)}{\eps_0}\ve{n_{12}}$}
$\ve{E_2}$ d�signant le champ au voisinage du point $M$ de la surface du c�t� $2$, $\ve{E_1}$ d�signant le 
champ au voisinage de ce point du c�t� $1$, et $\ve{n_{12}}$ la normale � la surface au point $M$ dans le 
sens $1$ vers $2$.
\sspartie{Distribution lin�ique.}
Ce sont les m�mes formules en changeant $\rho$ en $\lambda$ et $V$ en $L$.
Par contre ni le champ $\ve{E}$ ni le potentiel $V$ ne sont d�finis sur la distribution en question.
\sspartie{Distribution discr�te.}
Ce sont les m�mes formules en changeant les int�grales par des sommes et les $\rho\dd{\tau}$ par des
$q_i$.
Ni le potentiel ni le champ ne sont d�finis sur chacune des charges $q_i$.
\sspartie{Comportement face � une sym�trie.}
\index{symetrie@Sym�trie}
On suppose que la distribution de charge poss�de une sym�trie plane. Alors, le potentiel poss�de la m�me
sym�trie de m�me que la composante parall�le au plan du champ $\ve{E}$. La composante perpendiculaire du 
champ est elle antisym�trique par rapport au plan. Pour un plan d'antisym�trie, c'est le contraire.
\spartie{Conducteur en �quilibre.}
\index{Conducteur!equilibre@en �quilibre}
\sspartie{Equilibre �lectrique d'un conducteur.}
\point{Conditions d'�quilibre~:}
\index{equilibre@Equilibre!conducteur@d'un conducteur}
Le conducteur est assimil� � un milieu ohmique solide pour lequel $u$ et $s$ satisfont � $\dd{u} = T\dd{s}$.
(cf~: thermodynamique). Le bilan local d'�nergie interne permet d'obtenir celui d'entropie (en 
appliquant la loi de Fourier et le premier principe)~:\par
\index{Entropie}
$$\div\Bigl(\dfrac{\ve{J_{th}}}{T}\Bigr) + \rho\df{s}{t} = \grad\Bigl(\dfrac{1}{T}\Bigr)\cdot\ve{J_{th}} + 
\dfrac{\ve{j}\cdot\ve{E}}{T}$$
On en d�duit les flux et forces thermodynamiques dans le conducteur~:\par
\index{Flux!thermodynamique}
\index{Force!thermodynamique}
\vspace{1mm}
\centerline{$\begin{array}{l|l|l}
\mathrm{Flux} & \ve{J_{th}} & \ve{j}\\
\hline
\raise-4pt\hbox{Forces}&%
\raise-4pt\hbox{$\grad\bigl(\frac{1}{T}\bigr)$}&\raise-4pt\hbox{$\frac{\ve{E}}{T}$}\\
\end{array}$}\vspace{1mm}
Ce qui donne les conditions d'�quilibre du conducteur~:
\encadre{$T$ uniforme ; $\ve{E} = \ve{0}$}
\point{Cons�quences imm�diates.}
\encadre{Le potentiel est uniforme au sein du conducteur.}
On notera d�sormais $V$ la valeur de ce potentiel uniforme en adoptant la convention : $V(\infty) = 0$.
\encadre{Un conducteur en �quilibre ne peut �tre charg� qu'en surface.}
Le champ r�gnant dans le vide au voisinage du conducteur � l'�quilibre satisfait � la relation~:
\encadre{$\ve{E}(P) = \dfrac{\sigma(M)}{\eps_0}\ve{n}$}
\index{theoreme@Th�or�me!coulomb@de Coulomb}
Ce th�or�me (de Coulomb) d�coule imm�diatement de la condition aux limites d�j� �tablie. On en d�duit que
le lignes de champ sont orthogonales � la surface du conducteur.
\sspartie{Conducteur seul dans l'espace.}
\index{Conducteur!seul@seul dans l'espace}
\point{Th�or�me d'unicit�~:}
\index{theoreme@Th�or�me!unicite@d'unicit�}
La donn�e soit du potentiel $V$ soit de la charge $q$ suffit � d�terminer de mani�re unique l'�tat 
d'�quilibre d'un conducteur seul dans l'espace.\par
La d�monstration n'est pas facile et est faite en exercice. Ce r�sultat est tr�s puissant et tr�s utile 
dans tous les exercices sur les conducteurs.\par
En particulier le potentiel �lectrostatique solution de l'�quation de Laplace dans le vide et qui satisfait
aux deux conditions aux limites $V(P) = V$ sur une surface ferm�e $S$ et $V(\infty) = 0$ est unique.
\point{Relation charge-potentiel~:}
Le conducteur �tant en �quilibre, on note $\alpha(P)$ le potentiel qui r�gne dans le vide lorsqu'il est au 
potentiel unit�. Le th�or�me d'unicit� permet d'affirmer que le potentiel qui r�gne dans le vide lorsque 
le conducteur est port� au potentiel $V$ est $V\alpha(P)$. 
On peut alors calculer la charge du conducteur~:\\
$q = \integrale{S}{{}}{\sigma(M)\dd{S}} = \integrale{S'}{{}}{-\eps_0\grad(V)\cdot\ve{n}\dd{S}}$\\
o\`u $S'$ est une surface infiniment voisine de $S$ o\`u le champ $\ve{E}$ satisfait au th�or�me de 
Coulomb. On a alors~:
\encadre{$q = CV$}\par\index{capacite@Capacit�}
A condition de poser $C = -\eps_0\integrale{S'}{{}}{\grad(\alpha)\cdot\ve{n}\dd{S}}$ la capacit� du 
conducteur.
\point{Propri�t�s de la capacit�~:}
La capacit� d'un conducteur en �quilibre seul dans l'espace ne d�pend que de la g�om�trie du conducteur (de 
m�me que $\alpha$) et est toujours positive (se d�montre avec des consid�rations �nerg�tiques).
\point{Capacit� d'un conducteur sph�rique~:}
\index{Conducteur!spherique@sph�rique}
On utilise que le potentiel est uniforme et sa valeur peut �tre calcul�e facilement au centre de la boule
conductrice. On obtient~:
\encadre{$C = 4\pi\eps_0 R$}
En ordre de grandeur on trouve $C = 10^{-12}F$ et cette formule permet de retenir l'unit� de $4\pi\eps_0$.
\point{Capacit� d'un conducteur ellipso�dal~:}
\index{Conducteur!ellipsoidal@ellipso�dal}
On se donne un conducteur elliso�dal de param�tres $a,b,c$ \footnote{Tels que~: $a^2 = b^2+c^2$} seul dans l'espace.
Pour calculer sa capacit� il faudrait conna�tre l'expression du Laplacien en coordonn�es elliptiques, ce 
qui est compl�tement H-P. On utilise la m�thode des images �lectriques. En effet si l'ellipso�de est port� 
� un potentiel $V$, il cr�e dans le vide le m�me potentiel qu'un segment uniform�ment charg� $\lambda = 
\dfrac{4\pi\eps_0 V}{\ln(\frac{a+c}{a-c})}$ (gr�ce au th�or�me d'unicit� et � l'expression du potentiel cr�e 
par un segment) On trouve alors $q = 2c\lambda$, ce qui fournit la capacit�~:
\encadre{$C = \dfrac{8\pi c\eps_0}{\ln(\frac{a-c}{a+c})}$}
\rmq Si $c\to 0$ � l'aide d'un DL on retrouve la capacit� du conducteur sph�rique.\par
Si $\frac{a}{b}\to\infty$, l'ellipso�de s'allonge pour ressembler � une pointe. On constate alors (apr�s 
un calcul) que le champ � la pointe est tr�s grand devant celui � l'extremit� aplatie. C'est ce qu'on appelle
``l'effet de pointe''.  Cet effet justifie des ph�nom�nes comme le vent �lectrostatique.
\point{Conducteur creux~:}
Un conducteur creux seul dans l'espace et dont la cavit� ne contient aucune charge est �lectriquement 
�quivalent au conducteur plein de m�me g�om�trie et port� au m�me potentiel.
\sspartie{Conducteur en pr�sence de charges � l'ext�rieur.}
\point{Ph�nom�ne d'influence~:}
\index{phenomene@Ph�nom�ne d'influence}
La charge � la surface du conducteur d�pend de ces distributions ext�rieures.
\point{Effet d'�cran~:}
Un conducteur creux en �quilibre dont le potentiel est maintenu constant s�pare l'espace en deux r�gions 
�lectrostatiquement ind�pendantes. On dit de ce conducteur qu'il constitue un �cran. Cette prori�t� a de
nombreuses applications.
\index{effet@Effet d'�cran}
\sspartie{Forces subies par un conducteur.}
\point{Pression �lectrostatique~:}
\index{Pression!Electrostatique}
Ce r�sultat est d�montr� dans le cours. Sa d�monstration n'est pas tr�s difficile une fois qu'on la vue 
une fois. Par contre il est tr�s utile en pratique~:\par
Un �l�ment de surface $\delta S$ du conducteur subit une force $\delta\ve{F}$ telle que~:
\encadre{$\delta\ve{F} = \dfrac{\sigma^2}{2\eps_0}\ve{n}\dd{S}$}
\spartie{Condensateur.}
\index{Condensateur}
\sspartie{Syst�me de deux conducteurs en influence totale.}
Les deux conducteurs sont en �quilibres~: le champ $\ve{E}$ est nul en leur int�rieur. De plus on dit qu'ils 
sont en influence totale lorsqu'aucune ligne de champ ne va � l'infini et que de plus le potentiel est nul 
� l'infini.
Dans ce cas, en notant $\alpha(P)$ le potentiel cr�e dans le vide quand le conducteur $1$ est port� au 
potentiel unit� et le $2$ au potentiel nul. Le th�or�me d'unicit� donne alors l'expression du potentiel en 
fonction de $\alpha$~:
\encadre{$V = (V_1 - V_2)\alpha(P) + V_2$}
On en d�duit que les lignes de champ co�ncident avec celles de $\grad \alpha$. Elles ne d�pendent donc que de
la g�om�trie des conducteurs.
\sspartie{D�finition d'un condensateur.}
Etant donn� deux conducteurs en influence totale, on appelle condensateur le syst�me form� d'un tube de 
champ $\Sigma$ et des deux surface $S_1$ et $S_2$ qui lui correspondent sur chacun des conducteurs.
$S_1$ et $S_2$ sont dites correspondantes et portent des charges oppos�es.
On notera d�sormais $q = q_1 = -q_2$, $V = V_1 - V_2$ la charge et la ddp du condensateur. 
\sspartie{Capacit� d'un condensateur.}
\index{capacite@Capacit�}
La charge d'un condensateur est proportionnelle � la ddp interarmatures. En effet~:\par
$q = \integrale{S_1}{{}}{\sigma(M)\dd{S}} = \integrale{S_1'}{{}}{\eps_0\ve{E}(P)\cdot\ve{n}\dd{S}} = 
\eps_0(v_1-v_2)\integrale{S_1'}{{}}{\grad(\ve{\alpha})\cdot\ve{n}\dd{S}}$. Soit~:
\encadre{$q = Cv$}
$C$ ne d�pend que de la g�om�trie du condensateur (de m�me que $\alpha$). De plus elle est toujours positive.
\point{Calcul de capacit�~:}
On formule certaines hypoth�ses sur le potentiel interconducteur � partir de la g�om�trie du syst�me 
consid�r�. En effet les surfaces des deux conducteurs sont deux �quipotentielles et � partir de leur forme
on peut conjecturer la forme de toutes les �quipotentielles. On r�sout alors l'�quation de Laplace dans le 
vide avec les conditions aux limites. Cette solution est alors \textbf{la} solution gr�ce au th�or�me 
d'unicit�. On peut alors en d�duire le champ, la charge surfacique, d�finir proprement le condensateur et 
calculer sa capacit�.
\sspartie{Exemples de condensateurs.}
\point{Condensateur plan~:}
\index{Condensateur!plan}
On formule l'hypoth�se que les �quipotentielles sont des plans parall�les aux plans des armatures. On en 
d�duit que $V$ ne d�pend que de la distance � la premi�re armature. 

Le potentiel s'�crit donc~:\par
$V(z) = \left\lbrace
\begin{array}{ll}
V_1 & \mathrm{si}\;z \leq 0\\
(V_2-V_1)\frac{z}{e} + V_1 & \mathrm{si}\;0\leq z\leq e\\
V_2 & \mathrm{sinon}\end{array}\right.$\par
On en d�duit le champ par un calcul de gradient (ici trivial) et le th�or�me de Coulomb permet alors le 
calcul de la charge surfacique d'un armature~:\par
$\sigma(M) = \eps_0\dfrac{V_1 - V_2}{e}$\par
On en d�duit la capacit�~:
\encadre{$C = \eps_0 \dfrac{S}{e}$}
\point{Condensateur cylindrique~:}
\index{Condensateur!cylindrique}
La m�thode est la m�me, quoiqu'ici elle soit plus difficile au point de vue calculatoire.
En notant $r_1$ et $r_2$ les rayons des conducteurs cylindriques, on trouve pour la capacit� lin�ique~:
\encadre{$C_0 = \dfrac{2\pi\eps_0}{\ln(\frac{r_2}{r_1})}$}
Cette capacit� intervient dans l'�tude d'un coaxial.
\point{Condensateur sph�rique~:}
\index{Condensateur!spherique@sph�rique}
Totalement inutile en pratique ce condensateur sert � faire des calculs en sph�riques. La m�thode est encore
la m�me. On suppose que le potentiel ne d�pend que de $r$ et on le v�rifie apr�s calculs gr�ce au th�or�me 
d'unicit�~:
\encadre{$C = 4\pi\eps_0\dfrac{R_1 R_2}{R_2 - R_1}$}
On a encore fait un tas d'autres condensateurs comme le di�drique, l'ellipso�dal, le c�nique etc.
\sspartie{Energie d'un condensateur.}
\index{Energie!condensateur@d'un condensateur}
\point{D�finition~:}
C'est l'�nergie �lectromagn�tique contenue dans le condensateur (cf~: 3.6). Comme on se situe dans le cadre
de l'�lectrostatique, elle a pour expression~:
\encadre{$E_{em} = \integrale{\tau}{{}}{\frac{\eps_0}{2}E^2\dd{\tau}}$}
\point{Autres expressions~:}
Un calcul astucieux permet d'obtenir les expressions �quivalentes suivantes~:
\encadre{$E_{em} = \frac{1}{2}q v = \frac{1}{2} C v^2 = \frac{1}{2}\frac{q^2}{C}$}
On peut �galement interpr�ter cette �nergie comme l'�nergie que re�oit le condensateur lorsqu'on le charge
de mani�re r�versible.
\sspartie{Relation entre capacit� et r�sistance.}
\point{R�sistance �lectrique~:}
\index{resistance@R�sistance!electrique@�lectrique}
On consid�re un milieu de conductivit� �lectrique $\gamma$ compris entre deux �lectrodes. Le potentiel y 
satisfait � l'�quation de d'Alembert~: $\dalembert V = -\frac{\rho}{\eps_0}$. En r�gime stationnaire et
dans un milieu �lectriquement neutre cette �quation se ram�ne � celle de Laplace dans le vide : $
\laplacien V = 0$ avec les conditions aux limites $V$ uniforme sur les �lectrodes. On en d�duit, comme pour
le condensateur : $V(P) = (V_1 - V_2)\alpha(P) + V_2$ en introduisant le m�me $\alpha$.\par
Or le courant de charge satisfait � $\ve{j} = \gamma\ve{E}$ (milieu immobile. cf : cours de thermodynamique)
ou encore $\ve{j} = -\gamma\grad V$ en r�gime stationnaire. Soit $\ve{j} = -\gamma(V_1 - V_2)
\grad(\ve{\alpha})$. Les lignes de courant ne d�pendent donc que de la g�om�trie des �lectrodes et cela 
permet de d�finir le syst�me compris entre un tube de champ (ou de courant) $\Sigma$ et les deux surfaces 
$S_1, S_2$ qui lui correspondent sur les �lectrodes.\par
Le courant de charge �tant � flux conservatif, son flux � travers un surface d�limit�e par $\Sigma$, ne 
d�pend pas de cette surface. C'est l'intensit� qui traverse le r�sistor. Pour la calculer on choisit pour
surface $S_1'$, infiniment voisine de $S_1$.

On en d�duit la loi d'Ohm globale~:
\index{Loi!ohm@d'Ohm}
\encadre{$v = RI$}
En posant $\dfrac{1}{R} = -\gamma\integrale{S_1'}{{}}{\grad(\alpha)\cdot \ve{n}\dd{S}}$.\\
On reconna�t presque la capacit� du condensateur � vide de m�me g�om�trie, et on en d�duit l'expression~:
\encadre{$RC = \dfrac{\eps_0}{\gamma}$}
\point{R�sistance thermique~:}
\index{resistance@R�sistance!thermique}
On consid�re un milieu de conductivit� thermique $\lambda$ uniforme compris entre deux thermostats de
temp�rature $T_1,T_2$. Dans l'espace interthermostats, la temp�rature satisfait � l'�quation de diffusion~:
$\laplacien T = \dfrac{\rho c}{\lambda}\dpa{T}{t}$. En r�gime stationnaire, cela donne : $\laplacien T = 0$.
Avec les m�mes m�thodes que ci-dessus, on en d�duit : $T = (T_1-T_2)\alpha + T_2$. Puis comme le courant
thermique satisfait � la loi de Fourier :$\ve{J_{th}} = -\lambda\grad(T)$, on en d�duit que les lignes de
courant thermique ne d�pendent que de la g�om�trie des thermostats. En plus le vecteur courant thermique
est � flux conservatif (bilan local d'�nergie interne). On peut donc tout faire comme ci-dessus et on en 
d�duit~:
\encadre{$P_{th} = \dfrac{T_1 - T_2}{R_{th}}$}
En posant $\dfrac{1}{R_{th}} = -\lambda\integrale{S_1'}{{}}{\grad(\alpha)\cdot\ve{n}\dd{S}}$. On reconna�t
encore la capacit� du condensateur � vide de m�me g�om�trie et on en d�duit la relation~:
\encadre{$R_{th}C = \dfrac{\eps_0}{\lambda}$}

\spartie{Dip�le �lectrique.}
\index{dipole@Dip�le!electrique@�lectrique}
\sspartie{D�finition.}
Un dip�le �lectrique est une distribution de charge de charge totale nulle et de moment �lectrique non-nul
dont les dimensions sont faibles devant la distance s�parant en moyenne le dip�le des points d'observation
et devant la longueur caract�risant les variations spatiales du champ ext�rieur agissant sur le dip�le.
\sspartie{Potentiel et champ.}
\point{Potentiel~:}
La distribution de charge sera suppos�e discr�te pour les calculs. La loi de Coulomb permet d'exprimer le
potentiel cr�e par cette distribution en un point d'observation ``�loign�''~: 
$V(P) = \dfrac{1}{4\pi\eps_0}\serie{i}{{}}{\dfrac{q_i}{M_iP}}$. Si on choisit $O$ un point r�f�rence voisin
du dip�le et qu'on note $r = OP$ on peut faire un d�veloppement limit� � l'ordre 1 selon $OM_i/r$.
on fait alors appara�tre  un vecteur ind�pendant de $O$~:
\encadre{$\ve{p} = \serie{i}{{}}{q_i \ve{OM_i}}$}
Ce vecteur est le ``moment �lectrique'', non-nul par hypoth�se et tel que~:
\encadre{$V(P) = \dfrac{1}{4\pi\eps_0}\ve{p}\cdot\dfrac{\ve{u}}{r^2}$}
\rmq Le moment �lectrique peut �tre mis sous la forme $\ve{p} = q_{+}\ve{B_- B_+}$ o\`u $B_-$ et $B_+$ 
d�signent les barycentres des charges positives et n�gatives et $q_+$ la somme des charges positives.
\index{Moment!electrique@�lectrique}
\rmq Dans le cas o\`u le moment dipolaire est nul il faudrait poursuivre le DL, ce qui n'est pas 
difficile mais vite p�nible. On parle ensuite de quadrup�le, d'octup�le et de $2^n$-p�le.
\index{dipole@Dip�le!deux@$2^n$-p�le}
\point{Champ~:}
En prenant le gradient et en notant $\ve{u}$ le vecteur unitaire qui joint le dip�le au point d'observation,
on trouve l'expression suivante pour le champ~:
\encadre{$\ve{E}(P) = \dfrac{1}{4\pi\eps_0 r^3}\bigl(3(\ve{p}\cdot\ve{u})\ve{u} - \ve{p}\bigr)$}
\point{Lignes de champ~:}
On peut exprimer le champ en coordon�es sphr�riques � partir de l'expression ci-dessus~:
\encadre{$\begin{array}{l}
 E_r = \frac{1}{4\pi\eps_0}\frac{2p\cos\theta}{r^3}\\
 E_{\theta} = \frac{1}{4\pi\eps_0}\frac{p\sin\theta}{r^3}
\end{array}$}\par
Les lignes de champ sont donc situ�es dans des plans $\phim = cste$. Dans un tel plan elles ont pour 
�quation~:
\encadre{$r = r_0\sin^2\theta$}
\sspartie{Force subie par un dip�le.}
\index{Force!dipole@subie par un dip�le}
\point{R�sultante~:}
On plonge un dip�le dans un champ cr�e par un distribution de charge ext�rieure, il subit de la part de ce
champ une force : $\ve{F} = \sum\limits_i q_i\ve{E}(M_i)$ en notant $\ve{E}$ le champ cr�e par le 
distribution ext�rieure. 
On peut alors faire un DL puisqu'on a suppos� que la taille du dip�le �tait faible
devant la longueur caract�risant les variations spatiales du champ ext�rieur agissant sur le dip�le. On 
trouve � l'ordre 1~:
\encadre{$\ve{F} = (\ve{p}\cdot\grad)\ve{E}$}
\point{Moment~:}
Cette fois on fait un DL � l'ordre 0 qui est suffisant~:
\encadre{$\ve{M_P} = \ve{p}\land\ve{E}$}
\rmq Lorsque le champ est uniforme, la r�sultante est nulle et les forces subies constituent un couple.
\point{Energie potentielle~:}
Les forces �lectriques subies par une distribution discr�te de charge plong�e dans un champ $\ve{E}$ 
ext�rieur d�rivent d'une e-p : $E_p = \sum\limits_i q_iV(M_i)$.
Dans le cas du dip�le on peut faire un DL � l'ordre 1~:
\encadre{$E_p = -\ve{p}\cdot \ve{E}$}

%
%--------------------------------------------------PARTIE--------------------------------------------------
%

\partie{Magn�tostatique}
\spartie{Th�or�me fondamental.}
\sspartie{Enonc�.}
Toute distribution stationnaire de courant engendre un champ vectoriel $\ve{B}$ appel� champ
ma\-gn�\-to\-sta\-tique tel que la force subie par une charge ponctuelle $q$ de la part de cette distribution ait 
pour expression~:
\index{Champ!magnetostatique@Magn�tostatique}
\index{Force!lorentz@de Lorentz}
\encadre{$\begin{array}{c}
            \div \ve{B} = 0\\
            \rot{\ve{B}} = \mu_0\ve{j}
          \end{array}$}
\sspartie{Cons�quences imm�diates.}
\point{$\div \ve{B} = 0$~:} le champ magn�tique est � flux conservatif. il d�rive donc d'un potentiel
vecteur~:
\encadre{$\ve{B} = \rot \ve{A}$}
\rmq Ce potentiel n'est d�fini qu'� un gradient pr�s.\par
\index{Potentiel!vecteur}
\point{Th�or�me d'Amp�re~:} $\rot \ve{B} = \mu_0\ve{j}$ : � cette �quation locale correspond l'�quation int�grale~:\par
$$\integrale{L}{{}}{\ve{B}\cdot\ve{t}\dd{L}} = \mu_0\integrales{S}{{}}{\ve{j}\cdot\ve{n}\dd{S}}$$
De plus $\ve{j}$ est � flux conservatif d'apr�s le bilan local de charge total. On obtient l'�quation int�grale~:
\encadre{$\integrale{L}{{}}{\ve{B}\cdot\ve{t}\dd{L}} = \mu_0 I_L$}
C'est le th�or�me d'Amp�re.\par 
\index{theoreme@Th�or�me!ampere@d'Amp�re}
\point{Equation de Poisson~:}
Les �quations locales pr�c�dentes~: $\rot \ve{B} = \mu_0\ve{j}$ et $\ve{B} = \rot \ve{A}$ permettent 
d'obtenir un nouvelle �quation~: $\Delta \ve{A} = -\mu_0\ve{j} + \grad (\div \ve{A})$. Et en choisissant
judicieusement $\ve{A}$ (de sorte que sa divergence soit nulle, ce qui est possible en prenant un bon 
gradient)~:
\encadre{$\Delta \ve{A} = -\mu_0\ve{j}$}
C'est l'�quation de Poisson vectorielle.
\index{Equation!poisson@de Poisson}
\spartie{Champ et potentiel vecteur.}
\sspartie{Distribution volumique.}
L'�quation de Poisson vectorielle � laquelle on adjoint la condition $\ve{A}(\infty) = \ve{0}$ et dans le 
cas d'une distribution born�e, admet pour solution unique~:
\encadre{$\ve{A} = \dfrac{\mu_0}{4\pi}\integraless{V}{{}}{\dfrac{\ve{j}(M)}{r}\dd{\tau}}$}
On d�duit de cela la loi de Biot-et-Savart~:
\encadre{$\ve{B} = \dfrac{\mu_0}{4\pi}\integraless{V}{{}}{\ve{j}(M)\wedge \dfrac{\ve{u}}{r^2}\dd{\tau}}$}
Les champs $\ve{A}$ et $\ve{B}$ sont d�finis et continus en tout point de l'espace.
\rmq On peut d�montrer ces r�sultats en utilisant la r�solution de l'�\-qua\-tion de Poisson-scalaire.
\sspartie{Distribution surfacique.}
On garde tout sauf que $\ve{j}$ devient $\ve{k}$ et $V$ devient $S$.\par
Le potentiel vecteur est continu partout par contre le champ magn�tostatique conna�t une discontinuit� au 
voisinage de la distribution consid�r�e~:
\encadre{$\ve{B_2}-\ve{B_1} = \mu_0\ve{k}(M)\wedge \ve{n_{1 2}}$}
\index{condition@Condition aux limites}
\sspartie{Distribution lin�ique.}
C'est pareil mais comme $\ve{j}$ est � flux conservatif, $I$ a m�me valeur en tout point et donc~:
\encadre{$\begin{array}{cc}
\ve{A}(P) = \dfrac{\mu_0}{4\pi}I\integrale{L}{{}}{\dfrac{\ve{t}}{r}\dd{L}}\\
\ve{B}(P) = \dfrac{\mu_0}{4\pi}I\integrale{L}{{}}{\ve{t}\wedge\dfrac{\ve{u}}{r^2}\dd{L}}
          \end{array}$}
De plus ni le champ ni le potentiel vecteur ne sont d�finis sur la distribution consid�r�e.
\sspartie{Comportement face � une sym�trie.}
\index{symetrie@Sym�trie}
Le potentiel vecteur $\ve{A}$ se comporte comme le champ �lectrostatique, � l'inverse du champ $\ve{B}$.
\spartie{Circuits filiformes.}
\index{Circuit@Circuit filiforme}
\sspartie{D�finition.}
Un circuit filiforme est une distribution volumique de courant dont le vecteur $\ve{j}$ est quasi uniforme sur
toute section droite du fil.
\rmq Le courant circule le long du fil. Le vecteur $\ve{j}$ est donc tangent � la surface du fil. Comme il est
quasi-uniforme, il l'est donc sur toute section droite du fil et on peut dire qu'en tout point : 
$\ve{j} = \frac{I}{s}\ve{t}$ (Si on se place en magn�tostatique ou dans l'ARQS,
$\div \ve{j} = 0$). D'o\`u : $\ve{j}\dd{\tau} = I\ve{t}\dd{l}$. Cette relation permet de transformer toute 
int�grale volumique en une int�grale curviligne.
\rmq Toute distribution de courant peut �tre d�compos�e en circuits filiformes, d'o\`u cette �tude. Cependant
il importe de ne pas confondre circuit filiforme et distribution lin�ique de courant pour laquelle le champ n'est
pas continu.
\sspartie{Flux d'un champ magn�tique � travers un circuit filiforme.}
On notera $\ve{B},\ve{A}$ le champ magn�tique et un potentiel vecteur cr�es par une distribution quelconque, $L$
une courbe qui �pouse la forme d'un tel circuit et $S$ une surface qui s'appuie sur $L$. Dans ces conditions~:
\encadre{$\Phi = \integrale{L}{{}}{\ve{A}\cdot\ve{t}\dd{L}} = \integrale{S}{{}}{\ve{B}\cdot\ve{n}\dd{S}}$}
et $\Phi$ est appel� flux de $\ve{B}$ � travers le circuit filiforme.
\spartie{Inductance propre.}
\index{Inductance}
\point{D�finition~:}
Le flux du champ $\ve{B}$ cr�e par un circuit filiforme � travers lui-m�me est appel� ``flux propre'' du circuit.
On montre avec un calcul que~:
\encadre{$\Phi = LI$}
En posant $\Phi = \dfrac{\mu_0}{4\pi}\integrale{L}{{}}{\integrale{L'}{{}}{\dfrac{\ve{t}\cdot\ve{t'}}{%
r}\dd{L}\dd{L'}}}$. O\`u $L'$ est une courbe qui s'appuie sur le circuit filiforme et qui sert � exprimer le 
potentiel vecteur.
C'est l'inductance propre du circuit filiforme.
\point{Propri�t�s~:}
$L$ ne d�pend que des caract�ristiques g�om�triques du circuit elle est toujours positive (on le montrera avec des
consid�rations �nerg�tiques), et s'exprime en Henry. Typiquement elle vaut $10 mH$.
\point{Bobine torique~:}
\index{Inductance!bobine@d'une Bobine torique}
On prend une bobine torique � section rectangulaire comprenant $N$ spires jointives, parcourue par un courant
d'intensit� $I$. On note $a$ la distance cu c�t� int�rieur de la bobine � l'axe et $b$ la distance de cot� 
ext�rieur � l'axe ainsi que $h$ la hauteur du rectangle. Etant donn� que le potentiel vecteur $\ve{A}$ est continu
sur une boule compacte qui enveloppe le circuit, il y est uniform�ment continu. On peut donc dire que pour le 
calcul du flux, tout se passe comme si chaque spire �tait ind�pendante 
(elles sont suppos�es voisines les unes des autres). Or on conna�t le champ cr�e~: 
$\ve{B} = \dfrac{\mu_0 N I}{2\pi r}\ve{u_{\theta}}$. On peut donc calculer le flux du champ � travers chaque 
spire : $\phim = \dfrac{\mu_0N I h}{2\pi}\ln\bigl(\frac{b}{a}\bigr)$.
Il reste � sommer sur toutes les spires~:
\encadre{$L = \dfrac{\mu_0 N^2}{2\pi}h\ln\left(\dfrac{b}{a}\right)$}
\point{Bobine cylindrique~:}
\index{Inductance!bobine@d'une Bobine cylindrique}
On suppose que les spires ont une densit� lin�ique $n$ uniforme et que la bobine est de longueur $l$. Le circuit 
n'est pas ferm� mais on peut g�n�raliser et d�finir le flux comme la somme des flux sur toutes les spires.
En supposant la bobine infinie ou en n�gligeant les effets de bord on trouve que le flux est proportionnel � $I$.
On parle encore d'inductance propre et sa valeur est~:
\encadre{$L = \mu_0n^2l\pi R^2$}
\point{Ligne bifilaire~:}
\index{Inductance!bifilaire@d'une ligne bifilaire}
Deux circuits fils cylindriques parall�les sont parcourus par des courants $I$ et $-I$. Si on calcule le flux de
$\ve{B}$ � travers une surface rectangulaire s'appuyant les axes des fils et de hauteur $h$, on trouve qu'il est 
proportionnel � $I$ et on parle encore d'inductance propre~:
\encadre{$L = \dfrac{\mu_0}{\pi}\left(\dfrac{1}{2}+\ln\Bigl(\dfrac{a}{R}\Bigr)\right)$}\par
En notant $a$ la distance entre les axes et $R$ le rayon des fils.
\sspartie{Matrice inductance.}
\index{Matrice!inductance}
\point{D�finition~:}
Pour un syst�me de circuits filiformes, chacun �tant parcouru par une intensit� $I_j$, un calcul p�nible permet 
d'affirmer qu'il existe des coefficients $L_{ij}$ tels que : $\Phi_i = \sum\limits_j L_{ij}I_j$. Les 
coefficients $L_{ij}$ forment une matrice carr�e sym�trique dont les coefficients diagonaux sont les 
inductances propres des circuits. Les coefficients non-diagonaux peuvent �tre n�gatifs. Ils se calculent en 
annulant certaines intensit�s dans le circuit, en calculant le champ et son flux.
\spartie{Consid�rations �nerg�tiques en magn�tostatique.}
\sspartie{Energie magn�tique d'une distribution stationnaire de courants.}
\point{D�finition~:}
\index{Energie!magnetique@magn�tique}
L'�nergie �lectromagn�tique est un grandeur extensive de densit� volumique $\frac{1}{2\mu_0}B^2$ (dans le cadre de 
la magn�tostatique). On d�finit l'�nergie magn�tique d'une distribution de courant d�limit�e par un volume $\tau$
comme~:
\encadre{$E_{em} = \integrale{\tau}{{}}{\dfrac{1}{2}\ve{j}\cdot\ve{A}\dd{\tau}} = 
\integrale{\mathcal{E}}{{}}{\dfrac{1}{2\mu_0}B^2\dd{\tau}}$}\par
On montre facilement que ces deux expressions sont �quivalentes.
\point{Cas du circuit filiforme~:}
Un simple calcul donne~:
\encadre{$E_{em} = \dfrac{1}{2}LI^2$}
On en d�duit que $L$ est toujours positive. De plus cette expression permet de d�finir l'inductance propre de 
circuits non filiformes comme un cable coaxial.
\sspartie{Travail des forces magn�tiques.}
\point{Loi de Laplace~:}
\index{Loi!laplace@de Laplace}
On consid�re une particule de mati�re, plong�e dans un champ ma\-gn�\-to\-sta\-ti\-que ext�rieur quasi-uniforme sur le volume
$\delta\tau$ de la particule. Alors le calcul de la force subie par la particule se fait en s�parant les 
constituants dans la force de Lorentz et en reconnaissant le courant de charge total dans l'expression obtenue~:
\encadre{$\delta\ve{F} = \ve{j}\land\ve{B}\delta{\tau}$}
Cette expression pouvant �tre �tendue au cas de distributions surfaciques ou lin�iques.
\rmq Les forces ci dessus sont appel�es forces de Laplace.\index{Force!laplace@de Laplace}\par
Dans le cadre du syst�me S-I, l'amp�re est l'intensit� d'un courant qui, maintenu stationnaire dans deux fils 
rectilignes parall�les infiniment longs, plac�s � une distance de un m�tre, engendre entre ces deux fils des forces
de r�sultante $2~\cdot~10^{-7}$N.\index{ampere@Amp�re}
\point{Travail des forces subies par un circuit filiforme~:}
On consid�re un circuit filiforme ferm� se d�pla�ant dans un champ $\ve{B}$ ext�rieur stationnaire. Si on note 
$l$ et $l'$ les positions occup�es par le circuit � deux instants voisins $t$ et $t'$, on peut calculer le travail 
des forces magn�tiques lors de ce d�placement en commen�ant par calculer le travail subi par une portion de circuit
situ�e entre $M$ et $N$ points voisins. Par int�gration le long le long du circuit, on en d�duit :\par
$\delta\mathcal{T} = I \integrale{\Sigma}{}{\ve{B}\cdot\ve{n}\dd{S}}$\par
O\`u $\Sigma $ est la surface d�crite par le circuit lors de son d�placement. On peut calculer cette int�grale en 
exprimant la nullit� du flux du champ � travers une grande surface ferm�e s'appuyant sur $l$ et $l'$ et contenant
$\Sigma$. En notant alors $\phi$ le flux du champ � travers le circuit~:\par
$\integrale{\Sigma}{}{\ve{B}\cdot\ve{n}\dd{S}} + \phi - \phi' = 0$ soit~:
\encadre{$\delta\mathcal{T} = I\dd{\phi}$}
\exercice Deux sol�noides cylindriques coaxiaux tels que $R_2 < R_1$, parcourus par un courant stationnaire. 
D�terminer la r�sulante des forces subies par le petit sol�noide, puis en d�duire qu'il n'existe qu'une seule 
position d'�quilibre stable.\\
\textsl{Indication :} Consid�rer plusieurs cas et faire un raisonnement de type Joule-Thomson~: on trouve que 
la seule position se trouve au milieu du grand sol�noide ($x=0$).
\rmq Le calcul � partir de la force de laplace en n�gligeant les effets de bord conduit � une r�sulante 
uniform�ment nulle  Ceci montre que la force est due aux effets de bord.
\point{Cas d'un courant stationnaire~:}
Dans ce cas $\delta\ttt = \dd{I\phi}$. On en d�duit que les forces magn�tiques d�rivent d'une e-p de la forme :
\encadre{$E_p = - I \phi$}
\point{Moment magn�tique d'un circuit filiforme~:}
\index{Moment!magnetique@magn�tique}
Dans le cas o\`u le courant parcourant le circuit est uniforme et le champ ext�rieur est uniforme au voisinage du
circuit, les forces magn�tiques d�rivent de l'e-p~:\par
$E_p = -I\phi = -(I\integrale{S}{}{\ve{n}\dd{S}})\cdot\ve{B}$ Soit~:
\encadre{$E_p = -\ve{m}\cdot\ve{B}$}
o\`u $\ve{m}$ d�signe le moment magn�tique.
\rmq Un calcul simple permet de montrer que 
\encadre{$\ve{m} = I\integrale{S}{}{\ve{n}\dd{S}} = \frac{I}{2}\integrale{L}{}{\ve{r}\land\ve{t}\dd{l}}$}
Ce qui montre que le moment magn�tique ne d�pend que du circuit filiforme et pas de la surface $S$.
\exercice Calculer le moment d'un circuit filiforme contenu dans un plan. En d�duire le moment d'un sol�noide 
comprenant $N$ spires jointives.
\spartie{Dip�le magn�tique.}
\index{dipole@Dip�le!magnetique@magn�tique}
\sspartie{D�finition.}
Un dip�le magn�tique est une distribution de courant de moment magn�tique non-nul dont les dimensions sont faibles 
devant la distance s�parant en moyenne le dip�le des points d'observation et devant la longueur caract�risant les 
variations spatiales du champ ext�rieur agissant sur le dip�le.
\rmq A l'�chelle microscopique on d�finit le moment magn�tique (orbital ou de spin) de l'entit�. A l'echelle
macroscopique on d�finit le moment par unit� de volume, appel� $\ve{M}$ ``vecteur aimantation''. Si ce vecteur est
non-nul le milieu est dit aimant�.
\sspartie{Champ d'un dip�le.}
\point{Potentiel vecteur magn�tique~:}
Comme toute distribution de courant peut �tre d�compos�e en circuits filiformes, on se restreint aux circuits 
filiformes. Pour un tel circuit le potentiel vecteur peut �tre calcul� directement gr�ce � la relation~:
$\ve{A}(P) = \dfrac{\mu_0 I}{4\pi}\integrale{l}{}{\dfrac{\ve{t}}{MP}\dd{l}}$. En notant $O$ un point au voisinage
du dip�le $OM$ est tr�s petit devant $OP$. La m�thode naturelle consiste � faire un DL comme on l'avait fait pour
le dip�le �lectrique. Cette m�thode est laborieuse et in�l�gante. On introduit un champ vectoriel uniforme $\ve{a}$
interm�diaire de calcul~: $\ve{a}\cdot\ve{A}$ se calcule simplement gr�ce au th�or�me de Stokes. Puis, en 
effectuant un DL � l'ordre $0$ on obtient le r�sultat~:
\encadre{$\ve{A} = \dfrac{\mu_0}{4\pi}\ve{m}\land\dfrac{\ve{u}}{r^2}$}
En notant $r = OP$ et $\ve{u}$ le vecteur $\ve{OP}$ normalis�.
\point{Champ magn�tique~:}
On prend le rotationnel du potentiel ci-dessus et, via un petit calcul on obtient~:
\encadre{$\ve{B} = \dfrac{\mu_0}{4\pi r^3}(3(\ve{m}\cdot\ve{u})\ve{u}-\ve{m})$}
\rmq Les lignes de champ sont donc les m�mes que celles du dip�les �lectrique.
\point{Actions subies par un dip�le~:}
On peut calculer le torseur des forces subies directement mais c'est p�nible. on pr�f�re utiliser une m�thode
plus originale. On imagine un d�placement infinit�simal du dip�le au cours duquel celui-ci est suppos� ind�formable
est parcouru par un courant stationnaire. Le travail de ces forces peut �tre calcul� de deux mani�res.\\
D'un part $\delta\mathcal{T} = \dd{t}\mathcal{F}\mathcal{V}=\dd{t}(\ve{F}\cdot\ve{v_P}+\ve{M_P}\cdot\ve{\omega})$.
(cf~: m�canique du solide).\\
D'autre part :$\delta\ttt = -\dd{E_p}$ avec $E_p = -\ve{m}\cdot\ve{B}$ car $\ve{B}$ est uniforme au voisinage
du dip�le.
\begin{itemize}
\item{R�sultante~:}
On choisit pour d�placement  une translation de vecteur $\dd{x}\vx$. On obtient dans ce cas $F_x = 
\ve{m}\cdot\dpa{\ve{B}}{x}$. C'est-�-dire $(\ve{m}\cdot\grad)B_x$ dans le cas o\`u il n'y a pas de courant au 
voisinage du dip�le ($\rot\ve{B} = \ve{0}$). Puis~:
\encadre{$\ve{F} = (\ve{m}\cdot\grad)\ve{B}$}
\item{Moment en $P$~:}
Cette fois on choisit une rotation autour de $P$ et obtient facilement :
\encadre{$\ve{M_P} = \ve{m}\land\ve{B}$}
\end{itemize}

%
%--------------------------------------------------PARTIE--------------------------------------------------
%

\partie{Equations de Maxwell}
\index{Equation!maxwell@de Maxwell}
\spartie{Principe fondamental de l'�lectromagn�tisme.}
\index{Principe!electromagnetisme@de l'�lectromagn�tisme}
\sspartie{Enonc�.}
Toute distribution de charge et de courant engendre un champ �lectro\-magn�tique $(\ve{E}, \ve{B})$ tel que la
force exerc�e par cette distribution sur une charge ponctuelle $q$ ait pour expression~:\par
\encadre{$\ve{F} = q (\ve{E} + \ve{v}\wedge \ve{B})$}\par
\index{Force!lorentz@de Lorentz}
Ce champ e-m satisfait aux �quations de Maxwell~:\par\vspace{2pt}
\index{Champ!electromagnetique@�lectromagn�tique}
\centerline{\fbox{{\renewcommand{\arraystretch}{1.5}
$\begin{array}{lclc}
(1) & \rot\ve{E} = -\dpa{\ve{B}}{t} & (2) & \div\ve{B} = 0\\
(3) & \div\ve{E} = \dfrac{\rho}{\eps_0}   & (4) & \rot\ve{B} = \mu_0\Bigl(\ve{j}+\eps_0\dpa{\ve{E}}{t}\Bigl)
\end{array}$}}}
\sspartie{Propri�t�s.}
Les deux premi�res �quations sont celles dites de structure. Elles sont in\-d�\-pen\-dan\-tes du milieu consid�r�.
D'un point de vue th�orique elles traduisent les carac\-t�ristiques (masse nulle et spin unit�) du photon, 
particule associ�e au champ e-m.\par
Les deux derni�res ne sont valables que dans les milieux non-polaris�s non-aimant�s. Les �quations g�n�rales sont pr�sent�es
apr�s, mais ne sont pas au programme de MP.
Les �quations de Maxwell admettent comme cas particulier celles des deux chapitres pr�c�dents.\par
Dans le cas g�n�ral il n'est plus possible de s�parer l'�tude des champs $\ve{E}$ et $\ve{B}$ car ils sont
coupl�s.\par
Ces �quations portent des noms particuliers. Ainsi $(2)$ est l'�quation de Maxwell-Faraday, $(3)$ celle de Maxwell-Gauss
et $(4)$ celle de Maxwell-Amp�re. Elles montrent que la grandeur extensive ``charge totale'' est conservative. En prenant la 
divergence de Maxwell-Amp�re on tire~:
\encadre{$\div\ve{j}+\dpa{\rho}{t}=0$}
Les �quations de Maxwell sont lin�aires, ce qui justifie le th�or�me de superposition et la notation 
complexe.
\spartie{Le potentiel �lectromagn�tique.}
\sspartie{D�finition.}
\index{Potentiel!electromagnetique@�lectromagn�tique}
Etant donn� que $\div\ve{B} = 0$, $\ve{B}$ d�rive d'un potentiel vecteur not� $\ve{A}$. Ensuite en prenant le
rotationnel de $\ve{E} + \dpa{\ve{A}}{t}$ on trouve qu'il d�rive d'un potentiel scalaire. Il existe donc un 
couple $(V, \ve{A})$ tel que~:
\encadre{%
$\begin{array}{c}
\ve{B} = \rot\ve{A}\\
\ve{E} = -\grad V - \dpa{\ve{A}}{t}
\end{array}$}
\sspartie{Non-unicit�.}
Si $(V_0, \ve{A_0})$ est un potentiel dont d�rive le champ e-m, alors $(V_0-\dpa{f}{t},\ve{A_0}+\grad{f})$
en est un aussi. Cette non-unicit� est mise � profit pour simplifier les �quations de D'Alembert sur
le potentiel e-m.
\spartie{Equations de D'Alembert.}
\index{Equation!alembert@de d'Alembert}
En choisissant un bon potentiel qui satisfait � la jauge de Lorentz : $\div\ve{A}+\eps_0 \mu_0 \dpa{V}{t}
=0$ on peut arriver aux �quations suivantes~:\index{Jauge!lorentz@de Lorentz}
\encadre{{\renewcommand{\arraystretch}{1.5}$\begin{array}{c}
\laplacien V-\eps_0 \mu_0 \dpak{2}{V}{t} = -\dfrac{\rho}{\eps_0}\\
\laplacien \ve{A}-\eps_0 \mu_0 \dpak{2}{\ve{A}}{t} = -\mu_0 \ve{j}
\end{array}$}}
Et si on utilise l'op�rateur d'Alembertien d�fini par : $\dalembert = \laplacien - \eps_0 \mu_0 \dpak{2}{{}}{t}$, 
on peut r�crire ces �quations sous la forme plus jolie :\par
\encadre{{\renewcommand{\arraystretch}{1.5}$\begin{array}{c}
\dalembert V = -\dfrac{\rho}{\eps_0}\\
\dalembert \ve{A} = -\mu_0 \ve{j}\end{array}$}}
Dans le vide l'�quation de d'Alembert relative au potentiel �lectrique est une �quation d'onde.\\
$\laplacien V = \dfrac{1}{c^2}\dpak{2}{V}{t}$ dont la c�l�rit� est $c=\sqrt{\eps_0 \mu_0}$.
\spartie{L'ARQS}
\sspartie{Potentiel retard�.}
\index{Potentiel!retarde@retard�}
On g�n�ralise la r�solution de l'�quation de Poisson et on montre que le potentiel �lectromagn�tique est de la 
forme~:
\encadre{{\renewcommand{\arraystretch}{1.5}
$\begin{array}{c}
V(P, t) = \dfrac{1}{4\pi\eps_0}\integraless{V}{{}}{\dfrac{\rho(M,t-\frac{r}{c})}{r}\dd{\tau}}\\
\ve{A}(P, t) = \dfrac{\mu_0}{4\pi}\integraless{V}{{}}{\dfrac{\ve{j}(M,t-\frac{r}{c})}{r}\dd{\tau}}
\end{array}$}}
On interpr�te cette expression du potentiel comme le fait que l'information e-m se propage � la vitesse de la 
lumi�re et le retard $\frac{r}{c}$ correspond au temps mis part la charge ou le courant en M pour agir en P.
\sspartie{D�finition de l'ARQS.}
L'ARQS revient � se placer dans le cadre de la m�canique classique dans laquelle l'information se propage 
instantan�ment, ce qui veut dire qu'on consid�re que $c$ est infini.\\
On suppose donc que le temps mis par la distribution pour subir des variations notables est tr�s sup�rieur au 
temps $r/c$. D'o\`u l'appellation ``quasi-stationnaire''.
\sspartie{Cons�quences.}
On va donc n�gliger tous les termes qui contiennent des $\frac{1}{c}$ et les remplacer par 0, ce qui va 
simplifier toutes les �quations de l'�lectromagn�tisme.
\index{ARQS}
Ainsi les nouvelles �quations de Maxwell sont~:
\encadre{$\begin{array}{ll}
\rot\ve{E} = -\dpa{\ve{B}}{t} & \div \ve{B} = 0\\
\div\ve{E} = \dfrac{\rho}{\eps_0} & \rot\ve{B} = \mu_0\ve{j}
\end{array}$}
\index{theoreme@Th�or�me!gauss@de Gauss}
\index{theoreme@Th�or�me!ampere@d'Amp�re}
On a toujours le th�or�me de Gauss et on a en prime le th�or�me d'Amp�re sur la circulation de $\ve{B}$.
En prenant la divergence de la nouvelle �quation de Maxwell-Amp�re, on trouve que le vecteur courant de charge
$\ve{j}$ est � flux conservatif. Cette propri�t� permet de justifier la loi des noeuds.
\rmq On ne peut pas faire l'�tude du condensateur plan dans le cadre de l'ARQS.\par
Les potentiels retard�s se simplifient trop simple en les lois de Coulomb et Biot-et-Savart pour les potentiels,
de m�me le champ magn�tique satisfait � la loi de Biot-et-Savart mais pas le champ �lectrique.
\index{Jauge!lorentz@de Lorentz}\index{Jauge!coulomb@de Coulomb}
\index{Equation!poisson@de Poisson}
Les �quations de d'Alembert retombent sur les �quations de Poisson et la jauge de Lorentz retombe sur la jauge de 
Coulomb : $\div\ve{A} = 0$. En conclusion~:\par\index{Jauge!coulomb@de Coulomb}
Tous les r�sultats de la magn�tostatique restent valables et la plupart de ceux de l'�lectrostatique (except�
la loi de Coulomb et la relation $\ve{E} = -\grad V$).
\sspartie{Domaine de validit� de l'ARQS.}
La transformation de Fourier int�grale permet de ramener l'�tude de n'importe quelle fonction ``r�guli�re'' � 
celle d'une fonction p�riodique de p�riode T pour laquelle l'ARQS revient � supposer que $\dfrac{r}{c} \ll T$.
\spartie{Conditions aux limites du champ e-m.}
Soit une interface entre deux milieux (ex : vide / m�tal). Dans le cas limite du mod�le surfacique, le champ 
$(\ve{E}, \ve{B})$ n'est pas d�fini sur l'interface et pr�sente une discontinuit� � sa travers�e.
On va montrer que la discontinu�t� du champ est donn�e par la relation~:
\encadre{\vbox{\hbox{$\ve{E}(P_2, t)-\ve{E}(P_1, t)=\dfrac{\sigma(M, t)}{\eps_0}\ve{n}_{12}$}%
               \hbox{$\ve{B}(P_2, t)-\ve{B}(P_1, t)=\mu_0\ve{K}(M, t)\wedge \ve{n}_{12}$}}}\par
\sspartie{Forme int�grale des �quations de Maxwell :}
A chacune des �quations de Maxwell locales on fait correspondre une �\-qua\-tion in\-t�\-grale~:\par
\centerline{{\renewcommand{\arraystretch}{2}
$\begin{array}{lll}
\rot\ve{E}=-\dpa{\ve{B}}{t} & \longrightarrow &%
              \integrale{L}{{}}{\ve{E}\cdot\ve{t}\dd{L}} =-\integrales{S}{{}}{\dpa{\ve{B}}{t}\cdot\ve{n}\dd{S}}\\
\div\ve{B}=0 & \longrightarrow & \integrales{S}{{}}{\ve{B}\cdot\ve{n}\dd{S}} = 0\\
\div{\ve{E}}=\dfrac{\rho}{\eps_0} & \longrightarrow & \integrales{S}{{}}{\ve{E}\cdot\ve{n}\dd{S}} = %
                                                                \dfrac{q_S}{\eps_0}\\
\rot{\ve{B}} = \mu_0(\ve{j}+\eps_0\dpa{\ve{B}}{t}) & \longrightarrow & %
                 \integrale{L}{{}}{\ve{B}\cdot\ve{t}\dd{L}} = \mu_0 I_S+\eps_0\mu_0%
                 \integrales{S}{{}}{\dpa{\ve{B}}{t}\cdot\ve{n}\dd{S}}
\end{array}$}}
\sspartie{Etablissement des conditions aux limites.}
\index{condition@Condition aux limites}
On commence par choisir une surface comme figur�e ci-dessous, de sorte que $\sigma_l$ soit n�gligeable par rapport 
� $\sigma$. C'est � dire que $\eps\partial\sigma\ll\sigma$, en notant $\partial\sigma$ le p�rim�tre de la surface
$\sigma$, elle m�me infinit�simale, de sorte que les densit�s surfaciques de charge et de courant y soient uniformes.\par
\centerline{\includegraphics*[width=10cm,height=3.5cm]{images/cond_limites.eps}}\par
En int�grant sur cette surface les deux �quations centrales ci-dessus, on trouve deus conditions aux limites.
Pour trouver les deux autres, on int�gre le long de cadres qui sont parall�les aux axes de coordonn�es $\vx,\vy$ et
poss�dent les m�mes propri�t�s que la bo�te ci-dessus.
\centerline{{\renewcommand{\arraystretch}{1.5}$\begin{array}{lll}
\rot\ve{E}=-\dpa{\ve{B}}{t} & \longrightarrow & \ve{E}_{2\para} = \ve{E}_{1\para}\\
\div{\ve{E}}=\dfrac{\rho}{\eps_0} & \longrightarrow & \ve{E}_{2\bot}-\ve{E}_{1\bot}=\dfrac{\sigma}{\eps_0}%
\ve{n_{12}}\\
\div\ve{B}=0 & \longrightarrow & \ve{B}_{2\bot} = \ve{B}_{1\bot}\\
\rot{\ve{B}} = \mu_0(\ve{j}+\eps_0\dpa{\ve{B}}{t}) & \longrightarrow & \ve{B}_{2\para}-\ve{B}_{1\para}=\mu_0%
\ve{K}\wedge\ve{n_{12}}
\end{array}$}}\par
En regroupant ces informations et en utilisant que le courant surfacique est tangent � l'interface, on obtient
les conditions aux limites annonc�es au d�but du paragraphe.
\sspartie{Equations de Maxwell g�n�rales.}
\index{Equation!maxwell@de Maxwell!generales@g�n�rales}
On g�n�ralise les �quations de Maxwell dans le cas des milieux polaris�s et aimant�s en introduisant les 
vecteurs moment dipolaire �lectrique $\ve{P}$ et moment dipolaire magn�tique $\ve{M}$. Ainsi que les vecteurs 
excitation �lectrique $\ve{D} = \eps_0\ve{E}+\ve{P}$ et excitation magn�tique $\ve{H}=\dfrac{1}{\mu_0}\ve{B}%
-\ve{M}$. Dans ces conditions, les �quations de Maxwell deviennent~:\par
\centerline{$\begin{array}{ll}
\rot\ve{E} = -\dpa{\ve{B}}{t} & \div \ve{D} = \rho\\
\div\ve{B} = 0                & \rot\ve{H}=\ve{j}+\dpa{\ve{D}}{t}
\end{array}$}

Et en faisant la m�me m�thode que ci-dessus on retrouve des conditions aux limites (plus simples que les 
pr�c�dentes)~:\par
\centerline{{\renewcommand{\arraystretch}{1.5}$\begin{array}{ll}
\ve{E_{2\para}} - \ve{E_{1\para}} = \ve{0} & \ve{B_{2\bot}} - \ve{B_{1\bot}} = \ve{0}\\
\ve{D_{2\bot}} - \ve{D_{1\bot}} = \sigma\ve{n_{12}} & \ve{H_{2\para}} - \ve{H_{1\para}} = \ve{K}\wedge\ve{n_{12}}
\end{array}$}}

\spartie{Energie �lectromagn�tique.}
\index{Energie!electromagnetique@�lectromagn�tique}
\sspartie{D�finition.}  
En calculant la divergence ci-dessous et en utilisant les �quations de Maxwell on d�montre la relation~:
\encadre{$\div\left(\dfrac{\ve{E}\wedge\ve{B}}{\mu_0}\right)+\dpa{{}}{t}\left(%
          \dfrac{\eps_0}{2}E^2+\dfrac{1}{2\mu_0}B^2\right)=-\ve{j}\cdot\ve{E}$}
Une telle �quation traduit le bilan d'une grandeur extensive~: l'Energie �lectromagn�tique.
$\ve{J_{em}} = \dfrac{\ve{E}\wedge\ve{B}}{\mu_0}$ est le courant volumique d'�nergie e-m, on l'appelle le vecteur
de Poynting.
\sspartie{Puissance �lectromagn�tique.}
\noindent Dans le cas g�n�ral, il suffit de calculer le flux du vecteur de Poynting~:
$P_{em}^{r} = \integrales{S}{{}}{\ve{J_{em}}\cdot\ve{n}\dd{S}}$\par
Dans le cas particulier du r�gime stationnaire, on �crit $\ve{E} = -\grad V$ et on simplifie le vecteur de 
Poynting. puis on recalcule le flux et on trouve~:
\encadre{$P_{em}^{r} =\integrales{S}{{}}{V\ve{j}\cdot\ve{n}\dd{S}}$}
\rmq Tout ceci n'est valable que dans la cas d'une fronti�re fixe.\par
On peut alors facilement calculer la puissance �lectromagn�tique re�ue par un dip�le �lectrocin�tique parcouru
par un courant d'intensit� i et soumis � une ddp de v~:
\encadre{$P_{em}^{r} = vi$}
\index{Champ!complexe}
\spartie{Notation complexe en �lectromagn�tisme.}
\sspartie{Th�or�me.}
Si le couple $(\underline{\ve{E}}, \underline{\ve{B}})$ est solution des �quations de Maxwell pour
la distribution $(\underline{\rho}, \underline{\ve{j}})$, alors le couple $(\ve{E}, \ve{B})$ est solution des 
�quations de Maxwell pour la distribution $(\rho, \ve{j})$.
\sspartie{Int�r�t de la notation complexe.}
Le calcul direct du champ r�el est souvent plus difficile que le calcul en trois �tapes suivants~:\par
\centerline{\begin{tabular}{ccc}
Distribution r�elle $(\rho, \ve{j})$ & $(\longrightarrow)$ & Champ r�el $(\ve{E}, \ve{B})$\\
$\downarrow$ &                                           & $\uparrow$\\
Distribution complexe $(\underline{\rho}, \underline{\ve{j}})$ & $\longrightarrow$ & Champ complexe%
                                                           $(\underline{\ve{E}}, \underline{\ve{B}})$
\end{tabular}}

\newcommand{\ecompl}{\underline{\ve{E}}}
\newcommand{\bcompl}{\underline{\ve{B}}}
\rmq Montrer le r�sultat suivant dans le cas du r�gime sinuso\"idal~:
\encadre{$<\ve{J_{em}}> = \re{\dfrac{\vecom{E}\land\vecom{B}^{\star}}{2\mu_0}}$}
Ce r�sultat n'en a pas l'air mais il est utile dans le cours sur les ondes �lectromagn�tiques.
\partie{Induction �lectromagn�tique.}
\index{Induction}
\spartie{G�n�ralit�s.}
\sspartie{D�finition.}
Un conducteur est le si�ge d'un ph�nom�ne d'induction d�s qu'apparaissent des courants induits par la mobilit� 
du conducteur ou par la non-stationnarit� du champ magn�tique qui y r�gne.
\point{Cas de Lorentz~:}
\index{Cas!lorentz@de Lorentz}
Le conducteur est mobile dans un champ stationnaire.
\point{Cas de Neumann~:}
\index{Cas!neumann@de Neumann}
Le conducteur est immobile dans un champ stationnaire.
\sspartie{Mise en �vidence exp�rimentale.}
\point{Cas de Lorentz~:}
\begin{itemize}
\item{rail~:} La translation d'une tige conductrice dans un champ uniforme stationnaire, engendre un courant induit
non-nul d�s que la vitesse de la tige est non-nulle.
\item{Roue de Barlow~:}
\index{Roue de Barlow} La rotation d'une roue conductrice dans un champ uniforme engendre un courant non-nul d�s 
que la vitesse angulaire de la roue est non-nulle.
\end{itemize}
\point{Cas de Neumann~:}
\begin{itemize}
\item{Bobine et aimant~:} Un aimant se d�place entre les spires d'une bobine et y engendre un courant �lectrique.
\item{Transformateur~:} Le montage est constitu� de deux parties, circuit primaire et secondaire, chacun constitu� 
d'un enroulement de fils et travers�s sans contact par un barreau conducteur. Le circuit primaire est aliment� par 
une tension sinusoidale. On constate l'apparition d'un courant dans le circuit secondaire.\index{Transformateur}
\end{itemize}
\sspartie{Freinage par induction.}
On consid�re un pendule conducteur qui passe dans un zone o\`u r�gne un champ magn�tique au cours de ses 
balancements. On constate alors le ralentissement du pendule. De m�me avec une roue subissant un champ. Cet effet 
peut �tre utilis� pour le freinage par induction des poids lourds.\index{Freinage par induction}
\sspartie{Chauffage par induction.}
Le courant induit dans un conducteur engendre un effet joule qui permet de chauffer. Par exemple de la soudure � 
l'�tain ou des tripoux sur des plaques � induction.\index{Tripoux}
\spartie{Champ �lectromoteur d'induction.}
Dans un conducteur m�tallique, la charge est nulle et donc $\ve{j} = \ve{j}^{\star}$. Le courant total de charge
satisfait donc � la loi d'Ohm locale~:
\encadre{$\ve{j} = \gamma(\ve{E}+\ve{v}\land\ve{B})$}
o\`u $\ve{v}$ d�signe la vitesse macroscopique du m�tal au point consid�r�. On d�finit alors le champ �lectromoteur d'induction 
comme la quantit� vectorielle~:
\encadre{$\ve{E_{em}} = \ve{v}\land\ve{B}-\dpa{\ve{A}}{t} = \dfrac{\ve{j}}{\gamma} + \grad V$}
Ce champ est homog�ne � un champ �lectrique et est nul en l'absence de ph�nom�ne d'induction, d'o\`u son nom.
\spartie{Ph�nom�ne d'induction dans un circuit filiforme.}
\sspartie{Force �lectromotrice d'induction.}
\index{Force!electromotrice@�lectromotrice}
\point{D�finition~:}
Etant donn� un circuit filiforme quelconque, on d�finit la f.e.m comme~:
\encadre{$e = \integrale{l}{}{\ve{E_m}\cdot\ve{t}\dd{l}}$}
\rmq Cette valeur d�pend de la convention choisie pour le sens de circulation.
\point{Expression de la d.d.p~:}
Le calcul de la f.e.m se fait facilement, et sachant que $R = \integrale{l}{}{\dfrac{\dd{l}}{\gamma s}}$ d�signe
la r�sistance �lectrique du circuit (mise en s�rie de tubes conducteurs �l�mentaires, de r�sistance 
$\delta R = \frac{\dd{l}}{\gamma s}$) on trouve l'expression suivante~:
\encadre{$e = Ri + V_B - V_A$}
o\`u $A$ et $B$ d�signent les deux extr�mit�s du circuit, �ventuellement confondues.\par
Cette expression est valable que le circuit soit ouvert ou ferm�. Dans le cas o\`u il est ouvert, l'ARQS donne $i=0$ et 
$V_A - V_B = -e$. Dans le cas o\`u il est ferm�, $e=Ri$.
\sspartie{Loi de Faraday}
\index{Loi!faraday@de Faraday}
On consid�re un cicuit filiforme ferm� et on note $\phi$ le flux du champ magn�tique total � travers ce circuit.
Calculons sa variation entre $t$ et $t+\dd{t}$, en notant $S$ et $S'$ des surfaces ferm�es s'appuyant sur le 
circuit, l'une � l'instant $t$, l'autre � l'instant $t+\dd{t}$. On note �galement $\ve{B}$ et $\ve{B'}$ le champ 
magn�tique � ces deux instants.\par
$\dd{\phi}=\integrale{S'}{}{\ve{B'}\cdot\ve{n}\dd{S}}-\integrale{S'}{}{\ve{B}\cdot\ve{n}\dd{S}}+
           \integrale{S'}{}{\ve{B}\cdot\ve{n}\dd{S}}-\integrale{S}{}{\ve{B}\cdot\ve{n}\dd{S}}$\par
On scinde alors sur les deux diff�rences d'int�grales. On note la premi�re $\delta \phim_N$ et l'autre $\delta\phim_L$
car elles correspondent respectivement � un cas de Neumann ou de Lorentz.\par
{\begin{minipage}{7cm}
{\begin{eqnarray*}
\delta\phim_L & = & \integrale{S'}{}{\ve{B}\cdot\ve{n}\dd{S}}-\integrale{S}{}{\ve{B}\cdot\ve{n}\dd{S}}\\
              & = & \integrale{\Sigma}{}{\ve{B}\cdot\ve{n}\dd{S}}\\
              & = & \integrale{l}{}{\ve{B}\cdot\bigl\lbrack (\ve{v}\dd{t})\land\ve{t}\dd{l}\bigr\rbrack}\\
              & = & \dd{t}\integrale{l}{}{\ve{B}\cdot(\ve{v}\land\ve{t})\dd{l}}\\
              & = & -\dd{t}\integrale{l}{}{(\ve{v}\land\ve{B})\cdot\ve{t}\dd{l}}
\end{eqnarray*}}
{\begin{eqnarray*}
\delta\phim_N & = & \integrale{S'}{}{\ve{B'}\cdot\ve{n}\dd{S}}-\integrale{S'}{}{\ve{B}\cdot\ve{n}\dd{S}}\\
              & = & \dd{t}\integrale{S'}{}{\dpa{\ve{B}}{t}\cdot\ve{n}\dd{S}}\\
              & = & \dd{t}\integrale{S'}{}{\rot\dpa{\ve{A}}{t}\cdot\ve{n}\dd{S}}\\
              & = & \dd{t}\integrale{l'}{}{\dpa{\ve{A}}{t}\cdot\ve{t}\dd{l}}\\
              & = & \dd{t}\integrale{l}{}{\dpa{\ve{A}}{t}\cdot\ve{t}\dd{l}}
\end{eqnarray*}}
\end{minipage}}
\includegraphics*[width=6cm,height=4cm]{images/faraday.eps}\par
En regroupant ces deux termes, on obtient la loi de faraday, qui n'est valable que pour un circuit ferm�~:
\encadre{$\df{\phi}{t} = -e$}
La loi de Faraday permet de calculer le courant induit circulant dans un circuit filiforme \textbf{ferm�}, en 
utilisant : $e = Ri$.
\sspartie{Exemples et applications.}
\point{Bobine d'induction~:}
On consid�re une bobine parcourue par un courant d'intensit� $i$. La bobine est un circuit ouvert mais la forme
curviligne du flux permet de d�finir le flux $\phi = \integrale{l}{}{\ve{A}\cdot\ve{n}\dd{l}}$. Un calcul trivial
montre alors que la relation de Faraday reste valable (ce qui n'�tait pas pr�visible puisque le circuit est ouvert). 
En utilisant la formule $\phi = Li$ d�finie en magn�tostatique, on obtient les relations �lectrocin�tiques de la bobine~:
\encadre{$v=Ri+L\df{i}{t}$}\index{Bobine}
\exercice Couplage par inductance mutuelle de deux circuits~: transformateur.\par\index{Transformateur}
On consid�re deux circuits $L,C$ dont l'un contient une source de tension sinuso�dale et l'autre pas, coupl�s par
induction : le flux $\phi$ � travers l'une ou l'autre bobine fait intervenir le coefficient d'inductance mutuelle.
(cf : matrice inductance)~:\par
\centerline{\includegraphics*[width=6cm,height=5cm]{images/transfo.eps}}\par
Montrer alors que le circuit priv� secondaire est parcouru par un courant, d'intensit� non-nulle d�s que le coefficient 
d'inductance mutuelle est non-nul.
\point{Alternateur~:}
\index{Alternateur}\index{Alternateur}
C'est un syst�me produisant du courant � partir d'un mouvement de rotation : on consid�re $N$ spires rectangulaires
de cot�s $a$ et $b$, de centre $O$ et tournant autour de l'axe $Oz$ � la vitesse stationnaire $\omega$, plac�e dans
une r�gion de l'espace o\`u r�gne un champ magn�tique uniforme stationnaire $\ve{B_0}$, normal � l'axe de rotation~:\par
\centerline{\includegraphics*[width=8cm,height=5cm]{images/alternateur.eps}}\par
\begin{itemize}
\item{Equation �lectromagn�tique~:}
On note $L$ et $R$ l'inductance et la r�sistance du circuit. Le circuit �tant une bobine, on peut appliquer la loi 
de Faraday. On d�compose ensuite le flux en flux propre qui s'exprime avec l'inductance et en flux ext�rieur, donn�
par un calcul direct. On aboutit � l'�quation~:\par
$$L\df{i}{t} + Ri = \omega\phi_0\sin(\omega t)$$\par
En notant $\phi_0 = N B_0 ab$ une quantit� homog�ne � un flux, qui appara�t dans les calculs. La r�solution de 
cette ED se fait sans probl�me en complexe, en n�gligeant le r�gime transitoire~:
\encadre{$i = I_0\sin(\omega t+\phim)$}
\item{Equation m�canique~:}
On applique au cadre suppos� ind�formable le th�or�me du moment cin�tique selon l'axe $\vz$, fixe (cf : m�canique).
Le moment d'inertie �tant stationnaire ainsi que $\omega$, le moment des forces ext�rieures est nul. Or celles
ci se r�duisent au moment des forces exerc�es par un moteur pour entretenir la rotation ($M_{op}$) et au forces
magn�tiques de l'aimant (on n�glige la viscosit� de l'air, on suppose l'axe vertical et la liaison pivot parfaite).
Or le moment des forces magn�tiques s'exprime gr�ce qu moment magn�tique du circuit~: 
$M_{aim}=(\ve{m}\land\ve{B_0})\cdot\vz$. On trouve alors~:
\encadre{$M_{op} = i\phi_0\sin(\omega t)$}
\item{Travail de l'op�rateur durant une p�riode~:}
Entre $t$ et $t+\dd{t}$ : $\delta\mathcal{T}_{op} = M_{op}\omega\dd{t}$. Soit~:\par
$Ri^2 + Li\dd{i}$ gr�ce aux formules �tblies pr�cedemment. Par int�gration~:
\encadre{$\mathcal{T} = \frac{1}{2}T R I_0^2 = \dfrac{\pi R\omega\phi_0^2}{R^2+L^2\omega^2}$}
\end{itemize}
\exercice V�rifier que les forces magn�tiques subies par un circuit filiforme de moment magn�tique $\ve{m}$, plong�
dans un champ uniforme, constitue un couple de moment $\ve{M} = \ve{m}\land\ve{B}$.\\
\textsl{Indication :} La m�thode est la m�me que pour un dip�le, on imagine un d�placement �l�mentaire et on applique
un th�or�me �nerg�tique.
\spartie{Induction dans un con\-duc\-teur de forme quel\-con\-que.}
L'�tude d'un conducteur malsain d'un point de vue forme peut �tre ramen�e � celle du circuit filiforme en 
d�composant le long des lignes de courant. Dans d'autres cas il vaut mieux revenir aux �quations de Maxwell.
\sspartie{Roue de Barlow.}
\index{Roue de Barlow|textbf}
On consid�re un disque conducteur de rayon $a$ en rotation autour de son centre plac� dans un champ magn�tique
uniforme normal au plan du disque. Ce disque est parcouru par un courant~: il est aliment� en son centre par un 
fil parcouru d'un intensit� $i$ et ce courant ressort par un fil plac� en bas du disque, le contact �tant mobile.
On suppose que la liaison pivot est parfaite. On note $\ve{B} = B\vz$ le champ uniforme et stationnaire dans lequel 
est plong� le disque, $\vz$ �tant le vecteur unitaire et de m�me sens que $\ve{\omega}$. On note �galement 
$v = V_O - V_A$ la diff�rence de potentiel au bornes du disque.\par
\centerline{\includegraphics*[width=6cm,height=4cm]{images/barlow.eps}}\par
\point{Equation �lectromagn�tique~:}
L'id�e est de d�composer le disque en un r�seau de circuits filiformes d'ex\-tr�\-mi\-t�s $O$ et $A$, suivant les 
lignes de courant au sein du disque, et parcourus par un courant $\delta i$. On note alors $\delta G$ l'inductance
infinit�simale du circuit. Alors $v = \frac{\delta i}{\delta G} - e$. Or $\ve{E_{em}} = 
\ve{v}\land\ve{B} - \dpa{\ve{A}}{t}\sim\ve{v}\land\ve{B_0}$. En n�gligeant le champ cr�e par le fil devant celui de
l'aimant. Au final $\ve{E_{em}}\sim B_0r\omega\vr$.  Puis par int�gration le long le long de la ligne de courant, en 
utilisant que $\ve{t}\dd{l} = \dd{\ve{r}} = \dd{r}\vr + r\dd{\theta}\vtheta$, on trouve~:
\encadre{$e = \dfrac{1}{2}B_0a^2\omega$}
Soit en regroupant les informations pr�c�dentes et en sommant tous les circuits filiformes �l�mentaires~:
\encadre{$Ri = v + \dfrac{1}{2}B_0a^2\omega$}
\point{Equation m�canique~:}
On applique le TMC autour de $Oz$ fixe dans le r�f�rentiel du labo galil�en. On introduit un frottement visqueux de
moment $-f\omega$ et en reprenant la d�composition en circuits filiformes on trouve~:
\encadre{$M_{op} = \dfrac{1}{2}m a^2\dot{\omega}+f\omega+\dfrac{1}{2}B_0a^2i$}
\rmq Dans le cas o\`u $M_{op}$ est nul, on peut d�terminer le mouvement de la roueen trouvant une �quation diff�rentielle
du premier ordre sur $\omega$. En exprimant le temps carat�risant l'arr�t, on remarque le champ magn�tique introduit une 
contribution analogue au frottement fluide. En outre la rotation de la roue est la cause du ph�nom�ne d'induction. C'est 
un cas particulier de la Loi de Lenz~:
\encadre{Les effets de l'induction s'opposent � leur cause.}
\index{Loi!lenz@de Lenz}
\sspartie{Four � induction.}
\index{Four@Four � induction}
\point{Principe~:}
On consid�re un sol�no�de infiniment long de rayon $a$ parcouru par un courant sinuso�dal. L'int�rieur est rempli
par un conducteur ohmique immobile. Le champ magn�tique n'�tant pas stationnaire, des courants induits appel�s ``courants
de Foucault'' apparaissent dans le conducteur et provoque par effet joule un �chauffement du m�tal. On veut d�terminer le 
champ de temp�rature au sein du m�tal.
\point{Equations de Maxwell~:}
\index{metal@M�tal}
Dans un m�tal immobile~: $\ve{j} = \gamma\ve{E}$. La valeur de $\gamma$ dans un m�tal usuel est de l'ordre de 
$10^8\;\Omega^{-1}m^{-1}$. En se pla�ant dans le cas sinuso�dal (quitte � s'y ramener par analyse de Fourier) et
aux fr�quences usuelles de l'�lectrocin�tique ($N\leq 100\;MHz$), le courant de d�placement $\ve{j_D} = 
\eps_0\dpa{\ve{E}}{t}$ est n�gligeable devant le courant de charge $\ve{j} = \gamma\ve{E}$. Dans ces conditions :
$\rot\ve{B} \sim \mu_0\ve{j} = \mu_0\gamma \ve{E}$, ce qui montre que $\div{E} = 0$ et donc que la charge v�rifie
$\rho = 0$. Dans un m�tal et aux fr�quences usuelles les �quations de Maxwell s'�crivent alors~:
\index{Equations!maxwell@de Maxwell!metal@dans un m�tal}
\centerline{{\renewcommand{\arraystretch}{1.5}
$\begin{array}{ll}
\div{\ve{E}} = 0 & \div{\ve{B}} = 0\\
\rot\ve{B} = \mu_0\gamma\ve{E} & \rot{E} = -\dpa{\ve{B}}{t}
\end{array}$}}
\point{Equation magn�tique~:}
Un calcul classique � partir des �quation ci-dessus fournit l'�quation de diffusion~:
\encadre{$\laplacien\ve{B} = \mu_0\gamma\dpa{\ve{B}}{t}$}
\point{Temp�rature dans le m�tal~:}
Elle peut �tre d�duite du bilan local d'�nergie interne~:\par
$$\div\ve{J_{th}} + \rho\df{u}{t} = \sigma_u$$\par
En appliquant successivement la loi de Fourier, le premier principe, la loi d'ohm et la relation $\dd{u} = c\dd{T}$ 
valable pour une phase condens�e, on ram�ne l'�quation � une �quation de diffusion avec second membre~:
\encadre{$-\lambda\laplacien T +\rho c\dpa{T}{t} = \dfrac{j^2}{\gamma}$}
Ces deux �quations de diffusion peuvent se r�soudre (difficile) pour obtenir le champ de temp�rature en tout point 
� tout instant.

\partie{Ondes �lectromagn�tiques}
\index{Onde!electromagnetique@�lectromagn�tique}
\spartie{Ondes �lectromagn�tiques dans le vide.}
\sspartie{Equations d'onde.}
\index{Equation!onde@d'onde}
Tout potentiel �lectromagn�tique satisfaisant � la jauge de Lorentz : $\div \ve{A} + \dfrac{1}{c^2}\dpa{V}{t}$ 
satisfait aussi aux �quations de d'Alembert qui s'�crivent dans le vide ($\rho = 0,\ve{j} = \ve{0}$)~:
\encadre{$\begin{array}{cc}
         \laplacien V = \dfrac{1}{c^2}\dpak{2}{V}{t} & \laplacien{\ve{A}} = \dfrac{1}{c^2}\dpak{2}{\ve{A}}{t}
          \end{array}$}\par
Les �quations de Maxwell dans le vide conduisent rapidement aux �quations d'onde~:\par
\encadre{$\begin{array}{cc}
         \laplacien \ve{E} = \dfrac{1}{c^2}\dpak{2}{\ve{E}}{t} & 
         \laplacien{\ve{B}} = \dfrac{1}{c^2}\dpak{2}{\ve{B}}{t}
          \end{array}$}
Le champ �lectromagn�tique est donc une onde de c�l�rit� $c$ telle que $\eps_0\mu_0c^2 = 1$.
\sspartie{Structure d'une OPPM �lectromagn�tique dans le vide.}
On �crit, conform�ment � la d�finition d'une OPPM, le champ �lectro\-ma\-gn�\-tique en notation complexe sous la 
forme~:\par
$\underline{\ve{E}}(\ve{r},t) = \underline{\ve{E}}_0 \oppm$ et $\underline{\ve{B}}(\ve{r},t) = \underline{%
\ve{B}}_0\oppm$\par
\index{OPPM!electromagnetique@�lectromagn�tique}
Or, pour une OPPM, on a des expressions simples des op�rateurs usuels, � savoir~:\par
$$\div \ecompl=i\ve{k}\cdot\ecompl,\quad\rot\ecompl=i\ve{k}\land\ecompl,\quad\dpa{\ecompl}{t}=-i\omega\ecompl$$
\par
\noindent Ce qui fournit facilement les relations entre les champs $\ecompl$ et $\bcompl$, qui s'�tendent au vecteurs 
r�els car $\ve{k}$ est un vecteur r�el et $\omega$ est un r�el~:\par
$$\ve{E} = -c^2\dfrac{\ve{k}}{\omega}\land\ve{B}\quad \ve{B} = \dfrac{\ve{k}}{\omega}\land\ve{E}$$\par
\index{Relation!dispersion@de dispersion}
\noindent Par une simple substitution, on trouve la relation de dispersion du vide pour les OPPM 
�lectromagn�tiques : $\omega = k c$. Cette relation �tant lin�aire, on en d�duit que le vide est un milieu 
non-dispersif pour les ondes �lectromagn�tiques. On peut introduire le vecteur $\ve{u} = \ve{k}/k$ indiquant 
le sens et la direction de propagation de l'OPPM qui permet de simplifier les �critures des �quations ci-dessus.
L'OPPM est dite transversale vu que le vecteur d'onde est normal au vecteur amplitude (imm�diat vu les relations
ci-dessus).
On retiendra les deux relations suivantes, qui donnent la structure de l'OPPM dans le vide~:
\encadre{$\begin{array}{lcl}
\ve{k} & = & \frac{\omega}{c}\ve{u}\\
\ve{E} & = & -c\ve{u}\land\ve{B}
\end{array}$}
\sspartie{Polarisation d'une OPPM e-m.}
Cette notion n'a de sens que pour des ondes vectorielles avec une composante transversale.
\point{D�finition~:}
Etant donn� les relations ci-dessus, on ne s'int�resse qu'au champ, $\ve{E}$. En partant de l'expression du 
champ complexe, on va calculer les composantes du champ r�el. On pose $\vz = \ve{u}$ et on compl�te en une base orthonorm�e
directe $(\vx,\vy,\vz)$. On peut alors �crire~: $\underline{\ve{E}}_0 = E_{0x}e^{i\alpha}\vx +  E_{0y}e^{i\beta}\vy$ et on note 
$\oppm = e^{i\phim}$ de sorte que~:\\
$\ve{E} = E_{0x}\cos(\phim+\alpha)\vx+E_{0y}\cos(\phim+\beta)\vy$
et en posant en plus~:\par
$\phim' = kz-\omega t+\alpha$ et $\phi = \alpha - \beta$ on a l'expression~:
\encadre{$E_{0x}\cos(\phim')\vx+E_{0y}\cos(\phim'-\phi)\vy$}
Si $\phi$ est quelconque, le vecteur $\ve{E}$ d�crit une ellipse, on dit que l'onde est polaris�e 
elliptiquement.
\index{Polarisation}
Si $\phi$ est nul, $\ve{E}$ d�crit un segment de longueur $2 E_0$, on dit que l'onde est polaris�e rectilignement.

Si $\phi = \pm \frac{\pi}{2}$ et $E_{0x} = E_{0y}$ alors le vecteur $\ve{E}$ d�crit un cercle de rayon $E_0$. On dit que l'OPPM est 
polaris�e circulairement. Si $\phi$ est positif on parle de circulaire droite, de gauche dans le cas contraire.

\point{Th�or�mes~:}
Une OPPM de polarisation quelconque peut toujours �tre consid�r�e comme la superposition de deux OPPM 
polaris�es rectilignements et de directions orthogonales.

Une OPPM polaris�e rectilignement peut toujours �tre consid�r�e comme la superposition de deux OPPM polaris�es 
circulairement dans des sens oppos�s.

La d�monstration de ces th�or�mes est triviale.
\sspartie{Intensit� d'une OPPM e-m dans le vide.}
\index{intensite@Intensit�!oppm@d'une OPPM}
Calculons la valeur moyenne du vecteur de Poynting pour une OPPM �lectromagn�tique en notation r�elle~:\par
$\begin{array}{lclclcl}
\ve{\Pi} & = &\frac{1}{\mu_0}\ve{E}\land\ve{B} & = & \frac{1}{\mu_0}\ve{E}\land \Bigl(\frac{\ve{u}}{c}\land\ve{E}\Bigr) & = &
  \frac{1}{\mu_0} \Bigl(-c\ve{u}\land\ve{B}\Bigr)\land\ve{B}\\
        & & & = & \eps_0cE^2\ve{u} & = &c\frac{1}{\mu_0}B^2\ve{u}
\end{array}$\par
On obtient, en prenant la valeur moyenne temporelle que  la valeur moyenne du vecteur de Poynting est proportionnelle � $\ve{u}$. On
d�finit l'intensit� de l'OPPM comme cette constante de proportionnalit� et on la note $I$~:
\encadre{$<\ve{\Pi}> = I\ve{u}$}
On retiendra de ces calculs les diverses expressions de cette intensit�~:
\encadre{$\begin{array}{cc}
           c\eps_0<E^2> & \frac{c}{\mu_0}<B^2>\\
           \frac{c\eps_0}{2}\ecompl_0 \cdot \ecompl_0^{\star} & \frac{c}{2\mu_0}\ecompl_0 \cdot \ecompl_0^{%
                 \star}
           \end{array}$}\par
D'un point de vue interpr�tation physique, l'intensit� d'une OPPM se propageant dans la direction et le sens de 
$\ve{u}$ est la valeur moyenne temporelle de la puissance e-m qui tranvers une surface unit� orthogonale � 
$\ve{u}$ dans le sens de $\ve{u}$.
\exercice D�finir et calculer la puissance d'un laser �mettant une lumi�re monochromatique polaris�e 
circulairement � l'int�rieur d'un cylindre de section droite $s_0$.
\spartie{Ondes �lectromagn�tiques dans un milieu optique.}
\sspartie{Notion de milieu optique.}
\index{Milieu!optique}
Un milieu optique est un milieu polaris� non-aimant�. Il n'est nicharg�, ni parcouru par un courant. Les �quations 
de Maxwell g�n�rales (cf : 2.3.5) s'�crivent alors~:\par
\centerline{$\begin{array}{ll}
\rot\ve{E} = -\dpa{\ve{B}}{t} & \div\ve{D} = 0\\
\div\ve{B} = 0 & \rot\ve{B} = \mu_0 \dpa{\ve{D}}{t}\end{array}$}\par
En optique lin�aire, le vecteur $\ve{D}$ est li� au champ �lectrique $\ve{E}$ par la relation matricielle~:\par
\centerline{$\begin{pmatrix}
               D_x\\
               D_y\\
               D_z
             \end{pmatrix} = 
             \begin{pmatrix}
               \eps_x & 0 & 0\\ 
               0 & \eps_y & 0\\
               0 & 0 & \eps_z
             \end{pmatrix}
             \begin{pmatrix}
              E_x\\
              E_y\\
              E_z
             \end{pmatrix}$}\par
En se pla�ant dans une base de vecteurs propres de la matrice (On admet qu'elle est diagonalisable). Un milieu 
optique est dit homog�ne si cette matrice est uniforme, isotrope si elle est scalaire.
\sspartie{Milieu optique homog�ne isotrope.}
La relation matricielle pr�c�dent s'�crit alors $\ve{D} = \eps \ve{E}$. $\eps$ est une constante que l'on peut mettre
sous les formes~:\par
$\eps = \eps_0 \eps_r$ o\`u $\eps_r$ est la constante di�lectrique du milieu, sans dimension.\par
$\eps = \eps_0 n^2$ en notant $n = \sqrt{\eps_r}$ l'indice optique du milieu.
Les �quations de Maxwell s'ecrivent alors comme celles du vide � condition d'effectuer les substitutions~:\par
\encadre{$c\longrightarrow \dfrac{c}{n}\quad\eps\longrightarrow \eps_0n^2$}
La structure de l'OPPM est donc la m�me que dans le vide, � une constante multiplicative pr�s. L'intensit� de l'OPPM
a pour expression~: $I = \dfrac{c\eps_0 n}{2}\ecompl\cdot\ecompl^{\star}$\par
\sspartie{Milieu homog�ne anisotrope.}
On se contente d'�tudier le cas d'une lame polaro�d de normale $\vz$ qui se comporte comme un milieu optique
transparent pour une OPPM polaris�e rectilignement dans le sens de $\vx$ et comme un miroir pour une OPPM polaris�e
selon l'axe $vy$. L'axe $\vx$ est nomm� l'axe du polaro�d. Une OPPM polaris�e de fa�on quelconque qui rencontre ce 
milieu est partiellement transmise. Le champ �lectrique $\ve{E_{\tau}}$ transmis est polaris� selon $\vx$.
\exercice Loi de Malus. 
On fait arriver une OPPM polaris�e rectilignement dans une direction $\ve{u}$ formant un angle $\theta$ avec l'axe
du polaro�d. Calculer l'intensit� de l'OPPM transmise connaissant celle de l'onde incidente. On trouve~:
\encadre{$I_{\tau} = I \cos^2\theta$}
\index{Loi!malus@de Malus}
\sspartie{Milieu optique isotrope mais inhomog�ne.}
On r��crit les �quations de Maxwell g�n�rales en prenant garde que l'indice n'est plus uniforme~:\par
\centerline{$\begin{array}{ll}
 \rot\ve{E} = -\dpa{\ve{B}}{t} & \div(n^2\ve{E}) = 0\\
 \div\ve{E} = 0               & \rot\ve{B} = \dfrac{n^2}{c^2}\dpa{\ve{E}}{t}
             \end{array}$}
\point{Recherche d'une solution particuli�re des �quations de Maxwell~:}
Une simple OPPM ne convient plus, il faut chercher une solution sous la forme~:\par
\centerline{$\ecompl(\ve{r},t) = \ecompl_0(\ve{r})e^{i(k_0 R(\ve{r}) - \omega t)}$}\par
avec $k_0 = \omega / c$. En substituant dans Maxwell-amp�re et Maxwell-Faraday on trouve les relations~:\par
\centerline{$\begin{array}{l}
               \ecompl_0\land\grad R + c\bcompl_0 = -\dfrac{i}{k_0}\rot{\ecompl_0}\\
               \bcompl_0\land\grad R - \dfrac{n^2}{c}\ecompl_0 = -\dfrac{i}{k_0}\rot{\bcompl_0}
             \end{array}$}
\point{Approximation de l'optique g�om�trique~:}
\index{Approximation!geometrique@de l'optique g�om�trique}
Elle consiste � supposer $\rot\ecompl_0\ll k_0c\bcompl_0$ et $c\rot\bcompl_0\ll k_0n^2\ecompl_0$. Auquel cas les 
relations ci-dessus s'�crivent~:
\encadre{$\begin{array}{lr}
               \ecompl_0\land\grad R + c\bcompl_0 = \ve{0} & (1)\\
               \bcompl_0\land\grad R - \frac{n^2}{c}\ecompl_0 = \ve{0} & (2)
             \end{array}$}
On montre alors que les deux autres �quations de Maxwell sont satisfaites. Puis, en reportant $(1)$ dans $(2)$ on
obtient la relation importante~:\par
$\lbrack(\grad R)^2 - n^2\rbrack \ecompl_0 = \ve{0}$ ce qui fournit l'existence d'un vecteur $\ve{u}$ unitaire tel 
que $\grad R = n\ve{u}$ et on peut r��crire les �quations de structure~:
\encadre{$\begin{array}{l}
            \ve{E} = -\dfrac{c}{n}\ve{u}\land \ve{B}\\
            \ve{B} = n\dfrac{\ve{u}}{c}\land\ve{E}\end{array}$}
Ce qui rappelle celles du vide sauf que cette fois $\ve{u}$ n'est pas uniforme! On dit que le champ e-m a 
localement la structure d'une OPPM se propageant dans la direction et le sens de $\ve{u}$.
\point{D�finition d'un rayon lumineux~:}
C'est une ligne de courant d'�nergie �lectromagn�tique.
\index{rayon@Rayon lumineux}
Comme $\ve{\Pi}$ le vecteur de Poynting est de colin�aire et de m�me sens que le vecteur $\ve{u}$ on en d�duit que
$\ve{u}$ est tangent au rayon lumineux et que sons sens indique le sens de propagation de la lumi�re.
\point{Equation curviligne d'un rayon~:}
On se place le long d'un rayon lumineux et on y choisit deux points voisins $M$ et $M'$. On effectue ensuite la
diff�rentielle de la relation $\grad R = n\ve{u}$. Un calcul astucieux permet ensuite d'affirmer que~:
\encadre{$\df{n\ve{u}}{s} = \grad R$}
Dans le cas particulier o\`u $n$ est uniforme on retrouve que le rayon lumineux est une droite.
\point{Th�or�me de Fermat~:}
\index{theoreme@Th�or�me!fermat@de Fermat}
Avec ce qu'on a fait ci dessus on peut d�montrer le th�or�me de Fermat, habituellement �rig� en principe : �tant 
donn�e une courbe $L$ voisine d'un rayon lumineux $L_0$, l'int�grale $\int_L n\dd{L}$ est �gale � l'ordre 1 au chemin
optique le long de $L_0$.
\encadre{$(M_1 M_2) = \integrale{L}{}{n\dd{L}}$}
\point{Surface d'onde, th�or�me de Malus~:}
On appelle surface d'onde, ou surface �quiphase toute surface sur laquelle la phase est uniforme. Une surface d'onde
est donc caract�ris�e par une valeur uniforme de $R$.\par
Etant donn� que $\grad R$ est normal � une telle surface et que $n\ve{u} = \grad R$, les rayons lumineux sont
orthogonaux aux surfaces d'onde (de m�me que les lignes de champ �taient orthogonales aux surfaces �quipotentielles
en �lectrostatique).
\point{Identification de $R$ � un chemin optique~:}
\index{chemin@Chemin optique}
En introduisant le rayon lumineux passant par $M$, qui coupe la surface d'onde correspndant � $R=0$ en $M_0$, et en
utilisant le th�or�me ci-dessus on trouve~:
\encadre{$R(M) = (M_0 M)$}
On peut identifier le champ scalaire $R$ au chemin optique ainsi d�fini.
\point{Domaine de validit� de l'approximation~:}
Si on analyse pr�cis�ment l'approximation effectu�e pour faire tout cela, on constate que l'approximation de 
l'optique g�om�trique revient � supposer que les variations de l'amplitude du champ varie peu en valeur relative �
l'�chelle de la longueur d'onde.\par
Dans le cas contraire, les lois de l'optique g�om�trique doivent �tre remplac�es par celles de la diffraction.
\sspartie{Appoximation scalaire de l'optique g�om�trique.}
\index{Approximation!scalaire}
Elle consiste � remplacer l'onde �lectromagn�tique $(\ve{E}, \ve{B})$ se propageant dans un milieu optique d'indice 
$n$ par une onde scalaire $s$.\par
On se place dans une r�gion de l'espace o\`u les lois de  l'optique g�om�trique sont valables, dans laquelle on 
choisit deux point $M_1$ $M_2$ situ�s sur un m�me rayon alors l'onde scalaire $s$ v�rifie la relation~:
\encadre{$\dfrac{s(M_2, t)}{s(M_1, t)} = a_{12}e^{i\frac{2\pi}{\lambda}(M_1 M_2)}$}
On d�montre cette relation dans tous les cas usuels de l'optique g�om�trique~: faisceau cylindrique dans un milieu
homog�ne, faisceau conique et milieu inhomog�ne.
\spartie{R�flexion d'une OPPM sur un conducteur parfait plan.}
\sspartie{Onde incidente, Onde r�fl�chie.}
Une source �met une OPPM �lectromagn�tique qui se propage avant d'arriver sur un plan conducteur parfait situ� � 
droite du plan $z=0$. on d�sire d�terminer le champ en tout point de l'espace.
\point{Dans le conducteur parfait~:}
Le champ em y est nul en r�gime sinuso�dal et le conducteur ne peut �tre charg� ou parcouru par un courant que sur
l'interface $z=0$.\par
En effet dan sun conducteur parfait, $\gamma\to\infty$ et la loi d'ohm pour un conducteur immobile assure 
$\ve{E}\to\ve{0}$. En r�gime sinuso�dal l'�quation de Maxwell Faraday permet d'affirmer $\ve{B} = \ve{0}$. 
Puis Maxwell-Gauss et Maxwell-Amp�re donnent $\rho = 0$ et $\ve{j} = \ve{0}$ en l'int�rieur du conducteur. De plus 
l'OPPM engendre n�c�ssairement une distribution de charge ou de courant � la surface du conducteur. En effet dans le 
cas contraire le champ � l'int�rieur du conducteur serait r�duit de l'OPPM, non-nul.
\point{Dans le vide~:}
Il suffit de superposer � l'OPPM incidente une OPPM r�fl�chie, engendr�e par la distribution de charge et de courant
� la surface du conducteur, not�e $(\ve{K},\sigma)$.\par
Les conditions aux limites � l'interface vide conducteur s'�crivent~:\par
$\begin{array}{l}
-\ve{E}(x,y,0^-,t) = \dfrac{\sigma(x,y,t)}{\eps_0}\vz\\
-\ve{B}(x,y,0^-,t) = \mu_0\ve{K}(x,y,t)\land\vz
\end{array}$\par
On note alors $\ve{E_0},\ve{B_0}$ les amplitudes de l'OPPM incidente $(\ve{E_i},\ve{B_i})$ et $\ve{E_0'},\ve{B_0'}$ 
celles de l'OPPM r�fl�chie $(\ve{E_r},\ve{B_r})$, ainsi que $\ve{k},\ve{k'},\omega,\omega'$ leurs vecteurs d'onde
et pulsations respectifs.
\rmq Comme l'OPPM incidente v�rifie les �quations de Maxwell et que le champ total superposition de l'incidente et
de la r�fl�chie les v�rifie aussi par principe, par lin�arit�, le champ r�fl�chi les v�rifie aussi.\par
Les conditions aux limites s'�crivent alors de fa�on plus p�nible mais en faisant intervenir les deux champs 
pr�c�dents.
\sspartie{Lois de Descartes relatives � la r�flexion.}
\index{Loi!descartes@de Descartes}
Les conditions aux limites imposent~:\par
$\ve{E_{0,\para}}e^{i(k_xx+k_yy-\omega t)}+\ve{E_{0,\para}'}e^{i(k_x'x+k_y'y-\omega' t)}=\ve{0}$\par
Cette condition doit �tre v�rifi�e pour tout $x,y,t$ ce qui impose $\omega = \omega'$, $k'=k$ et $\ve{k_{\para}}=
\ve{k_{\para}}$. Les deux seules solutions sont alors que $\ve{k} = \ve{k'}$ ce qui impossible (le v�rifier) et
$\ve{k'}$ sym�trique de $\ve{k}$ par rapport au plan de r�flexion. Cela implique les lois de Descartes relatives �
la reflexion, c'est-�-dire que les plans d'incidence et de reflexion sont confondus et que les angles d'incidence
et de reflexion sont �gaux.
\sspartie{D�termination de l'OPPM r�fl�chie.}
Les conditions aux limites et les relations sur les amplitudes permettent de d�terminer les quatre amplitudes 
inconnues avec un calcul simple dans le cas d'une OPPM polaris�e rectilignement selon $\vy$ puis selon $\vz$,
le cas g�n�ral s'en d�duisant par superposition.
\sspartie{Cas d'une incidence normale.}
On suppose $\ve{u} = \vz$. Alors apr�s quelques calculs on trouve~:\par
$$\left\lbrace\begin{array}{l}
\ve{E}(M,t) = 2E_0\sin(\omega t)\sin(kz)\vx\\
\ve{B}(M,t) = \dfrac{2 E_0}{c}\cos(kz)\cos(\omega t)\vy
\end{array}\right.$$\par
On constate que le champ a une structure d'onde stationnaire.
\sspartie{Pression �lectromagn�tique.}
\index{Pression!electromagnetique@�lectromagn�tique}
Les lois de De broglie\index{Loi!broglie@de De Broglie} affirment qu'� chaque OPPM correspond une particule et
r�ciproquement. Cette dualit� se traduit par les relations~:
\encadre{$\begin{array}{l}
            \ve{p} = \hbar\ve{k}\\
            E = \hbar \omega\end{array}$}
En particulier � toute OPPM �lectromagn�tique est associ� un photon de quantit� de mouvement $\ve{p} = 
\dfrac{h\nu}{c}\ve{u}$ et d'�nergie $E = h\nu$.\par
Les photons arrivant sur le plan $z=0$ y exercent une force $\delta\ve{F}$ par �l�ment de surface $\delta S$. On
note $\delta^2 N$ le nombre de photons arrivant sur $\delta S$ pendant $\delta t$ (grand devant la p�riode du photon
de l'ordre de $10^{-15}$. Alors en notant $\ve{p_n}$ et $\ve{p_n'}$ la quantit� de mouvement du photon $n$ avant
int�raction et apr�s int�raction, la physique statistique (fait en td) nous apprend que la force a pour expression :
$\delta\ve{F} = \dfrac{1}{\delta t}\serie{n=1}{\delta^2N}{\ve{p_n}-\ve{p_n'}}$, ce qui donne $2\dfrac{h\nu}{c}
\cos(i_0)\vz$, ce qui montre que : $\delta\ve{F} = 2\dfrac{\delta^2N}{\delta t}\dfrac{h\nu}{c}\cos i_0\vz$. Or 
$h\nu\delta^2 N$ est l'�nergie e-m re�ue par $\delta S$ dans l'intervalle $\segment{t_1}{t_2}$ de longueur 
$\delta t$. Or cette �nergie peut �tre calcul�e gr�ce au vecteur de Poynting~:\par
$h\nu\delta^2 N = \integrale{t_1}{t_2}{\ve{\Pi}\cdot\vz\delta S\dd{t}}$. Or on a suppos� que $\delta t$ �tait grand
devant la p�riode de l'OPPM, c'est-�-dire que l'on peut assimiler $\Pi$ � sa valeur moyenne temporelle, ce qui 
permet de calculer l'int�grale simplement. Apr�s un bref calcul on obtient~:
\encadre{$\delta \ve{F} = \dfrac{2 I}{c}\cos^2i_0\delta S\vz$}
En faisant intervenir l'intensit� $I$ de l'OPPM, d�finie pr�c�demment. Cette force est proportionnelle � $\delta S$
et est analogue � la force de pression, soit :$\delta{\ve{F}} = p\delta S \vz$ avec :
\encadre{$p = 2\dfrac{I}{c}\cos^2i_0$}
Cette pression e-m est � l'origine de nombreux ph�nom�nes, comme la courbure de la queue des com�tes, la 
l�vitation d'une bille conductrice dans un faisceau laser ou le d�placement d'une voile solaire.
\exercice Calculer la pression e-m subie par une bille conductrice, plong�e dans un faisceau laser cylindrique et
stationnaire.
\spartie{Propagation guid�e, guide d'onde rectangulaire.}
\index{Propagation!guidee@guid�e}
\sspartie{Description.}
Le guide d'onde est un tuyau m�tallique creux dont la section est un rectangle de c�t�s $a>b$. Le mat�riau composant
le guide est assimil� � un conducteur parfait alors que l'air remplissant le tuyau est est consid�r� comme du vide.
On montre qu'un tel guide permet la transmission d'un signal em dont les longueurs d'ondes sont de l'ordre des 
dimensions transversales du guide.
\sspartie{Recherche d'une solution.}
On ne fait ici que rappeler les �tapes de la r�solution du probl�me, les calculs restant � faire.\\
On recherche un signal de la forme $f(Kz-\omega t)$ dans le vide inter-conducteurs, susceptible de traduire une 
propagation. La recherche d'une simple OPPM est vou�e � l'�chec du fait des conditions aux limites au voisinage des 
parois. Au final on recherche une onde de la forme :
\encadre{$\ve{E} = f(x,y)e^{i(Kz-\omega t)}\vy$}
\rmq On a choisi de chercher une onde transversale �lectrique polaris�e selon $\vy$ car le guide n'est pas carr�
et on verra qu'une onde selon $\vx$ n'�tait pas � privil�gier.
\sspartie{D�termination de la solution.}
Les �quations de Maxwell dans le vide fournissent directement les �quations~:\par
$$\div\ve{E} = 0\quad \dalembert\ve{E} = \ve{0}$$\par
En substituant dans ces �quations le champ �lectrique, on obtient des conditions sur $f$~:
\encadre{$\begin{array}{l}
            \dpa{f}{y} = 0\\
            \noalign{\vspace{1mm}}
            \dpak{2}{f}{x} + \left(\dfrac{\omega^2}{c^2}-K^2\right)f(x) = 0
          \end{array}$}
Ce qui montre que $f$ ne d�pend pas de $y$ (le domaine de d�finition de $f$ est un ouvert convexe) et qu'on a 
forc�ment $\dfrac{\omega^2}{c^2}-K^2 > 0$ pour avoir des solutions born�es. Les conditions aux limites sur les 
parois permettent alors d'obtenir la charge et le courant surfacique ainsi que la condition 
$\sqrt{\dfrac{\omega^2}{c^2}-K^2}\in \dfrac{\pi}{a}\zz$. C'est-�-dire qu'on a des modes propres et que~:\par
L'onde T-E cherch�e est une superposition de solutions particuli�res, appel�es modes propres qui s'�crivent~:
\encadre{$\ve{E_n}=A_ne^{i(n\frac{\pi}{a}x+K_nz-\omega t)}\vy-(-1)^{n}A_ne^{-i(n\frac{\pi}{a}x+K_nz-\omega t)}\vy$}
Avec $K_n^2 = \frac{\omega^2}{c^2}-n^2\frac{\pi^2}{a^2}$.
\sspartie{Interpr�tation.}
Ce mode peut �tre interpr�t� comme la superposition de deux OPPM de polarisation �lectrique rectiligne selon $\vy$,
se prpageant dans le vide � la c�l�rit� $c$ dont les vecteurs d'onde �voquent des r�flexions successives sur les
parois conductrices.
\sspartie{Mode fondamental.}
Ces r�flexions sont suivies d'une att�nuation du signal du fait des pertes d'�nergie e-m au sein du conducteur qui
en pratique n'est pas rigoureusement parfait. L'att�nuation globale est d'autant  plus faible que l'angle entre
la paroi et la direction de propagation de l'OPPM est faible. Le mode $n = 1$ est donc pr�dominant apr�s quelques
r�flexions. Pour cette raison, on supposera qu'il est seul � se propager dans le guide. On trouve donc un champ 
�lectrique r�el de la forme~:
\encadre{$\ve{E} = E_1\cos(\pi\frac{x}{a})\cos(K_1z-\omega t)\vz$}
Et le champ magn�tique s'en d�duit avec les �quations de Maxwell.
\sspartie{Fr�quence de coupure.}
Il n'y a propagation que si $K_1> 0$, ce qui se traduit par~:
\encadre{$\lambda<\lambda_c\quad\mathrm{avec}\quad\lambda_c = 2a$}
\sspartie{Vitesse de propagation.}
On peut trouver le r�sultat avec diff�rentes m�thodes et interpr�tations (optique ou �nerg�tique). La plus simple
consiste � calculer $c\cos\theta_1$ ce qui donne la vitesse de propagation de l'information :
\encadre{$v_g = c\sqrt{1-\left(\dfrac{\omega}{\omega_c}\right)^2}$}
\spartie{Propagation libre, rayonnement dipolaire.}
\index{Propagation!libre}
On consid�re un dip�le �lectrique oscillant selon une loi sinuso�dale~:\par
\centerline{\includegraphics*[width=4cm,height=4cm]{images/dipole_osc.eps}}\par
o\`u $\ve{p}(t) = p_0\cos(\omega t)\vz$.
\sspartie{D�termination des champs et potentiels.}
\point{Potentiel vecteur~:}
On admet par analogie avec les potentiels retard�s que le potentiel vecteur engendr� au point $P$ par le dip�le 
oscillant est de la forme~:\par
$$\ve{A} = \dfrac{\mu_0}{4\pi}\integrale{\tau}{}{\dfrac{\ve{j}(M,t-\frac{r}{c})}{r}\dd{V}}$$\par
En notant $r$ la distance entre le point source $M$ et le point d'observation $P$. Comme on se place dans le cadre
de l'approximation dip�laire, on peut faire un DL et assimiler $r$ � la distance entre le point source et l'origine.
Ici, le potentiel vecteur peut se mettre sous la forme~:\par
$\ve{A} = -\dfrac{\mu_0}{4\pi}\dfrac{\omega}{r}\sin(\omega(t-\frac{r}{c}))p_0\vz$\par
Ce qui, en notation complexe implicite donne~:
\encadre{$\ve{A} = -i\dfrac{\alpha}{r}e^{i(kr-\omega t)}\vz$}
en notant $\alpha = \dfrac{\mu_0}{4\pi}p_0\omega$ et $k = \frac{\omega}{c}$.
\point{Potentiel scalaire~:}
On cherche un potentiel vecteur qui satisfasse � la jauge de Lorentz~:\\
$\div\ve{A} + \eps_0\mu_0\dpa{V}{t} = 0$. Cette relation donne en complexe que 
$V = -i\dfrac{c^2}{\omega}\div\ve{A}$. Un calcul de divergence en complexe fournit alors le potentiel scalaire 
�lectrique~:
\encadre{$V = \dfrac{\alpha c^2}{\omega r^2}(ikr -1)e^{i(kr -\omega t)}\cos\theta$}
\point{Champ � grande distance $(r\gg\lambda)$~:}
Si on suppose qu'on est � grande distance du dip�le, le potentiel scalaire se simplifie en~:\par
$$-i\dfrac{\alpha}{r}ce^{i(kr -\omega t)}\cos\theta$$\par
On peut alors en d�duire le champ �lectromagn�tique � partir des relations~:\par
$\ve{E} = -\grad V -\dpa{\ve{A}}{t}$ et $\ve{B} = \rot\ve{A}$.\par
Un calcul assez sanglant (1 page) donne le r�sultaten notation r�elle~:
\encadre{\begin{tabular}{c}
$\ve{E} = -\dfrac{\mu_0}{4\pi}p_0\omega^2\cos(kr -\omega t)\dfrac{\sin\theta}{r}\vtheta$\\
$\ve{B} = -\dfrac{\mu_0}{4\pi}\dfrac{p_0\omega^2}{c}\cos(kr -\omega t)\dfrac{\sin\theta}{r}\vphi$
\end{tabular}}
On constate qu'� grande distance~:
\encadre{\begin{tabular}{c}
$\ve{B} = \dfrac{\vr}{c}\land\ve{E}$\\
$\ve{E} = -c\vr\land\ve{B}$
\end{tabular}}
Ce qui montre que le champ a localement la structure d'une OPPM.
\sspartie{Puissance rayonn�e par le dip�le.}
La valeur moyenne du bilan local d'�nergie �lectromagn�tique montre que la puissance rayonn�e � travers une surface
$\Sigma$, � savoir~:\par
$$P = \integrale{\Sigma}{}{<\ve{\Pi}>\cdot\ve{n}\dd{S}}$$\par
ne d�pend pas du choix de la surface ferm�e consid�r�e. On peut calculer cette puissance en calculant l'intensit� de
l'OPPM locale~:
\encadre{$I(r,\theta) = \dfrac{\alpha^2 \omega^2}{2\mu_0 c}\dfrac{\sin^2\theta}{r^2}$}
On constate que cette intensit� d�pende de $\theta$, d'o\`u le caract�re anistrope du rayonnement dip�laire.
Ensuite pour calculer $P$ on prend une sph�re de rayon $R$ et on trouve~:
\encadre{$P = \dfrac{\mu_0}{4\pi}\dfrac{p_0^2\omega^4}{3c}$}
\rmq La partie difficile est la structure d'OPPM locale. Pour retrouver rapidement le r�sultat, on peut se contenter
de l'admettre, auquel cas c'est beaucoup plus rapide.
\point{Propagation guid�e-propagation libre~:}
Gr�ce � la propagation guid�e, on n'a pas de perte d'intensit� avec la distance mais cela n�cessite un grande 
infrastructure mat�rielle, contrairement � la propagation libre.
\sspartie{Vue d'ensemble des ondes �lectromagn�tiques.}
Un petit sch�ma permet de visualiser la classification des ondes em selon leur longueur d'onde~:\par
\centerline{\includegraphics*[width=12cm,height=2.5cm]{images/ondes_em.eps}}\par


\include{thermodynamique/thermodynamique}
\include{thermochimie/thermochimie}
\chapter{M�canique}\index{mecanique@M�canique}
\partie{R�f�rentiels}\index{referentiel@R�f�rentiel}
\spartie{D�finitions.}
\sspartie{R�f�rentiels.}
On appelle r�f�rentiel un ensemble d'au moins trois points immobiles les uns par rapport aux autres.\\
Cela n�c�ssite l'introduction d'un rep�re d'espace. En m�canique classique le temps est invariant par
changement de r�f�rentiel. Un rep�re d'espace suffit � d�finir un r�f�rentiel d'o\`u la notation~: 
${\cal{R}} (O, \ve{u_x}, \ve{u_y}, \ve{u_z})$.\par
\rmq Si un r�f�rentiel est d�fini � partir d'un rep�re d'espace, le contraire est faux.
\sspartie{R�f�rentiel d'observation.}
C'est le r�f�rentiel dans lequel sont d�finies les grandeurs cin�matiques, ci\-n�\-ti\-ques et dynamiques du 
syst�me �tudi�.
\spartie{Changement de r�f�rentiel.}\index{referentiel@R�f�rentiel!changement}
\sspartie{Vecteur rotation de $\cal{R}$ par rapport � $\cal{R}'$.}
Le mouvement de $\cal{R}'(\vx', \vy', \vz')$ par rapport au r�f�rentiel d'observation $\cal{R}$ peut �tre 
d�crit par~:\par
\begin{itemize}
\item{} Le mouvement de l'origine $O'$ $/\cal{R}$~: $\ve{v_{O'}}$
\item{} Le mouvement de la base $(\vx', \vy', \vz')$ $/\cal{R}$~: $\ve{\omega_{\cal{R}'/\cal{R}}}$.
\end{itemize}
En effet on a la relation matricielle~:\par
$\df{{}}{t}
\begin{pmatrix}
\vx'\\
\vy'\\
\vz'
\end{pmatrix} = A
\begin{pmatrix}
\vx'\\
\vy'\\
\vz'
\end{pmatrix}$ 
La d�riv�e �tant prise dans le r�f�rentiel ${\cal{R}}$. En exprimant que les vecteurs $\vx', \vy', \vz'$ 
forment une base orthonorm�e de l'espace on trouve que $A$ est une matrice antisym�trique. Ce qui permet 
d'affirmer qu'il existe un vecteur $\ve{\omega}$ 
\index{Vecteur!rotation}
tel que~:\par
$\df{{}}{t}
\begin{pmatrix}
\vx'\\
\vy'\\
\vz'
\end{pmatrix} = \ve{\omega}\land
\begin{pmatrix}
\vx'\\
\vy'\\
\vz'
\end{pmatrix}$\par 
En plus on montre facilement que $\ve{\omega}$ ne d�pend pas du rep�re d'espace choisi pour d�finir l'un 
et l'autre r�f�rentiel. C'est pourquoi on le note~:
\encadre{$\ve{\omega} = \ve{\omega_{\cal{R}'/\cal{R}}}$}
\sspartie{Th�or�me relatif au changement de r�f�rentiel~}
Soient deux r�f�rentiels quelconques $\cal{R}$ et $\cal{R}'$ ainsi qu'un vecteur $\ve{V}$ de l'espace.
On a l'identit�~:\par
$\ve{V} = V_x\vx + V_y\vy + V_z\vz = V_x'\vx' +V_y'\vy' +V_z'\vz'$ qu'on peut d�river dans le r�f�rentiel
$\cal{R}$. En utilisant alors la propri�t� fondamentale du vecteur $\ve{\omega}$ on a la relation~:
\encadre{$\Bigl(\df{\ve{V}}{t}\Bigr)_{\cal{R}} = \Bigl(\df{\ve{V}}{t}\Bigr)_{\cal{R}'}+
          \ve{\omega_{\cal{R}'/\cal{R}}}\land\ve{V}$}
\sspartie{Th�or�mes annexes.}
\point{Trois r�f�rentiels~:}
\encadre{$\ve{\omega_{\cal{R}'/\cal{R}}}=\ve{\omega_{\cal{R}'/\cal{R}''}}+
          \ve{\omega_{\cal{R}''/\cal{R}}}$}
\point{Transformation de vitesses~:}
Etant donn�s deux points $A$ et $B$ d'un m�me r�f�rentiel $\cal{R}'$, leurs vitesses $\ve{v_A},
\ve{v_B}$ dans un autre r�f�rentiel $\cal{R}$ satisfont � la relation~:
\encadre{$\ve{v_A} = \ve{v_B}+\ve{\omega_{\cal{R}'/\cal{R}}}\land\ve{AB}$}
La d�monstration est tr�s simple, par contre ce corrollaire � une tr�s grande importance pratique dans le
cours de m�canique de solide.
\point{Lois de composition des vitesses et acc�l�rations~:}
\index{Loi!composition@de Composition des vitesses}
\encadre{$\begin{array}{l}
          \ve{v_P} = \ve{v_P}' + \ve{v_{P_c}}\\
          \ve{a_P} = \ve{a_P}' + \ve{a_{P_c}}+ 2 \ve{\omega}\land\ve{v_P}'
          \end{array}$}
O\`u on a fait les notations suivantes~:
$\ve{v_P} = \Bigl(\df{\ve{OP}}{t}\Bigr)_{\cal{R}}$, $\ve{v_P}' = \Bigl(\df{\ve{O'P}}{t}\Bigr)_{\cal{R}'}$
$\ve{a_P} = \Bigl(\df{\ve{v_P}}{t}\Bigr)_{\cal{R}}$, $\ve{a_P}' = \Bigl(\df{\ve{v_P}'}{t}\Bigr)_{\cal{R}'}$ et
$\ve{\omega} = \ve{\omega_{\cal{R}'/\cal{R}}}$\par
Et on d�finit le point $P_c$ comme le point de $\cal{R}'$ coincidant avec le point $P$ � l'instant $t$.\par
Pour d�montrer ce th�or�me, on fait le calcul en introduisant une origine et on applique le th�or�me ci-dessus.
On obtient une identit� qu'on sait �tre vraie pour $P_c$, ce qui permet de simplifier l'expression obtenue.
\rmq Les termes suppl�mentaires qui apparaissent dans ces lois de composition portent des noms~: 
$\ve{v_{P_c}}$ et $\ve{a_{P_c}}$ sont qualifi�s de vitesse et acc�l�ration d'entra�nement alors que
le terme~: $2 \ve{\omega}\land\ve{v_P}'$ se nomme acc�l�ration de Coriolis.
\spartie{Classification des mouvements d'un r�f�rentiel par rapport � un autre.}
Etant donn�s deux r�f�rentiels, le mouvement de $\cal{R}'$ par rapport � $\cal{R}$ peut �tre class� selon
le nombre de points communs � ces r�f�rentiels~:\\
$\va{\cal{R}\cap\cal{R}'} = 0$ et $\ve{\omega}\neq \ve{0}$~: Mouvement le plus g�n�ral.\\
$\va{\cal{R}\cap\cal{R}'} = 0$ et $\ve{\omega}= \ve{0}$~: Mouvement de translation.\\
$\va{\cal{R}\cap\cal{R}'} = 1$~: Mouvement de rotation par rapport � un point fixe.\\
$\va{\cal{R}\cap\cal{R}'} = 2$~: Mouvement de rotation par rapport � un axe fixe.\\
$\va{\cal{R}\cap\cal{R}'} = 3$~: Les deux r�f�rentiels sont confondus.
\sspartie{Translation.}
Il d�coule imm�diatement de la d�finition que tous les points de $\cal{R}'$ ont m�me vitesse dans $\cal{R}$.
En effet il suffit d'appliquer le second corrolaire ci dessus.
\sspartie{Rotation autour d'un axe fixe.}\index{Rotation!axe@autour d'un axe fixe}
On note $O$ et $A$ deux points communs � $\cal{R}'$ et � $\cal{R}$. En appliquant le second corollaire
on obtient imm�diatement que $\ve{\omega}\land\ve{OA} = \ve{0}$. C'est-�-dire qu'il existe $k\in\rr[\star]$
tel que $\ve{\omega} = k \ve{OA}$. Si on note $(O, \vz)$ cet axe $(OA)$ et $(r, \theta, z)$ les coordonn�es
cylindriques dans $\cal{R}$ d'un point quelconque $P$ de $\cal{R}'$~:\\
$\ve{v_P} = \ve{v_O} + \ve{\omega}\land\ve{OP} = r\omega\ve{u_{\theta}}$. Et comme de plus $\ve{v_P} = 
r point\vr + r\dot{\theta}\vtheta + \dot{z}\vz$. D'o\`u les relations~:
\encadre{$r = cste;\quad\dot{\theta} = \omega;\quad z = cste$}
Cela traduit que $\cal{R}'$ est en rotation autour de $(Oz)$.\par
\rmq On retiendra que $\theta$ d�signe un angle entre une direction $(0x)$ fixe dans $\cal{R}$ et une 
direction $(Op)$ fixe dans $\cal{R}'$ dont la valeur alg�brique est d�finie en conformit� avec le sens de 
$(Oz)$. (Cette remarque n'en a pas l'air comme �a mais elle est tr�s importante!).
\sspartie{Rotation autour d'un point fixe.}
\index{Rotation!point@autour d'un point fixe}
On note $O$ le seul point fixe commun aux deux r�f�rentiels. La m�me m�thode fournit la relation~:
\encadre{$\ve{v_P} = \ve{\omega}\land\ve{OP}$}
\exercice Calculer le vecteur rotation du r�f�rentiel ${\cal{R}'}(0, \vr, \vtheta, \vphi)$ et retrouver � 
l'aide de la loi de transformation des vitesses l'expression de la vitesse d'un point en coordonn�es 
sph�riques. On pourra introduire des r�f�rentiels interm�diaires en rotation autour d'axe fixes.
\sspartie{R�f�rentiels confondus.}
Dans le cas o\`u $\cal{R}$ et $\cal{R}'$  ont trois points communs(ou plus), les r�f�rentiels sont �gaux.
\spartie{R�f�rentiel du centre de masse.}
\index{referentiel@R�f�rentiel!centre@du centre de masse}
Etant donn� un syst�me mat�riel et un r�f�rentiel d'observation $\cal{R}$, on appelle r�f�rentiel du centre
de masse et note $\cal{R}^{\star}$ le r�f�rentiel en translation � la vitesse $\ve{v_G}$ du centre de masse
du syst�me mat�riel.

\noindent Avec ce r�f�rentiel, les lois de composition s'�crivent~:
\encadre{$\begin{array}{l}
          \ve{v_P} = \ve{v_P}^{\star} + \ve{v_{G}}\\
          \ve{a_P} = \ve{a_P}^{\star} + \ve{a_{G}}
          \end{array}$}
%--------------------------------------------------PARTIE--------------------------------------------------
\partie{Th�or�mes g�n�raux de la m�canique}
\spartie{Notion de torseur.}
\index{Torseur}
\sspartie{D�finition.}
Un champ vectoriel est une application de $\rr[3]$ dans lui m�me. On s'int�resse ici � des champs vectoriels
affines. C'est-�-dire que si on note $\ve{M_A}$ la valeur du champ au point $A$, il existe un application
lin�aire $u$ ou encore un matrice $\cal{M}$ telle que~: $\ve{M_B} - \ve{M_A} = \cal{M}\ve{AB}$. On 
qualifie le champ vectoriel d'antisym�trique lorsque la matrice $\cal{M}$ est antisym�trique. On sait
alors qu'il existe un vecteur $\ve{V}$ appel� ``vecteur caract�ristique du champ'' tel que~:
\encadre{$\ve{M_B} = \ve{M_A} + \ve{V}\land \ve{AB}$}
On d�finit alors le torseur associ� au champ vectoriel comme le couple $(\ve{M_A}, \ve{V})$ qu'on note~:
\encadre{$\cal{T} = \left\lvert\begin{matrix}\ve{M_A}\\
                                             \ve{V}\end{matrix}\right.$}
Le moment est enti�rement d�termin� par son valeur en un point $A$. C'est pourquoi on ne distingue pas deux 
torseurs dont la seule diff�rence est le point o\`u est pris le moment. Pour les d�finir intrins�quement on
pourrait passer � des classes d'�quivalence mais �a n'a pas d'int�r�t pratique. On ne s'attache pas non
plus � l'ordre entre moment et vecteur caract�ristique, du moment qu'on sait de quoi on parle. Ainsi parfois
le moment sera en bas et d'autres fois en haut.
\sspartie{``Alg�bre'' torsorielle.}
\point{Egalit� de deux torseurs~:}
$$\cal{T} = \cal{T}' \equivaut \left\lvert\begin{array}{l}
                                           \ve{V} = \ve{V}'\\
                                           \exists A\;|\; \ve{M_A} = \ve{M_A'}
                                         \end{array}\right.$$
\point{Somme de deux torseurs~:}
$${\cal{T}} = {\cal{T}}_1 + {\cal{T}}_2  \equivaut \left\lvert\begin{array}{l}
                                                          \ve{V} = \ve{V}_1 + \ve{V}_2\\
                                                          \ve{M_A} = \ve{{M_{1}}_A} + \ve{{M_{2}}_A}
                                                       \end{array}\right.$$
\point{Produit par un scalaire~:}
$$\cal{T} = \lambda\cal{T}' \equivaut \left\lvert\begin{array}{l}
                                           \ve{V} = \lambda\ve{V}'\\
                                           \ve{M_A} = \lambda\ve{M_A'}
                                         \end{array}\right.$$
\point{Produit de deux torseurs~:}
C'est l� que �a ne marche pas pour avoir une vraie alg�bre en effet ce produit ne d�finit pas une loi 
interne~: \par
$$\cal{T}\cal{T}' = \ve{V}\cdot\ve{M_A}' + \ve{V}'\cdot\ve{M_A}$$\par
Et cette d�finition ne d�pend pas du point $A$ choisi.
\point{Projection d'un torseur sur un axe~:}
Etant donn� un axe $(O,\ve{u})$ et un torseur $\cal{T} = \left\lvert\begin{matrix}\ve{M_O}\\
                                             \ve{V}\end{matrix}\right.$ on appelle projection du torseur sur
l'axe la quantit�~: $$M_{\Delta} = \ve{u}\cdot\ve{M_O}$$ 
Et �a ne d�pend pas du point $O$ choisi sur l'axe.
\index{Collection de pointeurs}
\sspartie{Torseur associ� � une collection de pointeurs.}
\point{D�finition~:}
Etant donn� une collection de pointeurs $(A_i, \ve{V_i})$, le champ vectoriel 
$\ve{M_A} = \sum\limits_{i}\ve{AA_i}\land\ve{V_i}$ est antisym�trique. Son vecteur caract�ristique est le 
vecteur $\ve{V} = \sum\limits_{i}V_i$.\\
On peut alors d�finir le torseur associ� � la collection de pointeurs~:
$$\cal{T} = \left\lvert\begin{array}{ll}\sum\limits_{i}\ve{V_i}&\mathrm{R\acute esultante}\\
                                     \sum\limits_{i}\ve{AA_i}\land \ve{V_i}&\mathrm{moment\;au\;point\;A}
                       \end{array}\right.$$
\point{Equivalence torsorielle de deux collections de pointeurs~:}
On dit que deux collections de pointeurs sont torsoriellement �quivalents lorsque leurs torseurs associ�s
sont �gaux.\par
\index{Torseur!equivalent@�quivalence}
Par exemple une collection de pointeurs parall�les $(A_i, \ve{V_i} = V_i \ve{u})$ est torsoriellement
�quivalente � un pointeur unique $(B, \sum\limits_{i}\ve{V_i})$, o\`u $B$ d�signe le barycentre des points
$A_i$ affect�s des coefficients $V_i$. \par
Il en est ainsi des forces de pesanteur~:\par
Les forces de pesanteur qui s'exercent sur un syst�me de masse totale $M$ sont torsoriellement �quivalentes
� une force unique $M\ve{g}$ qui s'exerce au centre de masse $G$ du syst�me.\par
(Encore une remarque qui n'en a pas l'air mais qui est tr�s importante!).
\spartie{Torseur cin�tique.}
\index{Torseur!cinetique@cin�tique}
\sspartie{D�finition.}
Etant donn� un syst�me mat�riel et un r�f�rentiel d'observation, le torseur cin�tique de ce syst�me dans 
$\mathcal{R}$ est celui associ� � la collection de pointeurs $(A_i, \ve{p_i})$ o\`u $\ve{p_i}$ d�signe
la quantit� de mouvement du point mat�riel $A_i$. On note ce torseur~:\par
$$\mathcal{C} = \left\lvert\begin{array}{ll}\ve{P} = \sum\limits_{i}m_i\ve{v_i}&
                                                  \mathrm{R\acute esultante\;cin\acute etique}\\
                                            \ve{L_A} = \sum\limits_{i}\ve{AA_i}\land \ve{p_i}&
                                                  \mathrm{Moment\;cin\acute etique\;au\;point\;A}
                           \end{array}\right.$$
Le moment cin�tique est un champ vectoriel antism�trique, d'o\`u la relation~:
\encadre{$\ve{L_B} = \ve{L_A} + \ve{P}\land\ve{AB}$}
La projection du torseur cin�tique sur un axe est appel�e moment cin�tique par rapport � cet axe.
\sspartie{Autre expression de la r�sultante cin�tique.}
\index{resultante@R�sultante cin�tique}
Par d�finition~: $\ve{P} = \sum\limits_{i}m_i\ve{v_i}$. En supposant le syst�me ferm� de masse stationnaire
et en appliquant la d�finition du centre de masse on a la relation~:
\encadre{$\ve{P} = M\ve{v_G}$}
En particulier dans le r�f�rentiel du centre de masse, vu que $\ve{v_G}^{\star} = \ve{0}$, on a~:
\encadre{$\ve{P}^{\star} = \ve{0}$}
\index{Moment!cinetique@cin�tique}
\rmq De ce fait le moment cin�tique d'un syst�me exprim� dans son r�\-f�\-ren\-tiel du centre de masse a m�me 
valeur en tout point. (encore une remarque importante).
\sspartie{Th�or�me de K\oe nig relatif au moment cin�tique.}
\index{theoreme@Th�or�me!koenig@de K\oe nig}
\encadre{$\ve{L_G} = \ve{L_G}^{\star}$}

\sspartie{Extensivit� du torseur cin�tique.}
Le torseur cin�tique est un grandeur extensive. En effet on peut d�finir une densit� volumique de r�sultante
cin�tique ($\rho\ve{v}$) et une densit� volumique de moment cin�tique ($\rho\ve{r}\land\ve{v}$). (encore
un truc important...)
\spartie{Torseur dynamique.}
\index{Torseur!dynamique}
\sspartie{D�finition.}
Le torseur dynamique d'un syst�me mat�riel dans un r�f�rentiel est celui associ� � la collection de 
pointeurs ($A_i, \df{\ve{p_i}}{t}$) o\`u $\df{\ve{p_i}}{t}$ d�signe la quantit� d'acc�l�ration du point 
$A_i$ dans $\mathcal{R}$~:\par
$$\mathcal{D} = \left\lvert\begin{array}{ll}\ve{P} = \sum\limits_{i}m_i\ve{a_i}&
                                                  \mathrm{R\acute esultante\;dynamique}\\
                                            \ve{L_A} = \sum\limits_{i}\ve{AA_i}\land \df{\ve{p_i}}{t}&
                                                  \mathrm{Moment\;dynamique\;au\;point\;A}
                           \end{array}\right.$$
\sspartie{Relation entre $\cal{C}$ et $\cal{D}$.}
En g�n�ral, on a la relation suivante~:
\encadre{Moment dynamique au point $A = \df{\ve{L_A}}{t} + M\ve{v_A}\land \ve{v_G}$}
\index{Moment!dynamique}
Ce qui est bien dommage, du coup on se limite � deux cas particuliers~: $A$ est un point commun
aux deux r�f�rentiels ou $A=G$. Pour ces deux points on a la relation bien plus pratique~:
\encadre{$\mathcal{D} = \df{\mathcal{C}}{t}$}
\spartie{Torseur force.}
\index{Torseur!force}
\sspartie{D�finition.}
Le torseur force d'un syst�me mat�riel est celui associ� � la collection de pointeurs $(A_i, \ve{f_i})$, 
o\`u $\ve{f_i}$ d�signe une force d'un type donn� subie par le point mat�riel $A_i$.\par
$$\mathcal{F} = \left\lvert\begin{array}{ll}\ve{F} = \sum\limits_{i}\ve{f_i}&
                                                  \mathrm{R\acute esultante}\\
                                            \ve{M_A} = \sum\limits_{i}\ve{AA_i}\land \ve{f_i}&
                                                  \mathrm{Moment\;au\;point\;A}
                           \end{array}\right.$$
Le moment des forces est un champ vectoriel antisym�trique~:
\encadre{$\ve{M_B} = \ve{M_A} + \ve{F}\land\ve{AB}$}
\rmq Le moment par rapport � un axe $\Delta$ d'un force $\ve{F}$ appliqu�e au point $P$ se calcule 
facilement~:\par
\centerline{\includegraphics*[width=9cm,height=4cm]{images/moment_axe.eps}}\par
$M_{\Delta} = \ve{u}\cdot \ve{M_p} = \ve{u}\cdot(\ve{pP}\land\ve{F}) = 
\ve{u}\cdot(\ve{pH}\land\ve{F_{\bot}})$. On en d�duit~:
\encadre{$M_{\Delta} = \pm d F_{\bot}$}
on choisit $+$ ou $-$ selon que la force a tendance � faire tourner le bras de levier dans le sens positif
ou n�gatif.
\sspartie{Cas particulier o\`u $F = 0$.}
\index{Couple}
Un ensemble de forces dont la r�sultante est nulle est appel�e un ``couple''. Son moment a m�me valeur en 
tout point.
\point{Exemple} Les forces �lectriques subies par un dip�le de moment $\ve{p}$ plong� dans un champ
�lectrostatique uniforme constitue un couple de moment $\ve{p}\land\ve{E}$.

\sspartie{Cas particulier o\`u $M_A = 0$.}
Un ensemble de forces dont le moment en un point $A$ est nul est torsoriellement �quivalent � une force
unique appliqu�e en $A$ et �gale � la r�sultante de ces forces. 
\spartie{Principe fondamental et th�or�mes g�n�raux.}
\sspartie{Principe fondamental.}
\index{Principe!meca@de la m�canique}
\begin{itemize}
\item{} Le torseur dynamique d'un syst�me ferm� quelconque dont le mouvement est observ� par rapport � un 
r�f�rentiel quelconque, est la somme du torseur des forces ext�rieures et du torseur des forces d'inertie~:
$$\mathcal{D} = \mathcal{F}_{ext} + \mathcal{F}_{inert}$$\par
\index{referentiel@R�f�rentiel!galileen@galil�en}
\item{} Il existe des r�f�rentiels dits galil�ens, dans lesquels les forces d'inertie sont nulles.
\end{itemize}
\point{Commentaires} On notera que le syst�me est ferm�, donc de masse stationnaire.
On notera ensuite que les forces int�rieures n'interviennent pas alors qu'elles interviendront dans les
th�r�mes �nerg�tiques.\par
Enfin, l'utilisation de ce principe n�cessite des d�finitions pr�cises du syst�me, du r�f�rentiel...
\sspartie{Th�or�mes g�n�raux.}
\index{theoreme@Th�or�me!resultante@de la r�sultante cin�tique}
\point{Th�or�me de la r�sultante cin�tique~:}
L'�galit� torsorielle implique l'�galit� des r�sultantes.
\encadre{$\df{\ve{P}}{t} = \ve{F_{ext}}+ \ve{F_{inert}}$}
\exercice Calculer la force de pouss�e d'un fus�e connaissant la vitesse d'�jection des gaz et leur
d�bit massique $D$.\par
\centerline{\includegraphics*[width=8cm,height=6cm]{images/fusee.eps}}\par\index{Gagarine Youri}
On choisit pour syst�me l'ensemble carcasse+�quipement+carburant contenu dans le r�servoir � l'instant $t$+
youri gagarine. On note $M(t)$ la masse de ce syst�me � l'instant $t$, $D(t)$ le d�bit massique et $u(t)>0$ 
la vitesse des gaz � la sortie par rapport au r�f�rentiel de la carcasse. Appliquer alors le th�or�me de la 
r�sultante cin�tique au syst�me dans le r�f�rentiel terrestre suppos� galil�en. En notant $v(t)$ la vitesse de
la carcasse par rapport au sol, exprimer la r�sultante cin�tique eu fonction des forces ext�rieures et d'un
terme compl�mentaire $D u$.
\point{Th�or�me du moment cin�tique~:}
\index{theoreme@Th�or�me!moment@du moment cin�tique}
La m�me �galit� torsorielle implique aussi l'�galit� des moments en un point quelconque. En pratique on 
choisit un point $O$ du r�f�rentiel d'observation ou le centre de masse $G$. Par exemple pour le point O~:
\encadre{$\df{\ve{L_O}}{t} = \ve{M_{O,ext}}+\ve{M_{O,inert}}$}
Et de m�me pour le centre de masse. On peut projeter cette relation sur un axe fixe passant par le point O~:
\encadre{$\df{L_{\Delta}}{t} = M_{\Delta,ext} + M_{\Delta, inert}$}
Et de m�me pour un axe mobile mais de direction stationnaire passant par $G$.\par
\rmq On appelle ``point mat�riel'' tout syst�me dont l'�tude m�canique peut �tre assur�e par le seul
th�or�me de la r�sultante cin�tique.
\point{Th�or�me d'interaction~:}
\index{theoreme@Th�or�me!interaction@d'interaction}
Par principe les forces int�rieures d'un syst�me quelconque sont invariantes par changement de r�f�rentiel.
Pour d�montrer une de leurs propri�t�s on peut donc choisir un r�f�rentiel galil�en. On commence par 
effectuer une partition du syst�me $\mathcal{S} = \bigcup\limits_{i\in I}\mathcal{S}_i$. Le torseur 
dynamique �tant une grandeur extensive on a~:
$\mathcal{D} = \sum\limits_{i\in I}\mathcal{D}_i = \sum\limits_{i\in I}\mathcal{F}_{ext, i} + 
\sum\limits_{\begin{array}{c}(i,j)\in I^2\\ i\neq j\end{array}}\mathcal{F}_{j/i}$. Et comme $\mathcal{D} = \mathcal{F}_{ext}$
on fait appara�tre que $\sum\limits_{\begin{array}{c}(i,j)\in I^2\\ i\neq j\end{array}}\mathcal{F}_{j/i} = 0$ d'o\`u le 
th�or�me d'interaction~:
\encadre{Le torseur des forces int�rieures d'un syst�me mat�riel est nul}
\exercice Montrer que l'�tude m�canique (cin�matique, cin�tique et dynamique) d'un syst�me isol� de deux 
point mat�riels $A_1$ et $A_2$ de masses $m_1$ et $m_2$ peut �tre ramen�e � celle de deux points mat�riels 
fictifs $G(M = m_1+m_2)$ et $A(\mu = \frac{m_1 m_2}{m1+m_2})$ tel que $G$ soit le centre de masse du syt�me 
et $\ve{OA} = \ve{A_1 A_2}$. (On se place dans un r�f�rentiel galil�en).
\sspartie{Torseur des forces d'inertie.}
\index{Force!inertie@d'inertie}
L'utilisation du principe fondamental de la dynamique n'est posible que si on conna�t le torseur des forces
d'inertie.
\point{D�termination dans le cas g�n�ral~:}
On introduit un r�f�rentiel galil�en interm�diaire (par principe il en existe) dans lequel le PFD s'�crit~:
$\mathcal{F}_{ext} = \mathcal{D}^{(g)}$. En effet les forces ext�rieures ne d�pendent pas du r�f�rentiel).
On a alors~: $\mathcal{F}_{inert} = \mathcal{D}-\mathcal{D}^{(g)}$ ce qui montre que $\mathcal{F}_{inert}$ 
est le torseur associ� � la collection de pointeurs $(A_i, m_i(\ve{a_i} - \ve{a_i}^{(g)}))$ et la loi
de composition des acc�l�rations permet ce calcul. En conclusion~:\par
$$\mathcal{F}_{inert} = \left\lvert\begin{array}{ll}%
   \sum\limits_{i}m_i(\ve{a_{i,ent}}+\ve{a_{i,cor}})& \mathrm{R\acute esultante}\\
   \sum\limits_{i}m_i\ve{AA_i}\land(\ve{a_{i,ent}}+\ve{a_{i,cor}})&\mathrm{Moment\;au\;point\;A}
                        \end{array}\right.$$
\point{Cas particulier de $\cal{R}$ en translation $/{\cal{R}}^{(g)}$~:}
Dans ce cas tout se simplifie puisque $\ve{a_{i,ent}} = \ve{a}$ et $\ve{a_{i,cor}} = \ve{0}$. En prenant
le moment au centre de masse on trouve~:\par
$$\mathcal{F}_{inert}\left\lvert\begin{array}{l}
                      -M\ve{a}\\
                      \ve{0}\end{array}\right.$$
\spartie{Energie m�canique.}

\sspartie{Puissance d'un ensemble de forces.}
\index{Puissance}
\point{D�finitions~:}
Etant donn� un syst�me mat�riel soumis � une distribution de forces $(A_i, \ve{f_i})$, on appelle puissance
de ces forces dans un r�f�rentiel $\mathcal{R}$ la quantit�~:
\encadre{$P = \sum\limits_{i}\ve{f_i}\cdot\ve{v_i}$}
o\`u $\ve{v_i}$ d�signe la vitesse dans ${\cal{R}}$ du point $A_i$.
\index{Travail}
On introduit �galement le travail �l�mentaire des ces forces~:
\encadre{$\delta\mathcal{T} = P\dd{t} = \sum\limits_{i}\ve{f_i}\cdot\ve{v_i}$}
\point{Puissances des forces int�rieures~:}
\encadre{La puissance des forces int�rieures est invariante par changement de r�f�rentiel.}
En effet, si on consid�re deux r�f�rentiels, $P_{int}-P_{int}'$ se calcule gr�ce � la loi de 
transformation des vitesses en introduisant $A_{i_c}$ le point de $\mathcal{R}'$ qui coincide avec $A_i$ � 
l'instant $t$~: $P_{int}-P_{int}' = -\sum\limits_{i}\ve{v_{i_c}}\cdot\ve{f_i}$ et comme $A_{i_c}$ et
$O'$ sont deux points du m�me r�f�rentiel~: $\ve{v_{i_c}} = \ve{v_{O'}} + \ve{\omega}\land\ve{OA_{i_c}}$.
On fait alors appara�tre r�sultante et moment des forces in�rieures qui sont nuls d'apr�s le th�or�me
d'interaction.
\point{Puissance des forces de Coriolis~:}
\encadre{La puissance des forces de Coriolis est nulle.}
\sspartie{Energie Cin�tique.}
\point{D�finition~:}
\encadre{$E_c = \sum\limits_{i}\frac{1}{2}m_i v_i^2$}
L'�nergie cin�tique ne d�pend que du mouvement du syst�me mat�riel.

\point{Th�or�me de K\oe nig~:}
\index{theoreme@Th�or�me!koenig@de K\oe nig}
\encadre{$E_c = \frac{1}{2}Mv_G^2 + E_c^{\star}$}
O\`u $E_c^{\star}$ d�signe l'�nergie cin�tique du syst�me dans le r�f�rentiel du centre de masse.\par
\exercice D�terminer l'�nergie cin�tique d'un chenille de dameuse qui roule sans glisser en fonction de la 
vitesse $\ve{v}$ de son centre de masse $G$.\par
\centerline{\includegraphics*[width=8cm,height=2.5cm]{images/dameuse.eps}}
\point{Th�or�me de l'�nergie cin�tique~:}

\index{theoreme@Th�or�me!energie@de l'�nergie cin�tique}
\encadre{$\dd{E_c} = E_c(t+\dd{t})-E_c(t) = \delta\mathcal{T}$}
O\`u $\delta\mathcal{T}$ d�signe le travail �l�mentaire de toutes les forces qui agissent sur le syst�me 
mat�riel.\par
En effet $\dd{E_c} = \dd{\Bigl(\sum\limits_i \frac{m_i}{2}v_i^2\Bigr)} = \sum\limits_i m_i\ve{v_i}\cdot
\ve{a_i}\dd{t}$ Or le th�or�me de la r�sultante cin�tique appliqu� au point $A_i$ nous assure que 
$m_i\ve{a_i} = \ve{f_i}$. O\`u $\ve{f_i}$ d�signe la r�sultante de toutes les forces qui s'exercent sur le 
point $A_i$. On d�duit le th�or�me.
\sspartie{Energie potentielle.}
\index{Energie!potentielle|textbf}
\point{D�finition~:}
On dit qu'un ensemble de forces relatives � un syst�me mat�riel d�rive d'une �nergie potentielle lorsque le
travail de ces forces peut �tre mis sous la forme~:
\encadre{$\delta\mathcal{T} = -\dd{E_p}$}
O\`u $E_p$ est une quantit� qui ne d�pend que de la position du syst�me mat�riel.\par
\exercice D�finir et calculer l'�nergie potentielle  d'un syst�me de deux points mat�riels �lastiquement li�s.
\point{Cas des forces de pesanteur~:}
Les forces de pesanteur subies par un syst�me mat�riel d�rivent d'une �nergie potentielle~:
\encadre{$E_p = -M\ve{g}\cdot\ve{OG} = M g z_G$}
en notant $z_G$ l'altitude du centre de masse.
\point{Cas des forces d'inertie~:}
On suppose le r�f�rentiel d'observation en rotation stationnaire autour d'un axe fixe dans un r�f�rentiel 
galil�en. Les forces d'inertie d�rivent alors d'une �nergie potentielle~:
\encadre{$E_p = -\dfrac{1}{2}I\omega^2$}
O\`u $I = \sum\limits_i m_i r_i^2$ d�sigle le moment d'inertie du syst�me mat�riel par rapport � l'axe.
\point{Cas des forces de frottements~:}
Les forces de frottements de glissement entre deux solides ne d�rivent pas d'une �nergie potentielle comme
le montrent les lois de Coulomb (III).
\sspartie{Energie potentielle d'un syst�me mat�riel.}
On d�finit l'�nergie potentielle d'un syst�me mat�riel comme la somme de toutes les �nergies potentielles
dont d�rivent un ensemble de forces qui s'exercent sur le syst�me.
\sspartie{Energie m�canique.}
\index{Energie!mecanique@m�canique|textbf}
\point{D�finition~:}
On d�finit l'�nergie m�canique d'un syst�me mat�riel comme la somme de son �nergie cin�tique et de son 
�nergie potentielle.
\point{Th�or�me de l'�nergie m�canique~:}
C'est un corollaire du th�or�me de l'�nergie cin�tique~:
\index{theoreme@Th�or�me!energie@de l'�nergie m�canique}
\encadre{$\dd{E_m} = \delta\mathcal{T}'$}
O\`u $\delta\mathcal{T}'$ d�signe le travail de toutes les forces s'exer�ant sur le syst�me mat�riel qui ne
d�rivent pas d'une �nergie potentielle.
\exercice Une corde parfaitement souple glisse sans frottement sur une table horizontale car son extr�mit�
pend sur le rebord. Determiner son mouvement en supposant qu'elle est lach�e sans vitesse initiale. L'absence
de frottement et l'unique degr� de libert� conduisent � utiliser un th�or�me �nerg�tique.
%
%--------------------------------------------------PARTIE--------------------------------------------------
%
\partie{M�canique du solide.}\index{mecanique@M�canique!solide@du solide}
\spartie{Etude cin�matique.}
\sspartie{Mouvement d'un solide par rapport � un r�f�rentiel quelconque.}
En m�canique, un solide est syst�me mat�riel constitu� de points mat�riels immobiles les uns par rapport aux 
autres. Il d�finit donc un r�f�rentiel, appel� r�f�rentiel du solide.\par
\index{referentiel@R�f�rentiel!solide@du solide}
L'�tude cin�matique d'un solide se r�m�ne donc � celles de son r�f�rentiel associ� qu'on notera dor�n�vant~:
$(\mathcal{S})$. Le calcul des vitesses se fait gr�ce � la relation torsorielle~:
\encadre{$\ve{v_B}= \ve{v_A} + \ve{\omega}\land\ve{AB}$}
On note que $A$ n'est pas forc�ment un point mat�riel.
\sspartie{Mouvement relatif de deux solides en contact ponctuel.}
\point{D�finition~:}
Etant donn�s deux solides en contact ponctuel on note~: $I$ le point de contact, $(I_1)$ le point de $\mathcal{
S}_1$ co�ncidant avec $I$ � l'instant $t$ et $I_2$ le m�me dans $\mathcal{S}_2$. On note $\Pi$ le plan tangent
aux deux solides, qui existe toujours en pratique. On d�finit alors la vitesse de glissement de $(1)$ par
rapport � $(2)$ comme la vitesse de $I_1$ par rapport � $\mathcal{S}_2$. On a le th�or�me suivant~:
\encadre{$\ve{v_{I_1/\mathcal{S}_2}} = \ve{v_{I_1}} - \ve{v_{I_2}}$}
On d�finit des vitesses de pivotement et de roulement comme les composantes parall�les et orthogonales du 
vecteur rotation de $(1)$ par rapport � $(2)$.
\point{Condition cin�matique de non-glissement~:}
\index{Glissement}
On dit qu'il y a non-glissement lorsque la vitesse de glissement est nulle soit~:
\encadre{$\ve{v_{I_1}} = \ve{v_{I_2}}$}
Et cette condition peut �tre g�n�ralis�e dans des cas o\`u le contact n'est plus ponctuel. On montre ensuite
que~: $\ve{v_{12}} = -\ve{v_{21}}$ et que s'il n'y a pas glissement la distance parcourue par $I$ sur chaque
solide est la m�me.
\sspartie{Exemples.}
\point{Disque roulant sur une r�gle~:}
Un disque de rayon $R$ roule sur un r�gle horizontale rectiligne. D�terminer la condition cin�matique de non 
glissement.
\point{Disque roulant dans un cylindre en rotation~:}
Un disque de rayon $a$ roule � l'int�rieur d'un cylindre de rayon $b$. Le mouvement se fait dans le plan 
$(O,x,y)$. D�terminer les vecteurs rotation de chaque solide par rapport au r�f�rentiel d'observation, puis
les vitesses $\ve{v_{I_1}}$ et $\ve{v_{I_2}}$ ainsi que $\ve{v_I}$ (� d�finir). En d�duire la vitesse de 
glissement, de roulement, de pivotement.
\point{Disque roulant sans glisser sur une table~:}
Un disque de rayon $b$ roule sans glisser sur une table horizontale immobile dans le r�f�rentiel d'observation
rep�r�e par les vecteurs $\vx,\vy$. Il est retenu par un arbre horizontal qui lie son centre $C$ � un point 
$C'$ de l'axe $(O,\vz)$. On note $I$ le point de contact entre le disque et la table et $\theta$ l'angle 
$(\vx,\ve{OI})$ tel que $\dot{\theta} = cste$. D�terminer l'acc�l�ration dun point du disque qui co�ncide 
avec $J$ diam�tralement oppos� au point $I$.
\spartie{El�ments cin�tiques.}
\sspartie{Centre de masse.}
\index{Centre de masse}
\point{D�finition~:}
Le centre de masse d'un syst�me mat�riel est le barycentre des points $(A_i,\!m_i)$
\point{Propri�t�s~:}
Le centre de masse est situ� sur tout �l�ment de sym�trie (plan, droite, point). La quantit�~:$M\ve{OG}$ est 
une grandeur extensive (tr�s utile!!).

\point{Exemples~:}
D�terminer les centres de masses des solides suivants~:\par
demi-boule homog�ne, c�ne homog�ne, c�ne surmont� d'une demi-boule de m�me rayon.
\point{Th�or�me de Guldin~:}
\index{theoreme@Th�or�me!guldin@de Guldin}
On se donne une courbe homog�ne de longueur $L$ et un axe $\Delta$ ne coupant pas cette courbe. La rotation
de $2\pi$ de la courbe autour de l'axe engendre une surface de surface $S$ telle que si $r_G$ d�signe la 
distance du centre de masse de la courbe � l'axe~:
\encadre{$2\pi L r_G = S$}
On en d�duit le centre de masse d'un demi-cercle homog�ne en consid�rant $\Delta$ le diam�tre du demi-cercle
car on conna�t la surface $S$. De m�me en appliquant le th�or�me dans l'autre sens on en d�duit la surface 
d'un tore.
\point{Th�or�me de Guldin n�2~:}
Cette fois on consid�re une plaque homog�ne de surface $S$ et un axe $\Delta$. Si le volume engendr� par 
rotation de la plaque autour de l'axe est $V$ on a la relation~:
\encadre{$2\pi S r_G = V$}
On en d�duit le centre de masse d'un demi-disque ou le volume d'un tore.

\sspartie{Moment d'inertie.}
\index{Moment!inertie@d'inertie}
\point{D�finition~:}
On d�finit le moment d'inertie d'un solide par rapport � un ensemble de points comme la quantit�~:
\encadre{$I_A = \sum\limits_i m_ir_i^2$}
O\`u $r_i$ d�signe la distance du point $A_i$ � l'ensemble $A$. On prend toujours pour $A$ un point, 
une droite ou un plan.
\point{Relations entre moments d'inertie~:}
Dans un rep�re $(O, \vx, \vy, \vz)$ la relation de pythagore fournit les relations~:
\encadre{$I_O = I_{xOy}+I_{yOz}+I_{zOx}=\frac{1}{2}(I_{xx}+I_{yy}+I_{zz})$}
Le moment d'inertie est une grandeur extensive (tr�s important!).
\point{Th�or�me d'Huygens~:} 
\index{theoreme@Th�or�me!huygens@d'Huygens}
Etant donn� un syst�me mat�riel de masse $M$ et une droite $\Delta$ quelconque~:
\encadre{$I_{\Delta} = I_{\Delta_G}+Md^2$}
O\`u $\Delta_G$ est la droite parall�le � $\Delta$ qui passe par $G$ le centre de masse du syst�me mat�riel.
$d$ d�signe la distance de $G$ � $\Delta$.
\point{Exemples~:}
Calculer le moment d'inertie par rapport � un diam�tre d'un boule homog�ne, d'un disque homog�ne, d'une tige
homog�ne par rapport � son centre, par rapport � une extr�mit�, d'un c�ne homog�ne plein par rapport � son axe,
d'un c�ne surmont� d'une demi-boule de m�me rayon par rapport � l'axe du c�ne...
\sspartie{Moment cin�tique d'un solide.}
\index{Moment!cinetique@cin�tique d'un solide}
\point{Moment cin�tique en un point commun � $\mathcal{S}$ et $\mathcal{R}$~:}
Le moment cin�tique par rapport � ce point $O$ vaut~:\par
$\ve{L_O}=\sum\limits_i m_i \ve{OA_i}\land\ve{v_i} = \sum\limits_i m_i\ve{OA_i}\land(\ve{\omega}\land\ve{v_i})
         =\sum\limits_i m_i(OA_i^2\ve{\omega} -(\ve{OA_i}\cdot\ve{\omega})\ve{OA_i})$\par
Et ensuite un calcul de composantes sur une base orthonorm�e directe fait appara�tre des moments d'inertie 
ainsi que des termes nouveaux baptis�s ``produits d'inertie''~:\par
$I_{xx} = \sum\limits_i m_i(y_i^2+z_i^2), I_{xy} = -\sum\limits_i m_ix_iy_i$. On obtient au final une relation
matricielle~:
\index{Matrice!inertie@d'inertie}
\index{Produit d'inertie}
\encadre{$\ve{L_O} = I \ve{\omega}$}
O\`u $I$ est la matrice d'inertie remplie des moments d'inertie et des produits d'inertie. Cette matrice 
d'inertie est �videmment sym�trique.

\point{Axes principaux~:}
\index{Axe principal}
Un axe est principal lorsque c'est une vecteur propre de la matrice d'inertie. On voit que tout axe de 
sym�trie est principal et que tout axe orthogonal � un plan de sym�trie est principal lors du calcul du moment
cin�tique au point d'intersection de l'axe et du plan.
\point{Exemples~:}
Matrice d'inertie d'une boule homog�ne, d'un cylindre homog�ne.
\point{Moment cin�tique d'un solide en son centre de masse~:}
Etant donn� que le centre de masse est un point commun entre le r�f�rentiel du solide et le r�f�rentiel du
centre de masse, on peut appliquer les relations pr�c�dentes en faisant les substitutions qui s'imposent~:\par
$O\to G, \mathcal{R}\to\mathcal{R}^{\star}$. En effet le th�or�me de K\oe nig permet d'identifier les 
vecteurs $\ve{L_G}$ et $\ve{L_G}^{\star}$ et les vecteurs $\ve{\omega}$ et $\ve{\omega}^{\star}$ sont �gaux 
puisque le r�f�rentiel du centre de masse est en translation.
\point{Moment d'inertie d'un solide en rotation autour d'un axe fixe~:}
Dans tous les cas, on a la relation suivante~:
\encadre{$L_{\Delta} = I_{\Delta}\omega$}
et si de plus $\Delta$ est un axe principal alors on �galit� des vecteurs.
\sspartie{Energie cin�tique d'un solide.}
\index{Energie!cinetique@cin�tique}
\point{Expression g�n�rale~:}
Un petit calcul montre l'�galit�~:
\encadre{$E_c = \dfrac{1}{2}\mathcal{C}\cdot\mathcal{V}$}
O\`u $\mathcal{C}$ et $\mathcal{V}$ d�signent les torseurs cin�tiques et torseurs vitesses du solide. Dans le
r�f�rentiel du centre de masse on retrouve vite le th�or�me de K\oe nig.
\point{Expression dans les cas d'une rotation autour d'un axe fixe~:}
Si on note~: $\Delta =(O, \vz)$ l'axe de rotation, on trouve la relation~:
\encadre{$E_c = \dfrac{1}{2}I_{\Delta}\omega^2$}
\point{Exemples~:}
-Calculer le moment cin�tique d'une barre homog�ne glissant le long de deux parois orthogonales. Calculer 
ensuite son �nergie cin�tique par le th�or�me pr�c�dent puis par le th�or�pe de K\oe nig.\par
-Une boule homog�ne roule sans glisser dans une corni�re fixe assimil�e � un di�dre de demi-angle au sommet
$\alpha$. Calculer son �nergie cin�tique en fonction de $\alpha, M$ et de $v_G$. Que peut on en d�duire quant
� la d�finition d'un point mat�riel?\par
-Une plaque rectangulaire homog�ne est en rotation autour d'une de ses diagonales. Calculer son moment 
cin�tique au centre de la plaque, en d�duire son $E_c$.
\spartie{Etude dynamique.}
\sspartie{Interaction de contact entre deux solides.}
\point{Notations~:}
Le contact �tant ponctuel on note~:
\begin{itemize}
\item{}$\ve{v_{12}}$ la vitesse de glissement de $1/2$ qui est parall�le au plan tangent ou nulle.
\item{}$\ve{R_{12}}$ la force exerc�e par $1$ sur $2$.
\end{itemize}\par
\centerline{\includegraphics*[width=6cm,height=3.5cm]{images/coulomb.eps}}
\point{Lois de Coulomb~:}
\index{Loi!coulom@de Coulomb|textbf}
Les lois �nonc�es ici sont relatives au mouvement de glissement mais il existe des lois analogues pour les 
mouvements de roulement et de pivotement. Elles fournissent des informations sur $\ve{R_{12}}$. Elles sont 
dites empiriques au sens o\`u elles ne reposent par sur une d�monstration mais sur l'exp�rience qui montre
qu'elles sont valables avec une excellente pr�cision. Elles s'�noncent en deux parties, l'une porte sur
$\ve{R_{12, \para}}$ l'autre sur $\ve{R_{12, \bot}}$~:
\encadre{\begin{tabular}{l}
$\ve{R_{12,\bot}}$ est colin�aire et de m�me sens que $\ve{n_{12}}$.\\
Le contact entre les deux solides a lieu ssi $\ve{R_{12,\bot}}\neq 0$
\end{tabular}}
\encadre{\begin{tabular}{ll}
Glissement & \begin{tabular}{l}$\ve{R_{12,\para}}$ est colin�aire et de m�me sens que $\ve{v_{12,\para}}$\\
                               Sa norme vaut~: $R_{12,\para} = f R_{12,\bot}$\end{tabular}\vspace{1mm}\\
\hline
Non-glissement & \begin{tabular}{l}\raise-1.5mm\hbox{$\ve{R_{12,\para}}$ est de direction et sens inconnus.}\\
                               Sa norme v�rifie~: $R_{12,\para} < f R_{12,\bot}$\end{tabular}
\end{tabular}}
\vspace{2mm}
$f$ est appel� coefficient de frottement statique ou dynamique selon qu'il y a glissement ou non. Dans tous les
cas (glissement ou non) on dispose d'un nombre suffisant d'informations pour l'�tude m�canique du syst�me 
compos� des deux solides. Les cas limites $f$ tendant vers z�ro ou vers l'infini sont approch�s avec certaines
surfaces de contact mais en pratique $f\sim 1$ et le non-glissement n'a lieu que si $f$ est sup�rieur � une 
certaine valeur qui d�pend des conditions.
\index{Coefficient!frottement@de frottement}
\sspartie{Exemples.}
- Une brique poreuse est d�pos�e sans vitesse initiale sur un plan inclin�. D�terminer les conditions de 
non-basculement et de non glissement. Justifier l'hypoth�se de porosit� de la brique.\par
- Une boule de billard homog�ne, de rayon $a$ glisse sur une table horizontale. On note $\ve{v_{G_0}},
\ve{\omega}_0$ la vitesse du centre de masse et le vecteur rotation initiaux de la boule. D�terminer le 
mouvement du centre de masse (sans tenir compte des frottements de pivotement et de roulement).\par
- Un cylindre homog�ne, � base circulaire, de rayon $b$, roule sur un plateau horizontal qui est en mouvement
de translation horizontale selon la loi~: $\ve{AO} = X_0\sin(\omega_0 t)\vx$. O\`u $A$ d�signe un point du 
r�f�rentiel d'observation et $O$ un point du plateau. D�terminer la valeur minimale du coefficient de 
frottement pour que le roulement ait lieu sans glissement.
\spartie{Etude �nerg�tique.}
\sspartie{Puissance des forces subies par un solide.}
Un petit calcul du m�me acabit que celui de l'�nergie cin�tique montre que la puissance dans le r�f�rentiel
d'observation d'un ensemble quelconque de forces subies par un solide a pour puissance~:
\encadre{$P = \mathcal{F}\cdot\mathcal{V}$}
En notant $\mathcal{F}, \mathcal{V}$ les torseurs forces et vitesses du solide. Dans le cas particulier d'un
mouvement de translation �a donne $P = \ve{F}\cdot\ve{v}$ et pour une rotation autour d'un axe fixe~:
$P = \omega M_{\Delta}$.
\sspartie{Forces int�rieures.}
\index{Force!interieure@int�rieures}
\encadre{La puissance des forces int�rieures est nulle.}
\sspartie{Puissance des forces de contact ponctuel entre deux solides.}
Pour ces forces l� tout d�pend du syst�me choisi, selon qu'il contient l'un ou les deux solides~:
\point{Le syst�me comprend les deux solides~:}
Les forces de contact sont int�rieures et leur puissance a m�me valeur dans tout r�f�rentiel. Par exemple on
peut calculer dans le r�f�rentiel d'observation et on trouve~:
\encadre{$P_{12} = \ve{R_{12}}\cdot\ve{v_{12}} = - \ve{R_{12, \para}}\cdot\ve{v_{12}}$} 
On en d�duit que cette puissance est n�gative s'il y a simultan�ment glissement et frottement et nulle dans 
tous les autres cas. En pratique on prendra de pr�f�rence ce syst�me l�, cependant on peut faire le calcul avec
un autre syst�me~:
\point{Le syst�me �tudi� comprend uniquement un solide (1)~:}
On trouve $P_1 = \ve{R_{12}}\cdot\ve{v_{I_1}}\neq P_{12}$

\sspartie{Puissance des forces de liaison.}
\index{Force!liaison@de liaison}
On appelle liaison tout dispositif mat�riel qui permet de diminuer le nombre de degr�s de libert�. Dans la 
suite on s'int�ressera surtout � deux types de liaison~: pivot et rotule. La premi�re impose un rotation 
autour d'un axe, la seconde autour d'un point. Une liaison est dite parfaite lorsque la puissance des forces
int�rieures de liaison est nulle.\par
Dans le cas d'une liaison pivot parfaite, cette condition se traduit par l'�\-ga\-li\-t�~: $M_{\Delta} = 0$.
\encadre{\begin{tabular}{c}Si un solide est soumis � une liaison pivot parfaite,\\
le moment par rapport � l'axe des forces
de liaison est nul.\end{tabular}}
De m�me pour la liaison rotule.


\include{ondes/ondes}
\chapter{Optique}\index{Optique}
\partie{Diffraction :}\index{Diffraction}
\spartie{Introduction :}
\sspartie{D�finition :}
On appelle diffraction tout ph�nom�ne optique qui ne suit pas les lois de l'optique g�om�trique.
\sspartie{Mises en �vidence exp�rimentale :}
\point{Diffraction par une fente ou par un trou :}
Lorsqu'on envoie un faisceau cylindrique sur une fente, l'�clairement observ� ne suit pas les lois de l'optique 
g�om�trique. En effet celles-ci pr�voient un �clairement sous la forme d'une seule tache situ�e en face de la fente
alors qu'en r�alit� on observe plusieurs taches.\par
De m�me lorsqu'on envoie un faisceau cylindrique sur un bord mince (lame de rasoir gilette par exemple), on 
n'observe pas la discontinuit� de l'�clairement pr�dite par l'optique g�om�trique mais des taches caract�ristiques.
De mani�re g�n�rale, on observe un tel ph�nom�ne d�s que les lois de l'optique g�om�trique pr�disent un 
discontinuit� de l'�clairement.
\sspartie{Interpr�tation th�orique :}
Les lois de l'optique g�om�trique ne sont valables que dans le cadre de l'approximation faite dans le cours sur
les ondes �lectromagn�tiques. Or, si elles pr�voient une discontinuit� du champ, c'est que celui-ci varie 
brutalement � l'�chelle de la longueur d'onde. C'est une contradiction. Pour la r�soudre, il faudrait r�soudre
les �quations de Maxwell, ce qui n'est fait que dans quelques cas particuliers. Il faut donc introduire un nouveau 
principe.
\spartie{Principe d'Huygens-Fresnel :}
\index{Principe!huygens@d'Huygens-Fresnel}
\sspartie{Enonc�~:}
\centerline{\begin{pspicture}(12,2.5)
\psdots[dotstyle=+](1,1.5)
\rput(1,1.8){A}
\rput(2,0){demi espace source}
\rput(1,0.5){\small{source primaire}}
\rput(4,1.5){{\Huge{(}}\begin{tabular}{c}Syst�me\\ 
                                optique\end{tabular}{\Huge{)}}}
\psline[linestyle=dashed,linewidth=0.1mm](6,1)(6,2.5)
\rput(6,0.5){\small{surface diffractante}}
\psline[linewidth=0.5mm](6,1.2)(6,1.6)
\rput(6.2,1.3){$\scriptstyle{\dd{S}}$}
\rput(5.8,1.2){\tiny{M}}
\rput(8,1.5){{\Huge{(}}\begin{tabular}{c}Syst�me\\ 
                                optique\end{tabular}{\Huge{)}}}
\psdots[dotstyle=+](11,1.5)
\rput(11,1.8){P}
\rput(11,0.5){\small{point d'observation}}
\rput(9,0){demi espace d'observation}
\end{pspicture}}\par
\vspace{0.5cm}
\index{montage@Montage de Frauh�fer}
Dans ce montage g�n�ral, la surface diffractante est une fente, un r�seau, un trou... on suppose que la lumi�re 
�mise par les sources primaires est coh�rente et monochromatique. \par
Le principe d'Huygens-Fresnel stipule que chaque �l�ment de surface $(M,\!\dd{S})$ de la surface diffractante se 
comporte comme une source ponctuelle qui �met une onde proportionnelle � $\dd{S}$ ainsi qu'� l'onde incidente au 
point $M$ et la constante de proportionnalit� ne d�pend pas de $M$ ni du temps.\par
Dans chaque demi-espace hors de la surface de diffraction, on peut appliquer les lois de l'optique g�om�trique.\par
Il peut arriver que les deux demis espaces soient confondus (comme dans un r�seau � r�flexion).
\sspartie{Formulation math�matique :}
L'�l�ment de surface $\dd{S}$ se comporte comme une source ponctuelle �mettant une onde :\\
$C s(M,t)\dd{S}$. Dans le cadre de l'approximation scalaire de l'optique physique (cours d'�lectromagn�tisme) on 
sait alors que l'onde re�ue au point $P$ a pour expression :\\
\centerline{$a_{MP}e^{i\frac{2\pi}{\lambda}(MP)}Cs(M,t)\dd{S}$}

O\`u $(MP) = \integrale{L}{{}}{n\dd{L}}$ d�signe le chemin optique de $M$ � $P$.\\
De plus dans les montages usuels les coefficients $a_{MP}$ ont quasiment tous la m�me valeur. Il reste donc � 
int�grer sur la surface $S$ en regroupant toutes les constantes dans la grande constante de proportionnalit� :\par
\encadre{$s(P,t) \propto \integrale{S}{}{s(M,t)e^{i\frac{2\pi}{\lambda}(MP)}\dd{S}}$}\par
\spartie{Diffraction � l'infini :}
\sspartie{Montage exp�rimental :}
\centerline{\begin{pspicture}(7.2,7.2)
\psset{xunit=0.6cm,yunit=0.6cm}
%grande ellipse
\psellipse(6,6)(2,4)  
%point C
\psdots[dotstyle=*](6,6)
\rput(6,5.6){\tiny{C}}
%plan diffractant
\psline(1.5,0)(1.5,10.5)
\psline(1.5,10.5)(3.8,12)
%ecran
\psline(10.5,0)(10.5,10.5)
\psline(10.5,10.5)(12.8,12)
%axe optique
\psline[linewidth=0.1mm]{->}(-0.5,6)(1,6)
\psline[linewidth=0.1mm](0,6)(1.5,6)
\psline[linewidth=0.1mm,linestyle=dotted](1.5,6)(2.8,6)
\psline[linewidth=0.1mm](2.8,6)(4,6)
\psline[linewidth=0.1mm,linestyle=dotted](4,6)(6,6)
\psline[linewidth=0.1mm](6,6)(10.5,6)
\psline[linewidth=0.1mm,linestyle=dotted](10.5,6)(11.8,6)
\psline[linewidth=0.1mm]{->}(11.8,6)(13.5,6)
%rayon passant par O
\psline[linewidth=0.1mm]{->}(2.8,6)(3.75,6.32)
\psline[linewidth=0.1mm](2.8,6)(4.035,6.416)
\psline[linewidth=0.1mm,linestyle=dotted](2.8,6)(5.65,6.96)
\psline[linewidth=0.1mm,linestyle=dotted](5.65,6.96)(12,8)
\psline[linewidth=0.1mm]{->}(5.65,6.96)(8.825,7.48)
\psline[linewidth=0.1mm](5.65,6.96)(10.5,7.749)
%point O
\psdots[dotstyle=*](2.8,6)
\rput(2.8,5.6){\tiny{O}}
%point F'
\psdots[dotstyle=*](11.8,6)
\rput(11.8,5.6){\tiny{F'}}
%rayon parall�le
\psline[linewidth=0.1mm]{->}(0,7)(1,7)
\psline[linewidth=0.1mm](1,7)(1.5,7)
\psline[linewidth=0.1mm,linestyle=dotted](1.5,7)(3.55,7)
\psline[linewidth=0.1mm,linestyle=dotted](3.55,7)(6.4,7.96)
\psline[linewidth=0.1mm](3.55,7)(4.12,7.192)
\psline[linewidth=0.1mm,linestyle=dotted](6.4,7.96)(12,8)
\psline[linewidth=0.1mm]{->}(6.4,7.96)(9.2,7.98)
\psline[linewidth=0.1mm](6.4,7.96)(10.5,7.984)
%Point I
\psdots[dotstyle=*](6.4,7.96)
\rput(6.4,8.26){\tiny{I}}
%point M
\psdots[dotstyle=*](3.55,7)
\rput(3.3,7.4){\tiny{M(x,y)}}
%rayon CP
\psline[linewidth=0.15mm,linestyle=dashed](6,6)(12,8)
%segment OM
\psline[linewidth=0.1mm,linestyle=dotted](2.8,6)(3.55,7)
%segment I?
\psline[linewidth=0.1mm,linestyle=dotted](5.65,6.96)(6.4,7.96)
%point P
\psdots[dotstyle=*](12,8)
\rput(12.2,8.4){\tiny{P(X,Y)}}
%point H et segment HM
\psline[linewidth=0.1mm,linestyle=dotted](3.55,7)(3.45,6.25)
\psdots[dotstyle=*](3.45,6.25)
\rput(3.7,6.1){\tiny{H}}
%vecteurs u et v
\psline[linewidth=0.2mm]{->}(6,6)(6.95,6.32)
\psline[linewidth=0.2mm,linestyle=dotted](6.95,6.32)(6.49,6.32)
\psline[linewidth=0.1mm]{->}(6,6)(6.49,6.32)
\rput(6.9,6.6){$\scriptstyle{\ve{u}}$}
\rput(6.3,6.6){$\scriptstyle{\ve{v}}$}
%textes
\rput(12.5,0.5){\begin{tabular}{c} 
               \tiny{Ecran dans le plan}\\
               \tiny{focal image}
            \end{tabular}}
\rput(6,.5){\begin{tabular}{c}
               \tiny{Lentille mince}\\
               \tiny{Convergente}
           \end{tabular}}
\rput(3,0.5){\begin{tabular}{c}
               \tiny{Surface}\\
               \tiny{diffractante}\\
               \tiny{plane}
           \end{tabular}}
\rput(-0.5,0.5){\begin{tabular}{c}
               \tiny{Faisceau cylindrique}\\
               \tiny{en incidence normale}
           \end{tabular}}
\end{pspicture}}
\par\vspace{0.5cm}
La diffraction � l'infini est caract�ris�e par le montage de Frauh�fer dans lequel la lentille est utilis�e dans
les conditions de Gauss. D'un point de vue notations, $\ve{v}$ d�signe la projection de $\ve{u}$ dans le plan de la
lentille. On note ses coordonn�es $(\alpha, \beta)$, celles de $M$ $(x, y)$ et celles de $P$ $(X, Y)$. 
On montre g�om�triquement que $\ve{v} = \frac{\ve{F'P}}{f'}$. Ce qui se traduit par l'�galit� : $X = \alpha f'$ et
$Y = \beta f'$.
\sspartie{D�termination de l'onde diffract�e :}
Elle peut �tre d�termin�e en connaissant l'onde incidente et le chemin optique $(MP)$.
\point{Onde incidente, Transmittance :}
\index{Transmittance}
On distingue plusieurs cas selon la nature de l'objet diffractant.
\begin{itemize}
\item{Cas d'un simple trou :}\par
\centerline{\begin{pspicture}(3,3)
\psset{xunit=0.5cm,yunit=0.5cm}
\psline[linewidth=0.1mm]{->}(0,1)(1,1)
\psline[linewidth=0.1mm](0,1)(2,1)
\psline[linewidth=0.1mm]{->}(0,2)(1,2)
\psline[linewidth=0.1mm](0,2)(2,2)
\psline[linewidth=0.1mm]{->}(0,4)(1,4)
\psline[linewidth=0.1mm](0,4)(2,4)
\psline[linewidth=0.1mm]{->}(0,5)(1,5)
\psline[linewidth=0.1mm](0,5)(2,5)
\psline[linewidth=0.1mm]{->}(0,3.5)(1,3.5)
\psline[linewidth=0.1mm](0,3.5)(2,3.5)
\psline{-|}(2,0)(2,2.05)
\psline{|-}(2,3.95)(2,6)
\psdots[dotstyle=*](2,3.5)
\rput(2.5,3.5){\small{M}}
\rput(2.4,5){\small{S}}
\end{pspicture}}
L'onde incidente en incidence normale a m�me valeur � un instant donn� en tout point de $S$, qui est une surface
d'onde. Elle s'�crit donc :\par
$$s(M,t) = s_0 e^{-i\omega t}$$
\end{itemize}

\begin{itemize}
\item{Cas d'une pellicule de transparence variable :}
La surface $S$ est une plaque inhomog�ne d'indice $n$ d'�paisseur $e$, accol�e � un trou comme ci-dessous.\par
\centerline{\begin{pspicture}(3,3)
\psset{xunit=0.5cm,yunit=0.5cm}
\psline[linewidth=0.1mm]{->}(0,1)(1,1)
\psline[linewidth=0.1mm](0,1)(3,1)
\psline[linewidth=0.1mm]{->}(0,2)(1,2)
\psline[linewidth=0.1mm](0,2)(3,2)
\psline[linewidth=0.1mm]{->}(0,4)(1,4)
\psline[linewidth=0.1mm](0,4)(3,4)
\psline[linewidth=0.1mm]{->}(0,5)(1,5)
\psline[linewidth=0.1mm](0,5)(3,5)
\psline[linewidth=0.1mm]{->}(0,3.5)(1,3.5)
\psline[linewidth=0.1mm](0,3.5)(3,3.5)
\psline[linewidth=0.01mm](2,0)(2,6)
\psline[linewidth=0.01mm](3,0)(3,6)
\psline{-|}(3,0)(3,2.05)
\psline{|-}(3,3.95)(3,6)
\psdots[dotstyle=*](3,3.5)
\psdots[dotstyle=*](2,3.5)
\rput(3.5,3.5){\small{M}}
\rput(1.6,3.1){\small{M'}}
\rput(2.5,0.5){\textcircled{n}}
\end{pspicture}}\par
$\dfrac{s(M,t)}{s(M',t)} = a_{M'M}e^{i\frac{2\pi}{\lambda}(M'M)}$. Or de m�me que ci-dessus : $s(M',t) = 
s_0'e^{-i\omega t}$ et de plus : $(M'M) = e n(M)$. Si on pose $\tau(M) = a_{M'M}$ qui est un nombre compris entre 
$0$ et $1$ appel� transparence, et $s_0 = s_0' e^{i\frac{2\pi}{\lambda}(M'M)}$ on peut simplifier le r�sultat :\par
$$s(M,t) = s_0\tau e^{-i\omega t}$$\par
\item{Cas d'une lame transparente d'�paisseur variable :}
Cette fois la plaque est encoch�e au niveau du trou et l'encoche, situ�e du cot� ext�rieur, est d'�paisseur $a$. 
L'indice du milieu constituant la plaque est $n$ suppos� uniforme. Tout se fait comme ci-dessus si ce n'est :\par
$a_{M'M} = \frac{2}{1+n}\cdot 1 \cdot{\frac{2 n}{n+1}} = \frac{4n}{(n+1)^2}=T$ (cf : pb sur le transmission par des 
dioptres). Le chemin optique d�pend de la position du point $M$ par rapport � l'encoche. Selon les cas il vaut 
$n(e-a)+a$ ou $ne$. Si on pose : $s_0 = s_0' T e^{i\frac{2\pi}{\lambda}ne}$ alors : on retrouve le r�sultat 
habituel~:\par
$$s(M,t) = s_0\tau(M)e^{i\omega t}$$\par
En posant $\tau = e^{i\frac{2\pi}{\lambda}(1-n)a}$ dans l'encoche et $1$ ailleurs. 
\end{itemize}
Ce r�sultat est g�n�ralisable. L'onde arrivant en $M$ peut �tre mise sous cette forme dans laquelle $\tau$ est 
appel� transmittance.\par
\point{Chemin optique :}
\index{Chemin optique}
Le chemin optique $(MP)$ peut �tre calcul� dans le cadre de l'optique \\
g�om�trique en prenant le rayon issu de $O$ 
pour r�f�rence : $(MP) = (HP) = (OP) - \ve{u}\cdot\ve{OM} = (OP) - \ve{v}\cdot\ve{OM}$.
\point{Onde diffract�e :}
Un simple calcul montre que :\par
$s(P,t) = s_0e^{-i\omega t}e^{i\frac{2\pi}{\lambda}(OP)}\integrale{S}{}{\tau(M)e^{-i\frac{2\pi}{\lambda}\ve{v}\cdot
\ve{OM}}\dd{S}}$. On retiendra que :\par
L'onde diffract�e dans la direction $\ve{u}$ et arrivant au point $P$ tel que $\ve{F' P} = f'\ve{v}$ est 
proportionnelle � : $$\integrale{S}{}{\tau(M)e^{-i\frac{2\pi}{\lambda}\ve{v}\cdot\ve{OM}}\dd{S}}$$\par
\rmq L'int�grale ci dessus n'est autre que la transform�e de Fourier de la transmittance.
\sspartie{Figure de diffraction :}
\index{Figure de diffraction}
On appelle ``�clairement'' la valeur moyenne temporelle de la puissance\\
�lectromagn�tique re�ue par la surface de 
l'�cran et figure de diffraction la r�partition de l'�clairement.
L'�nergie lumineuse diffract�e en $M$ et re�ue en $P$ a une valeur moyenne temporelle proportionnelle � 
$|s(P,t)|^2$. En cons�quence l'intensit� diffracte dans la direction $\ve{u}$ ou l'�clairement en $P$ sont 
proportionnels � $\left\lvert\integrale{S}{}{\tau(M)e^{-i\frac{2\pi}{\lambda}\ve{v}\cdot\ve{OM}}\dd{S}}
\right\rvert^2$ ainsi qu'� l'intensit� $I_0$ de l'onde incidente.\par
\sspartie{Propri�t�s de la figure de diffraction � l'infini :}
\begin{itemize}
\item{}
La figure de diffraction n'est pas modifi�e lorsque l'on d�place la surface diffractante. Cette propri�t� montre 
qu'on peut accoler lentille et surface diffractante.
\end{itemize}

\begin{itemize}
\item{}
Une rotation de la surface diffractante autour de l'axe optique de la lentille engendre une rotation identique la 
figure de diffraction.\par
\item{}
La figure de diffraction par un trou en incidence quelconque se d�duit de celle en incidence normale par une 
translation de vecteur $\ve{F'A'}$, $A'$ �tant l'image g�om�trique du faisceau incident en l'abscence de 
diffraction.\par
\item{}
Pour une surface diffractante dont la transmittance est invariante par translation le long de l'axe $(Oy)$, la 
lumi�re diffract�e est concentr�e sur l'axe $(F'X)$.
\end{itemize}
\spartie{Exemples de diffration � l'infini :}
\sspartie{Diffraction par un trou carr� :}
On suppose le trou carr�, de c�t�s $a$ et $b$. D'apr�s les propri�t�s ci-dessus, on peut : confondre le centre du 
rectangle avec l'axe optique, prendre les axes parall�les aux c�t�s du rectangle, supposer le faisceau d'incidence 
normale. Le calcul de l'int�grale donnant $s(P,t)$ � une constante multiplicative pr�s se fait alors sans probl�me 
et on trouve que :\par
\encadre{$\mathcal{E}(P,t) =\mathcal{E}_0\sinc^2(u)\sinc^2(v)$}\par
en posant $u = \dfrac{\pi X a}{\lambda f'}$ et $v =\dfrac{\pi Y b}{\lambda f'}$. Le calcul fait appara�tre que
l'�clairement est proportionnel au carr� de la surface diffractante.\par
\centerline{\includegraphics*[width=8cm,height=6cm]{images/eclairement.eps}}
On remarque que si a et b tendent vers l'infini, l'�clairement est concentr� au point $F'$, ce qui est concordant 
avec les pr�visions de l'optique g�om�trique. Si b tend vers l'infini ($b \gg a$) la lumi�re est concentr�e sur 
l'axe $(F'X)$. Ce r�sultat donne la figure de diffraction par une fente mince, ce qui peut �tre retrouv� en 
utilisant la quatri�me propri�t� du 1.3.4
\sspartie{Apodisation :}
\index{Apodisation}
C'est une technique visant � supprimer les pieds de la figure de diffraction cr�ee par une fente mince, afin 
d'am�liorer la r�solution d'instruments optique. 
Elle peut �tre r�alis�e en choisissant une transmittance idoine dans la fente. Si 
par exemple on prend $\tau(x) = \cos (\pi\frac{x}{a})$, le calcul de l'int�grale donne l'onde diffract�e dans la 
direction $\alpha = X/f'$ (� une constante multiplicative pr�s), ce qui fournit l'�clairement au point $P$ :\par
\encadre{$\dfrac{\mathcal{E'}}{\mathcal{E'}_0} = \left(\dfrac{\cos u}{1 - \frac{4}{\pi^2}u^2}\right)^2$}\par
O\`u $u = \pi\frac{a}{\lambda}\alpha$. 

D'un point de vue graphique :\par
\centerline{\includegraphics*[width=6cm,height=4cm]{images/apodisation.eps}}\par
On note que les pieds ont quasi disparu et qu'on garde la lumi�re localis�e au voisinage de $F'$.
\sspartie{Fentes d'Young :}
\index{Fentes d'Young}
Il s'agit de deux fentes minces de m�me largeur $a$ �cart�es de $a'$. Toujours en utilisant la propi�t� (4), on 
sait que l'onde diffract�e dans la direction $\alpha$ est proportionnelle � l'int�grale si bien connue. Celle ci 
se calcule sans probl�me et apr�s factorisation, on trouve $2a\sinc u \cos u'$ en posant 
$u = \pi\frac{a}{\lambda}\alpha$ et 
$u' = \pi\frac{a'}{\lambda}\alpha$ :
\encadre{$2 a \sinc u \cos u'$}\par
Ce terme est le produit de deux termes, dont l'un (le sinus cardinal) correspond � l'�clairement d'une seule fente 
et l'autre � un ph�nom�ne d'interf�rence.
\sspartie{Diffraction par un trou circulaire~:}
La surface diffractante �tant invariante par rotation, il en est de m�me de la figure de diffraction et il suffit de 
calculer l'onde diffract�e en un point de l'axe $(F'X)$. on sait que celle-ci est proportionnelle � l'int�grale :\par
$\integrale{S}{}{e^{-i\frac{2\pi}{\lambda}\ve{v}\cdot\ve{OM}}\dd{S}}$. Or on se place en coordonn�es polaires, pour 
lesquelles $\ve{v}\cdot\ve{OM} = \rho r\cos{\theta}$. On en d�duit que l'onde diffract�e est proportionnelle � 
l'int�grale double~:\par
$\integrale{0}{2\pi}{\left\lbrack\integrale{0}{a}{e^{-i2\pi\frac{r\rho}{\lambda f'}\cos\theta}\rho\dd{\rho}}\right
\rbrack\dd{\theta}}$. On ``reconnait'' la fonction de Bessel. Puis l'�clairement est 
proportionnel au carr� de cette fonction de Bessel, ce qui donne un �clairement relatif de cette forme~:\par
\centerline{\includegraphics*[width=8cm,height=6cm]{images/circulaire.eps}}\par
On constate que la figure de diffraction pr�sente une tache centrale, de rayon $0.61\lambda f'/a$.
\sspartie{Diffraction par $N$ surfaces diffractantes, d�duites les unes des autres par translations :}
\point{Cas g�n�ral :}
On se donne une surface diffractante r�union de $N$ surfaces toutes identiques � la surface $S_1$, centr�es chacune
sur un point $M_n$ telle que la surface $S_n$ est d�duite de $S_1$ par une translation de vecteur $\ve{M_1M_n}$.

On peut choisir $M_1$ sur l'axe optique de la lentille. L'onde diffract�e selon la direction $\ve{u}$ est alors
proportionnelle � l'int�grale :\par
$\integrale{S}{}{\tau(M)e^{-i\frac{2\pi}{\lambda}\ve{v}\cdot\ve{OM}}\dd{S}} = \serie{n = 0}{N}{\integrale{S_n}{}{%
\tau(M)e^{-i\frac{2\pi}{\lambda}\ve{v}\cdot\ve{OM}}\dd{S}}}$.\\
Or $\ve{OM} = \ve{OM_n}+ \ve{M_nM}$ et en rempla�ant dans l'int�grale :\par
$\serie{n=0}{N}{e^{-i\frac{2\pi}{\lambda}\ve{v}\cdot\ve{OM_n}}\integrale{S_n}{}{\tau(M)e^{-i\frac{2\pi}{\lambda}%
\ve{v}\cdot\ve{M_nM}}\dd{S}}}$. Comme l'int�grale ci-dessus ne d�pend pas de l'indice de sommation on peut la sortir
de la somme et on trouve un produit de deux termes. L'�clairement v�rifie donc :\par
$$\mathcal{E}(P) = \mathcal{E}_1(P)\left\lvert\serie{n=1}{N}{e^{-i\frac{2\pi}{\lambda f'}\ve{F'P}\cdot\ve{OM_n}}}%
\right\rvert^2$$\\
Le premier terme �tant l'�clairement obtenu la premi�re surface seule, l'autre terme correspond � un terme 
d'interf�rences.
\point{Trous d'Young :}
\index{Trous d'Young}
On consid�re deux trous circulaires, de m�me rayon $R$ et espac�s de $a$. Un peu de trigonom�trie �l�mentaire permet
de simplifier le terme d'interf�rences et on trouve :\par
$$\mathcal{E} = 2\mathcal{E}_1 (1+\cos\frac{2\pi X a}{\lambda f'})$$
On trouve la figure de diffraction d'un trou seul mais parcouru de raies d'interf�rences, comme le montre ce graphe :
\par
\centerline{\includegraphics*[width=8cm,height=6cm]{images/young.eps}}\par
\sspartie{Cas de $N$ surfaces diffractantes r�parties r�guli�rement, notion de r�seau :}
On consid�re une surface diffractante r�partie r�guli�rement dans une direction donn�e. C'est-�-dire telle que :
$\ve{M_1M_n} = (n-1)a\vx$ et qu'en plus $N$ soit tr�s grand devant $1$. Alors la somme du terme d'interf�rences
appara�t comme une somme g�om�trique qui se calcule ais�ment. On trouve :\par
$I(\alpha, \beta) = I_1(\alpha, \beta)\left(\dfrac{\sin(Nu)}{\sin u}\right)^2$. En posant 
$u = \frac{\pi\alpha a}{\lambda}$. Lorsqu'on suppose $N$ tr�s grand devant $1$ on obtient des pics d'�clairement, 
r�partis sur tous les multiples entiers de $\pi$. On en d�duit que la figure de diffraction par un r�seau est 
concentr�e dans les directions : $\alpha_p = p\frac{\lambda}{a}$.\par
Cette formule est caract�ristique de l'incidence normale et dans les conditions de Gauss. En affinant les calculs
(fait en exercice), on d�montre la formule du r�seau :\par
\encadre{$\sin i_p - \sin i = p\dfrac{\lambda}{a}$}\par
Comme ces angles de diffraction d�pendent de $\lambda$, le r�seau peut-�tre utilis� en spectrophotom�trie.

\sspartie{Cas de surfaces r�parties al�atoirement :}
On reprend l'expression du terme d'interf�rences~:\par
$\begin{array}{ccc}
\left\lvert\serie{n = 1}{N}{e^{-i\frac{2\pi}{\lambda f'}\ve{F'P}\cdot\ve{OM_n}}}\right\rvert^2
 & = & \serie{1\leq n,m\leq N}{}{e^{-i\frac{2\pi}{\lambda f'}(\ve{F'P}\cdot\ve{OM_n} - \ve{F'P}\cdot\ve{OM_m})}}\\
\serie{1\leq n, m\leq N}{}{e^{-i\frac{2\pi}{\lambda f'}\ve{F'P}\cdot\ve{M_mM_n}}}
  & = & N + \serie{1\leq n\neq m\leq N}{}{e^{-i\frac{2\pi}{\lambda f'}\ve{F'P}\cdot\ve{M_mM_n}}}
\end{array}$\par
$$N + \sum_{1\leq n\neq m\leq N}e^{i\theta_{mn}}$$\par
En posant $\theta_{mn} = \frac{2\pi}{\lambda f'}N + \ve{F'P}\cdot\ve{M_nM_m}$.\par
Le caract�re al�atoire de la distribution impose alors que la somme est n�gligeable devant $N$ sauf en $F'$ et son 
voisinage o\`u elle vaut $N^2$. On obtient donc un �clairement de la forme :\par
\encadre{$\left\lbrace\begin{array}{lcll}
\mathcal{E}(P) & = & N\mathcal{E}_1(P) & \mathrm{si}\;\;P\neq F'\\
\mathcal{E}(F') & = & N^2\mathcal{E}_1(F') & \\
\end{array}\right.$}

\partie{Interf�rences :}\index{interferences@Interf�rences}
\spartie{D�finition et r�alisation :}
\sspartie{D�finition :}
\index{Onde!coherentes@coh�rentes}
\index{Onde!incoherentes@incoh�rentes}
On dit que deux ondes interf�rent en un point lorsque leur superposition en ce point y donne un �clairement diff�rent
de la somme des �clairements obtenus avec chaque onde s�par�ment. Les deux ondes sont dites alors coh�rentes.\par
Dans le cas contraire, les deux ondes n'interf�rent pas, on dit qu'elles sont incoh�rentes.\par
Par exemple dans l'exemple ci-dessus, lorsque les $N$ surfaces sont r�parties al�atoirement, elles fournissent $N$
ondes incoh�rentes.
\sspartie{R�alisation :}
\noindent D'un point de vue exp�rimental, il existe deux sortes de montages interf�rentiels.
\point{Diviseur d'onde :}
Les rayons �mergents qui interf�rent sont issus de deux rayons distincts. Dans ce type de montage, les interf�rences
sont visibles dans un grand domaine spatial, mais seulement si la source est de faibles dimensions.
\point{Diviseur d'amplitude :}
Cette fois, ils sont issus du m�me rayon incident. Celui-ci subit des reflexions transmissions qui le scindent avant
d'arriver au point $P$.\par
Par exemple une lame � face parall�les donne cet effet : le rayon incident peut �tre transmis directement ou subir
deux reflexions avant de ressortir. Une lentille permettant de r�unir les deux au point d'observation.
\spartie{Source ponctuelle monochromatique :}\index{Source!ponctuelle!monochromatique}
On suppose que la source est ponctuelle et monochromatique. On se limite � la superposition de deux ondes seulement
et � l'approximation scalaire de l'optique physique.\\
$s(P,t) = s_1(P,t)+s_2(P,t) = s_0e^{-i\omega t}a_1e^{i\frac{2\pi}{\lambda}(SP)_1} + %
                              s_0e^{-i\omega t}a_2e^{i\frac{2\pi}{\lambda}(SP)_2}$\par
Soit :\\
$s(P,t) = s_0e^{-i\omega t}e^{i\frac{2\pi}{\lambda}(SP)_1}(a_1 + a_2e^{i\phim})$\par
En posant \fbox{$\delta(P) = (SP)_2 - (SP)_1$} la diff�rence de marche entre les deux rayons et 
\fbox{$\phim = 2\pi\frac{\delta}{\lambda}$} le d�phasage entre les deux rayons. $\phim$ d�pend de $P$. Par contre, 
on supposera par la suite que $a_1$ et $a_2$ n'en d�pendent pas. 

On en d�duit l'�clairement :\par
\vspace{0.5mm}
\encadre{$\mathcal{E}(P) = |s_0|^2\bigl(|a_1|^2+|a_2|^2 + 2|a_1||a_2|\cos(\phim-\phim_{12})\bigr)$}\par
En posant $\phim_{12} = \arg(a_1 a_2^{\star})$.\\
Cette expression met en �vidence l'existence d'interf�rences.\par
On appelle : \begin{itemize}
\item{} Figure d'interf�rences : La r�partition de l'�clairement.
\item{} Franges d'interf�rences : Les courbes $\phim = cte$.\index{Franges}
\item{} Franges brillantes : Les courbes $\mathcal{E} = \mathcal{E}_{max}$.
\item{} Franges sombres : Les courbes $\mathcal{E} = \mathcal{E}_{min}$.
\item{} Visibilit� :
\end{itemize}\index{visibilite@Visibilit�}\par
\encadre{$V = \dfrac{\mathcal{E}_{max} - \mathcal{E}_{min}}{\mathcal{E}_{max}+\mathcal{E}_{min}}$}\par
Dans le cas de ci-dessus, si on pose $\mathcal{E}_i = |s_0a_i|^2$, l'�clairement s'�crit :\par
$\eee = \eee_1+\eee_2+2\sqrt{\eee_1\eee_2}\cos(\phim-\phim_{12})$. Ce qui fournit la visibilit� :\par
$$V = \dfrac{2\sqrt{\rho}}{1+\rho}$$\par
En posant $\rho = \frac{\eee_1}{\eee_2}$. On constate que $V$ est maximale lorsque $\eee_1 = \eee_2$. C'est le cas,
en particulier, lorsque $a_1 = \pm a_2$. Dans ce cas $\eee = 2\eee_1(1\pm\cos\phim)$. $V$ est quasi-nulle si les deux
�clairements sont tr�s diff�rents.
\spartie{Source ponctuelle non-monochromatique :}\index{Source!ponctuelle!non-monochromatique}
\sspartie{Densit� spectrale de la source :}
\index{densite@Densit�!spectrale}
$I(\lambda)$ appel� ``densit� spectrale de la source'', caract�ristique une source non-mo\-no\-chro\-ma\-ti\-que telle que 
$I(\lambda)\dd{\lambda}$ d�signe l'intensit� �mise dans l'intervalle $\segment{\lambda}{\lambda+\dd{\lambda}}$.\par
\point{Finesse~:}
Pour une raie, c'est-�-dire une r�partition d'intensit� au voisinage d'un valeur $\lambda_0$, on d�finit la finesse
de la raie comme $F = \frac{\lambda_0}{\Delta\lambda}$. Ainsi pour la raie intense du mercure,
$F\sim 10^{-3}$. pour un \textsc{Laser}, $F\sim 10^{-6}$. Pour une raie monochromatique, la finesse est infinie, ce 
qui n'existe pas en r�alit�.
\sspartie{Expression int�grale de l'�clairement :}
On se limite aux montages interf�rentiels tels que $a_1 = a_2$. Dans ce cas, pour une source monochromatique, 
$\mathcal{E} = \mathcal{I}(1+\cos\phim)$. Dans le cas g�n�ral, on admettra que la source peut �tre consid�r�e comme 
la superposition de sources quasi-monochromatiques de $\lambda$ et $\delta I = I(\lambda)\dd{\lambda}$ qui �mettent
des ondes incoh�rentes. Alors :\par
\encadre{$\mathcal{E} = \integrale{0}{+\infty}{\bigl(1+\cos(2\pi\dfrac{\delta}{\lambda})\bigr)\mathcal{I}(\lambda)\dd{\lambda}}%
         $}\par
\rmq En utilisant la relation $\lambda = 2\pi\frac{c}{\omega}$, on peut aussi d�finir la densit� spectrale � partir
de la pulsation $\omega$. On trouve alors :\par
\encadre{$\mathcal{E} = \integrale{0}{+\infty}{\bigl(1+\cos(\delta\dfrac{\omega}{c})\bigr)\mathcal{I}(\omega)\dd{\omega}}%
         $}\par
\sspartie{Exemple : le doublet du sodium :}
\index{Doublet du sodium}
La vapeur de sodium �met dans le jaune des raies $\lambda_1$, $\lambda_2$ voisines de $\lambda_0 = 589.3$ nm : 
$\lambda_1 = \lambda_0 - \frac{\Delta\lambda}{2}$, $\lambda_1 = \lambda_0 + \frac{\Delta\lambda}{2}$. La quantit�
$\Delta\lambda = \lambda_2-\lambda_1$ �tant la largeur du doublet. D'un point de vue num�rique : $\Delta\lambda = 
0.6$ nm. On en d�duit la finesse : $F = 10^{3}$. Si on suppose la source ponctuelle, et 
$\mathcal{I}_1 = \mathcal{I}_2$ alors l'�clairement s'�crit :\par
$\eee = \mathcal{I}\left(2 + \cos(2\pi\frac{\delta}{\lambda_1})+\cos(2\pi\frac{\delta}{\lambda_2})\right)$.\\
En factorisant la somme de cosinus et en d�veloppant les termes obtenus � l'ordre $1$ en 
$\frac{\Delta\lambda}{\lambda_0}$ : on obtient l'�clairement sous la forme simplifi�e :\par
$$\eee = \eee_0\left(1+\cos(\pi\frac{\delta}{L})\cos(2\pi\frac{\delta}{\lambda_0})\right)$$\par
En posant $L = \dfrac{\lambda_0^2}{\delta\lambda}$. Repr�sentons l'�clairement relatif en fonction de $\delta$ :\par
\centerline{\includegraphics*[width=5cm,height=3cm]{images/sodium.eps}}\par
Etant donn� que le rapport des p�riodes des deux cosinus pr�c�dents est un IP1, localement la fonction est 
``quasi-p�riodique''. On d�finit alors la visibilit� locale au voisinage d'une valeur de $\delta$ comme la premi�re
d�finition, mais en prenant des extr�mas locaux d'�clairement. On trouve ici :\par
$$V = \va{\cos(\pi\dfrac{\delta}{L})}$$\par
$V$ est $L-$p�riodique. La mesure de $L$ avec un interf�rom�tre permet celle de $F$ soit $\Delta\lambda$. On constate
exp�rimentalement que la visibilit� d�croit l�g�rement avec $\delta$. Cela est d� � la largeur non-nulle de chacune
des raies du doublet.
\sspartie{Exemple d'une raie gaussienne :}
\index{Raie gaussienne}
Cette fois la densit� spectrale s'�crit :\par
$$I(\omega) = I_0e^{-\frac{(\omega-\omega_0)}{\Delta\omega}^2}$$\par
de sorte que $F = \frac{\Delta\omega}{\omega_0}\gg 1$. L'�clairement s'exprime avec la formule usuelle. Dans cette
int�grale, on effectue le changement de variables $x = \frac{\omega-\omega_0}{\Delta\omega}$. 
En posant alors $L = c\dfrac{2\pi}{\Delta\omega}$ et $\lambda_0 = 2\pi\dfrac{c}{\omega_0}$, on se ram�ne au calcul
de l'int�grale :\par
$$\integrale{-\frac{\omega_0}{\Delta\omega}}{+\infty}{\left\lbrack 1+\cos\bigl(2\pi\dfrac{\delta}{L}x+2\pi\dfrac{%
\delta}{\lambda_0}\bigr)\right\rbrack e^{-x^2}\dd{x}}$$\\
Or on a suppos� que la finesse �tait grande devant 1. Etant donn� la forte d�croissance de la fonction � int�grer, on
peut raisonnablement supposer que l'int�grale part de $-\infty$. En �crivant le cosinus sous forme exponentielle, 
il reste � calculer deux int�grale du type :\\
$\integrale{-\infty}{+\infty}{e^{-x^2}e^{i a x}\dd{x}}$. En voyant cela comme une fonction $F(a)$, les th�or�mes de 
Lebesgue de d�rivation sous le signe int�grale permettent de trouver une �quation diff�rentielle satisfaite par cette
fonction (apr�s une IPP). On r�sout alors cette �quation et sachant que $F(0) = \sqrt{\pi}$, on trouve :\\
$F(a) = \sqrt{\pi}e^{-\frac{a^2}{4}}$. Il reste � appliquer dans notre cas et on obtient apr�s simplifications :\par
$$\eee = \eee_0\left(1+\cos\bigl(\dfrac{2\pi\delta}{\lambda_0}\bigr)e^{-\pi^2\frac{\delta^2}{L^2}}\right)$$\par
Repr�sentons l'�clairement en fonction de $\delta$ :\par
\centerline{\includegraphics*[width=5cm,height=3cm]{images/gaussienne.eps}}\par
On peut encore d�finir la visibilit� locale et cette fois on trouve :\par
$$V = e^{-\pi^2\frac{\delta^2}{L^2}}$$\par
\sspartie{Int�rf�rences en lumi�re blanche :}
\index{lumiere@Lumi�re blanche}
La lumi�re blanche est mod�lis�e par une densit� nulle en dehors d'un segment et con\-stan\-te sur ce segment. Si on note
$\omega_1$ et $\omega_2$ les bornes inf�rieures et sup�rieures de ce segment et $I_0$ la valeur uniforme de la 
densit� spectrale dans l'intervalle $\segment{\omega_1}{\omega_2}$ et qu'on pose $\lambda_0$ et $L$ comme ci-dessus
($\omega_0$ �tant le milieu du segment), on obtient apr�s calculs :\par
$$\eee = \eee_0\left(1+\sinc(\pi\dfrac{\delta}{L})\cos(2\pi\dfrac{\delta}{\lambda_0})\right)$$\par
En lumi�re blanche, la visibilit� locale n'a aucun sens puisque le rapport des p�riodes est de l'ordre de $1$. De 
plus l'�clairement s'att�nue.\par\index{longueur@Longueur de coh�rence!temporelle}
\rmq On retiendra que $L$ appel� la ``longueur de coh�rence'' vaut $F\lambda_0$ et est telle que la visibilit� est
de l'ordre de $1$ si $\delta$ est n�gligeable devant $L$ et quasi-nulle si c'est le contraire.\\
De plus $L$ est de l'ordre de $1 \mu$m pour de la lumi�re blanche et de $1$ m pour une \textsc{Laser}\par
\sspartie{Interpr�tation microscopique de la coh�rence temporelle :}
\index{coherence@Coh�rence temporelle}
\point{Train d'onde et densit� spectrale :}
La source ponctuelle est constitu�e d'un grand nombre d'atomes qui �mettent de mani�re al�atoire des signaux 
p�riodiques amortis appel�s ``trains d'onde'', de la forme :\par
\centerline{$\begin{array}{lcll}
  f(t) & = & 0                                   & \mathrm{si}\;\; t < 0\\
       & = & ae^{-\frac{t}{\tau}}\cos(\omega_0t) & \mathrm{si}\;\; t > 0
 \end{array}$}\par
La statistique permet de relier la densit� spectrale $I(\omega)$ � la transform�e de Fourier de $f$ et au nombre $N$ 
d'atomes :\par
\encadre{$I(\omega) \propto N \va{F(\omega)}^2$}\par
De plus si $\tau$ d�signe la dur�e du train d'onde et $\Delta\omega$ la largeur spectrale de la source :\par
$$\tau\Delta\omega\sim 1$$
\point{Train d'onde et visibilit� :}
Chaque train d'onde est re�u en $P$ avec un retard diff�rent selon le rayon suivi. Ainsi, les interf�rences sont 
visibles si les trains d'onde se superposent, c'est-�-dire si $\va{\dfrac{\delta}{c}}\ll\tau$.
\index{Train d'onde}
\spartie{Source monochromatique, mais �tendue :}\index{Source!etendue@�tendue}
\sspartie{Flux �mis :}
Une telle source est caract�ris�e par une fonction $\phim_E(M)$ telle que $\phim_E(M)\dd{S}$ repr�sente la valeur 
moyenne de la puissance �lectromagn�tique  �mise par l'�l�ment $(M, \dd{S})$ de la source. ``L'intensit�'' totale
�mise par la source est alors :
\encadre{$\integrale{S}{}\phim_E(M)\dd{S}$}
\sspartie{Expression int�grale de l'�clairement :}
On se limitera aux montages interf�rentiels tels que $a_1 = a_2$ pour lesquels l'�clairement obtenu pour une source
ponctuelle est de la forme $(1+\cos(2\pi\frac{\delta}{\lambda}))\mathcal{I}$ � une constante multiplicative pr�s.
En supposant que la source �tendue peut �tre consid�r�e comme un ensemble de sources ponctuelles �mettant des ondes
incoh�rentes, l'�clairement pour la source �tendue devient :
\encadre{$\eee(P) = \integrale{S}{}{\bigl(1+\cos(2\pi\frac{\delta}{\lambda})\bigr)\phim_E(M)\dd{S}}$}
\rmq La diff�rence de marche $\delta$ d�pend du point $M$ qui parcourt la source �tendue.
\sspartie{Longueur de coh�rence spatiale :}
Un montage interf�rentiel est caract�ris� par sa ``longueur de coh�rence spatiale'' \index{longueur@Longueur de coh�rence!spatiale} qui est telle que :
\encadre{%
$\begin{array}{ll}
V\sim 1 & \;\; si\;\; l\ll L\\
V\sim 0 & \;\; si\;\; l\gg L
\end{array}$}
O\`u $l$ est la dimension caract�risant celles de la source (largeur, rayon).




\include{mecaquantique/hydrogene}
\printindex
%
\end{document}

